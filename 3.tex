\chapter{求函数从$a$到$b$的和与积分}
函数有二个基本性质——变率与和,在前一章,我们研究了函数的瞬时变率的概念,即
\[f'(x_0) =\lim_{h\to 0} \frac{f (x_0+h) -f (x_0)}{h}\]
以及它的应用,在本章我们将研究对于一个给定函数$f(x)$“求从$a$到$b$的和”这个概念。在给出定义之前,我们先举几个实例来看一看。

\paragraph{速率与距离} 
一列行驶中的火车,它的行进速率$v$是时间$t$的函数,即
$v=f(t)$, 如图3.1, 我们可以从速率表读得当时的速率,很自然地,我们想知道火车在从$t=a$到$t=b$这一段时间间隔内一共走了多少路程。从函数的观点看,所谓求在时间$[a,b]$内所走过的距离就是求速率函数$v=f(t)$由$t=a$到$t=b$的和。

\begin{figure}[htp]
    \centering

    \caption{}
\end{figure}

让我们用$D(f,[a,b])$
表示所走过的距离,这个记号强调$D$依赖于$f$和区间$[a,b]$。

\paragraph{变力所作的功}

假定某物体在一个平行于$OX$轴的力$P$的作 用下沿直线
$OX$运动,力$P$的方向与物体运动的方向一致,并且力的大小随离开$O$点的距离而改变,即变力$P$是所在点的横坐标$x$的函数$P=P(x)$, 如图3.2。假定物体在这个变力$P$作用之下,从直线$OX$的一点$a$移到另一点$P$, 那么力$P$所作的功就是变力函数$P(x)$由$x=a$到$x=b$的和,我们用
$W (P, [a,b])$
表示力$P$所作的功,它表明$W$依赖于$P(x)$和$[a,b]$。
\begin{figure}[htp]
    \centering

    \caption{}
\end{figure}

\begin{figure}[htp]
    \centering

    \caption{}
\end{figure}

\paragraph{曲边梯形的面积}

令$y=f(x)$是函数$f$的图象,表示一条曲线,如图3.3. 我们要求曲线上的一段弧$M_0M$ 与其两端的纵坐标线及$x$轴上的线段$[a,b]$所围成的图形的面积。这样的图形(它有三条边是直线,其中两条互相平行,第三条与前两条
互相垂直,而第四条边是曲线)叫做曲边梯形。显然曲边梯形的面积$A$依赖于$y=f(x)$和$x$轴上的线段$[a,b]$, 记这一面积为$A (f, [a,b])$。

在上一章,我们利用函数$f(x)$和它的图象之间的对应关系,就可以把函数的变率和它的图象的切线的斜率相对应地一并分析讨论,这样,一方面可以把函数的变率这种“数量”的概念用切线来形象化,便于想象;而另一方面又可以把“切线”这种几何概念数量化,便于计算,在这一章,我们也要利用函数$f(x)$和它的图象之间的对应关系,把函数$f(x)$由$x=a$到$x=b$的“和”与曲线$y=f(x)$的曲边梯形的“面积”相对应地一并分析讨论,并且说明函数的“求从$a$到$b$的和”恰好对应于求曲边梯形的面积。这也就是为什么把函数“求从$a$到$b$的和”这种基本运算叫做积分的道理。

\section{“和”与“面积”}
对于任意曲线围成的图形,我们还没有规定它的“面积”的意义,和它密切相关的“函数从$a$到$b$的和”的概念至今也没有明确的解析的定义。在这一节我们要把这两个概念由“直观的定性理解”推进到“数量化的定量定义”。唯有确立了它们的“解析的定义”,它们才真正地成为能算好用的量。

\subsection{“和”与“面积”的基本性质}
现在让我们先从“函数的和”与“曲线形的面积”的直观内涵来分析一下它们分别所应有的基本性质。

\subsubsection{“曲线形的面积”的基本性质}

从面积的直观内涵容易看出下列两点:
\begin{enumerate}
\item (单调性)设区域$R_1$包含在$R_2$之内,即$R_1\subseteq R_2$, 则:
$R_1$的面积$\le R_2$的面积.(图3.4)
\item (可加性)设区域$R$可以用一条曲线分割成两块区
域$R_1+R_2$, 则有:
$R$的面积$=R_1$的面积$+R_2$的面积.(图3.5)
\end{enumerate}

\begin{figure}[htp]
    \centering

    \caption{}
\end{figure}

\begin{figure}[htp]
    \centering

    \caption{}
\end{figure}

\subsubsection{“函数的和”的基本性质}

同样地,从“函数从$a$到$b$的和”的直观内涵容易看出下列两个基本性质,即















































