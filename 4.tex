
\chapter{微积分学基本定理}
在前两章中,我们分别引入了函数的变率(导数),函数的和(定积分)的基本概念,本章将研究函数的导函数与函数的求和函数这两者之间的互逆关系,并说明我们可以用求导函数的逆运算方法来计算定积分。

\section{微积分学基本定理}
\subsection{导函数与求和函数}
函数$f(x)$在点$x_0$处的变化率(导数)的定义是
\[f' (x_0) =\lim_{h\to 0}\frac{f (x_0+h) -f (x_0)}{h}\]
显然。$f(x_0)$的值与$f(x)$在点$x_0$的值以及在点$x_0$的邻近的函数值有关,当点$x_0$在$(a,b)$内变化时,$f(x_0)$也跟着变化,那么$f(x_0)$便是一个新函数称为$f(x)$的导函数。计算一个函数的导函数是一件比较简便的事情。

定积分$\int^b_a f(x)\dif x$的定义是把区间$[a,b]$无限细分
而得到上下夹逼阶梯函数的和的共同极限,其几何意义是曲线$y=f(x)$和直线$x=a$, $x=b$, 及$y=0$所围成的区域的有号面积。

假如我们考虑$f(x)$在一个变动的区间$[a,x]$上的和,
即让区间的左端点固定,右端点变动,则
\[S_f(x)=\text{函数$f$从$a$到$x$的和}=\int^x_a f (x) \dif x\]
可以看作上限变量$x$的函数,在这里,积分符号中的$x$既表示积分变量,又表示积分上限,容易混淆,因此,为了区别起见,我们用字母$t$来代表积分变量,这样上式就写成
\[S_f (x)=\int^x_a f(t)\dif t\]

和函数$S_f(x)$在$x=x_0$处的值$S_f(x_0)$的几何意义就是曲线$y=f(t)$, 直线$t=a$, $t=x_0$, $y=0$所围成的区域的有号面积,它是随区域的变动界线$t=x_0$的变动而变动的(图4.1)。



例如,当变动界限(积分上限)在图中的$PM$位置时,则
\[S_f(x_0)=\int^{x_0}_a f(t)\dif t=ACB\text{的面积}-CPM\text{的面积}\]

当变动界限在图中的$P'M'$位置时,则
\[\begin{split}
    S_f(x'_0)&=\int^{x_0'}_a f(t)\dif t=-\int^a_{x_0'} f(t)\dif t\\
    &=-\left\{-P'DM'\text{的面积}+DAB\text{的面积}  \right\}\\
    &=P'DM'\text{的面积}-DAB\text{的面积}
\end{split}\]

例如,折线函数
\[g(x)=\begin{cases}
    \frac{1}{2}(x-1) & x\in[1,2]\\
    \frac{3}{2}(x-2)+\frac{1}{2}(2-1) & x\in [2,3]\\
    (x-3)+\frac{1}{2}(2-1)+\frac{3}{2}(3-2) & x\in [3,4]
\end{cases}\]
的导函数(除去在折线段的那些交接点处不作定义外)是阶梯函数
\[g'(x)=\begin{cases}
    \frac{1}{2} & x\in [1,2)\\
    \frac{3}{2} & x\in (2,3)\\
    1& x\in (3,4]
\end{cases}\]
反过来,该阶梯函数的和函数,是上述折线函数$g(x)$, 我们有如下的图解关系:
\begin{center}
    \begin{tikzpicture}[>=latex]
        \node (A) at (0,0) {\{折线函数\}};
        \node (B) at (5,0) {\{阶梯函数\}};
\draw[->](A) to [bend left=30]node[above]{求导} (B);
\draw[->](B) to [bend left=30]node[above]{求和} (A);

    \end{tikzpicture}
\end{center}

上述简明的例子表明“微分”与“积分”(或求函数由$a$到$b$的和)之间的运算关系应该是互逆的。

\subsection{微积分学基本定理}

\begin{blk}
  {定理1} 设$f(t)$是在$[a,b]$上的分段单调连续函数,又它的和函数是
\[S_f(x)=\int^x_a f (t) \dif t\]
那么
\[\frac{\dif }{\dif x}S_f(x)=\frac{\dif }{\dif x}\int^x_a f (t) \dif t=f(x),\qquad a\le x\le b\]
\end{blk}






