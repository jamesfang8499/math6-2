\documentclass[b5paper, openany]{ctexbook}


\usepackage[margin=2.5cm]{geometry}

\usepackage{physics}
\usepackage{pifont}
\usepackage[perpage,symbol*]{footmisc}
\DefineFNsymbols{circled}{{\ding{192}}{\ding{193}}{\ding{194}}
{\ding{195}}{\ding{196}}{\ding{197}}{\ding{198}}{\ding{199}}{\ding{200}}{\ding{201}}}
\setfnsymbol{circled}



\usepackage{amsmath,amsfonts,mathrsfs,amssymb}
\usepackage{graphicx}

\usepackage[font=bf,labelfont=bf,labelsep=quad]{caption}

\usepackage{tikz, pstricks}
%\usepackage{nicematrix}
\usepackage{bookmark}
\usepackage{ntheorem}
\theoremseparator{\;}



\usepackage{blkarray}
\usepackage{bm}
\usepackage[colorlinks=true, linkcolor=black]{hyperref}

%\usepackage{enumerate}


\theoremstyle{plain}
\theoremheaderfont{\normalfont\bfseries} 
\theorembodyfont{\normalfont}


\usepackage[framemethod=tikz]{mdframed}

\usepackage{circuitikz}

\newtheorem{example}{\bf 例}[chapter]
\newenvironment{solution}{\noindent {\bf 解:}}{}  %{\hfill $\clubsuit$\par}
\newenvironment{analyze}{\noindent {\bf 分析:}}{}
\newenvironment{rmk}{\noindent {\bf 注意:}}{}
\newenvironment{note}{\noindent {\bf 说明:}}{}



\renewcommand{\proofname}{\bf 证明:}
\newenvironment{proof}{{\noindent \bf 证明:}}{}%{\hfill $\square$\par}

\newcommand{\E}{\mathbb{E}}
\renewcommand{\Pr}{\mathbb{P}}
\newcommand{\EP}{\mathbb{E}^{\mathbb{P}}}
\newcommand{\EQ}{\mathbb{E}^{\mathbb{Q}}}
\newcommand{\dif}{\,{\rm d}}
\newcommand{\Var}{{\rm Var}}
\newcommand{\Cov}{{\rm Cov}}
\newcommand{\x}{\times}
\renewcommand{\dd}{\,{\rm d}}

 \usepackage{tcolorbox}
 \tcbuselibrary{breakable}
 \tcbuselibrary{most}



\newtcolorbox{ex}[1][]
  {colback = white, colframe = cyan!75!black, fonttitle = \bfseries,
    colbacktitle = cyan!85!black, enhanced,
    attach boxed title to top center={yshift=-2mm},breakable, 
    title=练习, #1}

\newtcolorbox{blk}[2][]
  {colback = white, colframe = magenta!75!black, fonttitle = \bfseries,
    colbacktitle = magenta!85!black, enhanced,
    attach boxed title to top left={xshift=5mm, yshift=-2mm},breakable, 
    title=#2, #1}


\setcounter{tocdepth}{2}

\setcounter{secnumdepth}{3}



\ctexset {
section = {
	name = {第,节},
 	number = \chinese{section}},
subsection = {
	name = {,、\hspace{-1em}},
	number = \chinese{subsection}
},
subsubsection = {
	name = {(,)\hspace{-1em}},
	number = \chinese{subsubsection},
}
}



\renewcommand{\contentsname}{目~~录}

\newcommand{\poly}{\polynomial[reciprocal]}



\usepackage{mathtools}

\setlength{\abovecaptionskip}{0.cm}
\setlength{\belowcaptionskip}{-0.cm}

\usetikzlibrary{decorations.pathmorphing, patterns}
\usetikzlibrary{calc, patterns, decorations.markings}
\usetikzlibrary{positioning, snakes}

\newcommand{\Lim}{\displaystyle\lim}

\usepackage{yhmath}
\usepackage{longdivision}
\usepackage{polynom}
\usepackage{polynomial}
\usepackage{multicol}

\renewcommand{\frac}{\dfrac}
\newcommand{\oc}{$^{\circ}{\rm C}$}
\usepackage{longtable}
\usepackage{tkz-euclide, tkz-base}
\newcommand{\NC}{\text{N}/\text{C}}
\newcommand{\ms}{\text{m}/\text{s}}
\newcommand{\cms}{\text{cm}/\text{s}}
\newcommand{\msq}{\text{m}/\text{s}^2}
\newcommand{\cmsq}{\text{cm}/\text{s}^2}
\newcommand{\kmh}{\text{km}/\text{h}}
\newcommand{\Int}{\displaystyle\int}
\usepackage{cases}
\begin{document}
%\fontsize{10.5}{11}\selectfont














\title{中学数学实验教材\\第六册}



\author{中学数学实验教材编写组编}
\date{1981年6月}

\maketitle




\frontmatter

%\input{preface.tex}
\tableofcontents


\mainmatter

\chapter{函数的极限和连续函数的性质}

\section{函数的极限}
\subsection{函数极限的概念}

在第四册下,我们研究了数列的极限,数列是一种特殊的函数,这里的自变数$n$取自然数列$1, 2, 3,\ldots,n,\ldots$的值,现在我们来研究更一般的情形,即函数$f(x)$随$x$连续变化而变化的情形,下面转到一般函数的极限。

\begin{blk}{定义1}
    如果$x$通过任何一个无限增大的数列$\{x_n\}$, 对应的函数值数列
$f (x_1 ) ,f (x_2) , \ldots,f (x_n ) ,\ldots$
都以定数$\ell$为它的极限,就说函数$f(x)$, 当$x\to +\infty$时,以$\ell$为极限,记作
\begin{equation}
    \lim_{x\to+\infty}f(x)=\ell\qquad \text{或}\qquad f(x)\to \ell\quad (x\to +\infty)
\end{equation}
\end{blk}
 
从几何上看,极限式(1.1)表示随着$x$无限增大,曲线$y=f(x)$以直线$\ell$为渐近线(图1.1)。
\begin{figure}[htp]
    \centering
    
    \caption{}
\end{figure}

类似地,可以定义函数极限$\Lim_{x\to -\infty} f(x)=\ell$, 这时,变量
$x$通过代数值无限地变小,而绝对值无限地增大的任何一个数列$\{x_n\}$。

如果函数$f(x)$当$x\to+\infty$和$x\to-\infty$时,都以定值为极限,就说$f(x)$当$x\to \infty$时,以定值$\ell$为极限,记作$\Lim_{x\to\infty}f(x)=\ell$, 或者$f(x)\to \ell\; (x\to\infty)$。

\begin{example}
    证明$\Lim_{x\to\infty}\frac{1}{x}=0$
\end{example}

\begin{proof}
任何数列$\{x_n\}$的值$x_1,x_2,\ldots,x_n,\ldots$趋向于$+\infty$或$-\infty$时,对应的函数数列
\[\frac{1}{x_1},\frac{1}{x_2},\ldots, \frac{1}{x_n},\ldots\]
的绝对值$\left|\frac{1}{x_n}\right|$便趋向于零,即
    \[\lim_{n\to\infty}\left|\frac{1}{x_n}\right|=0\]
    从而$\Lim_{x\to\infty}\frac{1}{x}=0$
\end{proof}

\begin{example}
证明函数$\sin x$, 当$x\to\infty$时,没有极限.
\end{example}

\begin{proof}
令自变量$x$取数列$x_n=-\frac{\pi}{2}+2n\pi\quad (n=1,2,3,\ldots)$的值趋向$+\infty$,则对应的函数值数列
\[\begin{split}
    \sin x_n&=\sin\left(-\frac{\pi}{2}+2n\pi\right)\\
&=\sin\left(-\frac{\pi}{2}\right) =-1 \qquad (n=1, 2, 3, \ldots) .
\end{split}\]
恒取定值$-1$, 于是
\[\lim_{n\to\infty} \sin x_n=\lim_{n\to\infty} \sin \left(-\frac{\pi}{2}+2n\pi\right)=-1\]

再令自变量$x$取数列
\[x_n=\frac{\pi}{2}+2n\pi\qquad  (n=1, 2, 3, \ldots )\]
的值趋向于$+\infty$, 则对应的函数值数列
\[\sin x_n =\sin\left(\frac{\pi}{2}+2n\pi\right) =1\qquad  (n=1, 2, 3, \ldots)\]
恒取定值1,于是
\[\lim_{n\to\infty} \sin x_n=\lim_{n\to\infty} \sin\left(\frac{\pi}{2}+2n\pi\right) =1\]
由于$x$取趋向$+\infty$的两个不同数列时,$y=\sin x$可以有不同的极限,因此
$\Lim_{x\to\infty} \sin x$不存在。
\end{proof}

\begin{blk}{定义2}
 设函数$f(x)$在点$a$附近有定义(但在$x=a$时,
可以没有定义),如果当自变量$x$不论通过怎样一个以$a$为极限但始终不等于$a$的数列$\{x_n\}$, 对应的函数值数列$f (x_1) ,f (x_2) , \ldots,f(x_n ) ,\ldots$
总有极限$\ell$, 就说:

当$x$趋近于$a$时,$f(x)$以$\ell$为极限,记作
\begin{equation}
   \lim_{x\to a}f(x)=\ell,\qquad \text{或}\qquad f(x)\to \ell\quad (x\to a) 
\end{equation}
\end{blk}

极限式(1.2)的几何意义如图1.2所示:当$x$无限地靠近$a$,但总不能等于$a$时,曲线$y=f(x)$上的点$(x,f(x))$无限地近$(a,\ell)$点.

\begin{figure}[htp]
    \centering
    
    \caption{}
\end{figure}

初学的人常常要问:为什么在定义中谈到$x$趋近于$a$时,要限制$x$始终不等于$a$呢?这是因为我们关心的是函数$f(x)$在$a$附近的变化趋势,它和函数$f(x)$在
$x=a$这一点的值没有什么必然关系,这也就是说,无论$f(x)$在点$a$取什么值甚至没有定义,都不影响在这一点的极限的存在和极限值。

\begin{example}
设$f(x)=\frac{3}{4}\cdot \frac{x^2-1}{x-1}$,$x\in(-\infty,1)\cup(1,+\infty)$,求$\Lim_{x\to 1}f(x)$
\end{example}

\begin{solution}
    $f(x)=\frac{3}{4}\cdot \frac{x^2-1}{x-1}$在$x=1$时无意义,因为那时
    分母就变成零,因此,这里没有函数值$f(1)$, 曲线$y=f(x)$也没有相应于横坐标为1的那个点,但是让$x$任意地趋近于1是完全可以的,若$x\ne 1$, 则有
  \[  f (x) =\frac{3}{4}\cdot  \frac{(x-1)(x+1)}{x-1}=\frac{3}{4}(x+1)\]
  因此,不论$x$通过怎样一个以1为极限的数列$\{x_n\}$, 对于相应的数列$\{f(x_n)\}$, 我们都有
  \[\lim_{x_n\to 1} f (x_n) =\frac{3}{4} (1+1) =\frac{3}{2}\]

    从几何上看,曲线$y=f(x)$除去点$\left(1,1\frac{1}{2}\right)$外是与直线$y=\frac{3}{4}(x+1)$一致的,唯独在那一点,曲线有个空隙,而在$x=1$的邻近的点只要充分接近于点1, 所对应的函数值与$\frac{3}{2}$的差的绝对值可以任意小,如图1.3。
\end{solution}
    
\begin{figure}[htp]
    \centering
\begin{tikzpicture}[>=latex]
    \draw[->](-3,0)--(5,0)node[right]{$x$};
    \draw[->](0,-1.5)--(0,4.5)node[right]{$y$};
\foreach \x in {1,2,3}
{
    \draw(0,\x)node[left]{$\x$}--(.1,\x);
    \draw(\x,0)node[below]{$\x$}--(\x,.1);
}
\node at (.25,-.25){$O$};
\draw[dashed](0,1.5)--(1,1.5)--(1,0);
\draw[domain=-2:4, samples=10, very thick]plot(\x, {0.75*(\x+1)});
\draw(1,1.5)[fill=white]  circle(1.5pt) node[right]{$\left(1,1\tfrac{1}{2}\right)$};
\node at (3.25,3)[right]{$y=\frac{3}{4}\cdot \frac{x^2-1}{x-1}$};
\end{tikzpicture}
    \caption{}
\end{figure}

\begin{example}
证明当$x\to 0$时,函数$f(x)=\sin\frac{1}{x}$没有极限。
\end{example}

\begin{proof}
函数$f(x)=\sin\frac{1}{x}$对于一切$x\ne 0$的值有定义,因此这个函数在点$x=0$的领域内有定义。

当$x$取数列$\{x_n\}=\left\{\frac{2}{(2n+1)\pi}\Big| n=1,2,3,\ldots\right\}$的值而趋于零时,数列$\left\{\frac{1}{x_n}\right\}$相应的值是
\[\frac{3\pi}{2},\frac{5\pi}{2},\frac{7\pi}{2},\ldots,(2n+1)\frac{\pi}{2},\ldots\]
此时数列$\left\{\sin\frac{1}{x_n}\right\}$便交替地取$-1$和$+1$这两个数值,换言之
\[\sin\frac{1}{x_n}=(-1)^n,\qquad n=1,2,3,\ldots\]
因此,当$n\to\infty$时,数列$\left\{\sin\frac{1}{x_n}\right\}$不趋于任何极限值。这就证明了当$x\to 0$时,函数$f(x)=\sin\frac{1}{x}$的极限不存在.
\end{proof}

函数的图象大致如图1.4所示。曲线关于原点对称,在包含原点的每一个对称邻域$(-\delta, \delta)$内,曲线$y=\sin\frac{1}{x}$
在原点的邻近作无数次振动,且曲线的振幅恒为1, 虽将原点的邻域的长缩小,也不能减少振动的次数。

\begin{figure}[htp]
    \centering
\begin{tikzpicture}[>=latex]
    \draw[->](-4,0)--(4,0)node[right]{$x$};
    \draw[->](0,-2)--(0,2)node[right]{$y$};
    \foreach \x in {1,-1}
    {
        \draw[dashed](-4,\x)--(4,\x)node[right]{$y=\x$};
    }
\draw[domain=-4:-.05, samples=1000, thick]plot(\x, {sin(180/pi/\x)});
\draw[domain=.05:4, samples=1000, thick]plot(\x, {sin(180/pi/\x)});
\foreach \x/\xtext in {1/1, .637/\frac{2}{\pi}}
{
    \draw(\x, 0)node[below]{$\xtext$}--(\x,.1);
}
\end{tikzpicture}    
    \caption{}
\end{figure}

上面是用数列的极限来说明函数的极限,其实我们也可以直接定义函数极限。

\begin{blk}{定义3}
  如果函数$f$在点$a$邻域上有定义(可能去掉点$a$本身),使得当$0<|x-a|<\delta$时,就有$|f(x)-\ell|<\varepsilon$, 那么就说$\ell$为当$x$趋近于$a$时,函数$f$在点$a$的极限值。
\end{blk}

我们对这个定义需要说明以下几点:
\begin{enumerate}
    \item 用定义3验证某数$\ell$是函数$f$在点$a$的极限的办法就是对于任给的$\varepsilon>0$, 要找到这样的正数$\delta$使得能够由不等式$|x-a|<\delta$推出不等式$|f(x)-\ell|<\varepsilon$,虽然$\varepsilon$是任意的正数,但是在找$\delta$的过程中,$\varepsilon$是固定不变的,$\delta$依赖于$\varepsilon$。
    \item 对于已给的$\varepsilon$, 只要证明有一个$\delta>0$存在就行.因为如果有一个$\delta$存在,把$\delta$再缩小一些,显然仍满足我们的要求。
    \item 不等式$|x-a|>0$只是说明$x\ne a$, 即把$x$等于$a$的情况去掉,这是因为我们关心的是函数$f$在点$a$附近的变化趋势,而和函数在$x=a$这点的值无关。
    \item 我们指出定义2和定义3是等价的.
\end{enumerate}

\begin{example}
用定义3证明$\Lim_{x\to 1}\frac{x^3-1}{x-1}=3$
\end{example}

\begin{proof}
    任给$\varepsilon>0$, 要找$\delta>0$, 使由$0<|x-1|<\delta$推出
    $\left|\frac{x^3-1}{x-1}-3\right|<\varepsilon$成立。

    当$x\ne 1$时,
\[\begin{split}
    \left|\frac{x^3-1}{x-1}-3\right|&=|(x^2+x+1)-3|\\
    &=|(x-1)(x+2)|=|x-1|\cdot |x-2|
\end{split}\]

要由$|x-1|\cdot |x+2|<\varepsilon$找$\delta$,显然,这里因子$|x+2|$引起了麻烦。为方便起见,先假定$0<|x-1|<1$,即取$\delta_1=1$,这样
\[0<|x-1|<1\quad\to \quad |x+2|=|(x-1)+3|\le |x-1|+3<4\]因此,要使
\[\begin{cases}
    0<|x-1|<1\\
    |x-1|\cdot |x+2|<\varepsilon
\end{cases}\]
只须
\[\begin{cases}
    0<|x-1|<1\\
    4|x-1|<\varepsilon
\end{cases}\to \quad \begin{cases}
    0<|x-1|<1\\
    |x-1|<\frac{\varepsilon}{4}
\end{cases}\]
由此可见,只须取$\delta=\min\left(1,\frac{\varepsilon}{4}\right)$,即取$\delta$为1与$\frac{\varepsilon}{4}$中的较小者。

$\therefore\quad $对于任意$\varepsilon>0$,取$\delta=\min\left(1,\frac{\varepsilon}{4}\right)$,则当$0<|x-1|<\delta$时,即有
\[\left|\frac{x^3-1}{x-1}-3\right|<\varepsilon\]
这也就证明了
\[\lim_{x\to 1}\frac{x^3-1}{x-1}=3\]
\end{proof}

\begin{example}
    证明$\Lim_{x\to a}\sqrt{x}=\sqrt{a}\quad (a>0)$
\end{example}

\begin{proof}
对于任意的$\varepsilon>0$, 我们必须找到一个$\delta>0$, 使得当$|x-a|<\delta$时,$|\sqrt{x}-\sqrt{a}|<\varepsilon$成立,因为
\[|\sqrt{x}-\sqrt{a}|=\frac{|x-a|}{\sqrt{x}+\sqrt{a}}<\frac{|x-a|}{\sqrt{a}}\]
所以要使$\frac{|x-a|}{\sqrt{a}}<\varepsilon$,只须$|x-a|<\sqrt{a}\varepsilon$

如果取$\delta=\sqrt{a}\varepsilon$,则$\frac{|x-a|}{\sqrt{a}}<\frac{\sqrt{a}\varepsilon}{\sqrt{a}}=\varepsilon$
因此,对于任意$\varepsilon>0$,取$\delta=\sqrt{a}\varepsilon$,则当$|x-a|<\delta$时,就有$|\sqrt{x}-\sqrt{a}|<\varepsilon$成立。这也就证明了
\[\lim_{x\to a}\sqrt{x}=\sqrt{a}\quad (a>0)\]
\end{proof}

有时虽然$f(x)$在某点的左边(或右边)没有定义,如图1.5中的$a,b$两点,我们也可以谈论$f$在点$a$或点$b$两点的极限,譬如对于所有小于$a$的数,$f$虽然没有定义,但是我们可以考察,当$x$从$a$的右侧趋近于$a$时,函数$f$的变化趋势,也就是考察$f$的单边极限是否存在。

\begin{figure}[htp]
    \centering
\begin{tikzpicture}[>=latex]
    \draw[->](-1,0)--(3,0)node[right]{$x$};
    \draw[->](0,-1)--(0,2.5)node[right]{$y$};
\draw[very thick](.5,1)to[bend right=-15]node[above]{$y=f(x)$} (2.5,1.5);
\draw(.5,1)--(.5,0)node[below]{$a$};
\draw(2.5,1.5)--(2.5,0)node[below]{$b$};
\node at (-.25,-.25){$O$};
\end{tikzpicture}
    \caption{}
\end{figure}

\begin{blk}{定义}
     设$f(x)$在区间$(a,b)$上有意义,如果任给$\varepsilon>0$, 总存在某个$\delta>0$使得当$x\in (a,a+\delta)$时,总有$|f(x)-\ell|<\varepsilon$,我们就说函数$f$在点$a$以$\ell$为\textbf{右极限},记作:
\[\lim_{x\to a^+}f(x)=\ell\]
\end{blk}

类似地,可以定义\textbf{左极限},只需把开区间$(a,a+\delta)$换成$(b-\delta,b)$就行了,并记作
\[\lim_{x\to b^-} f(x)=\ell\]

\begin{example}
    设函数$f(x)=\begin{cases}
        x-1,&x\le 1\\
x+1,&x>1
    \end{cases}\quad $    
    求$\Lim_{x\to 1^-}f(x)$和$\Lim_{x\to 1^+}f(x)$
\end{example}

\begin{figure}[htp]
    \centering
    \begin{minipage}[t]{0.48\textwidth}
    \centering
    \begin{tikzpicture}[>=stealth, scale=.6]
        \draw[->](-3,0)--(4,0)node[right]{$x$};
        \draw[->](0,-3)--(0,4.5)node[right]{$y$};
        \draw(0,1)node[left]{1}--(.1,1);
        \draw(0,2)node[left]{2}--(.1,2);
        \draw(1,0)node[below]{1}--(1,.1);
        \node at (-.25,-.25){$O$};
        \draw[domain=-2:1, samples=10, very thick]plot(\x,{\x-1});
        \draw[domain=1:3, samples=10, very thick]plot(\x,{\x+1});
    \node at (-2,-3)[left]{$y=x-1$};
    \node at (3,4)[right]{$y=x+1$};
    \draw[dashed](1,0)--(1,2)--(0,2);
    \draw (1,2)[fill=white] circle(2.5pt);
    
    \end{tikzpicture}
    \caption{}
    \end{minipage}
    \begin{minipage}[t]{0.48\textwidth}
    \centering
    \begin{tikzpicture}[>=stealth, scale=.8]
        \draw[->](-3,0)--(4,0)node[right]{$x$};
    \draw[->](0,-2.5)--(0,3)node[right]{$y$};
    \foreach \x in {-2,-1,1,2,3}
    {
        \draw(\x,0)node[below]{$\x$}--(\x,.1);
    }
\foreach \x in {-2,-1,1,2}
{
    \draw(0,\x)--(.1,\x)node[right]{$\x$};
}
\foreach \x in {-2,-1,...,2}
{
    \draw[very thick](\x,\x)--(\x+1,\x);
    \draw (\x+1,\x)[fill=white]circle(2pt);
}
    \node at (.25,-.25){$O$};
    \end{tikzpicture}
    \caption{}
    \end{minipage}
    \end{figure}


\begin{solution}
\[\begin{split}
    \lim_{x\to 1^-}f(x)&=\lim_{x\to 1^-}(x-1)=0\\
    \lim_{x\to 1^+}f(x)&=\lim_{x\to 1^+}(x+1)=2
\end{split}\]
图1.6显示了上面的结果。
\end{solution}

\begin{example}
    函数$y=[x]$代表不超过$x$的最大整数,即若$n\le x<n+1$,$n\in\mathbb{Z}$,则$y=[x]=n$。(图1.7)

求$\Lim_{x\to 2^+}\frac{[x]}{x},\quad \Lim_{x\to 2^-}\frac{[x]}{x}$
\end{example}

\begin{solution}
    显然,当$x=2$时,$\frac{[x]}{x}=\frac{[2]}{2}=\frac{2}{2}=1$, 又
\[\frac{[x]}{x}=\begin{cases}
    \frac{1}{x},& x\in(1,2)\\
    \frac{2}{x},& x\in(2,3)
\end{cases}\]
所以
\[\begin{split}
    \lim_{x\to 2^+}\frac{[x]}{x}&=\lim_{x\to 2^+}\frac{2}{x}=1\\
     \lim_{x\to 2^-}\frac{[x]}{x}&=\lim_{x\to 2^-}\frac{1}{x}=\frac{1}{2}
\end{split}\]    
\end{solution}


下面的命题说明函数的极限与函数的单边极限的关系:

\begin{blk}{定理}
极限$\Lim_{x\to a} f(x)$
存在的必要和充分条件是左极限$\Lim_{x\to a^-} f(x)$和右极限$\Lim_{x\to a^+} f(x)$都存在,并且二者相等。
\end{blk}

\begin{proof}
    必要性
    
    如果$\Lim_{x\to a} f(x)=\ell$, 就是说任给$\varepsilon>0$, 总存在$\delta>0$,
使得当$0<|x-a|<\delta$, 即当$x\in (a-\delta,a)\cup(a,a+\delta)$时,有$|f(x)-\ell|<\varepsilon$。换言之,当$x\in (a-\delta,a)$和$x\in (a,a+\delta)$时,都有$|f (x) -\ell|<\varepsilon$,因此
\[\Lim_{x\to a^-} f(x)=\ell,\qquad \Lim_{x\to a^+} f(x)=\ell\]

充分性

如果$\Lim_{x\to a^-} f(x)=\ell$且$\Lim_{x\to a^+} f(x)=\ell$, 那么总存在$\delta_1>0$, 使得当$x\in(a-\delta_1,a)$时,有
$|f (x) -\ell|<\varepsilon$。

又存在$\delta_2>0$, 使得当$x\in (a,a+\delta_2)$时,有$|f (x) -\ell|<\varepsilon$。

取$\delta=\min(\delta_1,\delta_2)$,于是当$x\in(a-\delta,a+\delta)$时,有$|f (x) -\ell|<\varepsilon$。
这就是说:
\[\Lim_{x\to a} f(x)=\ell\]
\end{proof}

\begin{example}
说明$\Lim_{x\to 3}\frac{|x-3|}{x-3}$是否存在?
\end{example}

\begin{solution}
\[\begin{split}|x-3|&=\begin{cases}
    x-3, & x>3\\
    3-x, &x<3
\end{cases}\\
    \Lim_{x\to 3^-}\frac{|x-3|}{x-3}&=\Lim_{x\to 3^-}\frac{3-x}{x-3}=\Lim_{x\to 3^-}(-1)=-1\\
    \Lim_{x\to 3^+}\frac{|x-3|}{x-3}&=\Lim_{x\to 3^+}\frac{x-3}{x-3}=\Lim_{x\to 3^+}(1)=1
\end{split}\]

$\because\quad \Lim_{x\to 3^-}\frac{|x-3|}{x-3}\ne \Lim_{x\to 3^+}\frac{|x-3|}{x-3}$
    
$\therefore\quad \Lim_{x\to 3}\frac{|x-3|}{x-3}$
不存在。
\end{solution}

\subsection{函数值趋于无穷大}

如果函数$f$在点$a$的邻域上有定义(可能去掉点$a$本身)对于无论多么大的正数$G$, 总存在一个够小的正数$\delta$, 使得当$0<|x-a|<\delta$时,就有$|f(x)|>G$, 那么就说当$x$趋于$a$时,函数$f(x)$趋于\textbf{无穷大},记作
\[\lim_{x\to a} f (x) =\infty\]

\begin{example}
    求证$\Lim_{x\to 0}\frac{1}{x}=\infty$
\end{example}
   
\begin{proof}
设$G$是任意给定的正数,我们要求出一个$\delta>0$, 使得当$|x|<\delta$时,$|f(x)|=\left|\frac{1}{x}\right|>G$。

事实上,要使$\frac{1}{|x|}>G$, 
只须$0<|x|<\frac{1}{G} $。取$\delta=\frac{1}{G}$, 于是当$|x|<\delta$时,就有$\frac{1}{|x|}>G$
因此,
$$\Lim_{x\to 0}\frac{1}{x}=\infty$$
\end{proof}

\begin{example}
    证明当$x\to 0$时,函数$f(x)=\frac{1}{x}\sin\frac{1}{x}\quad (x\ne 0)$不趋于无穷大.
\end{example}

\begin{proof}
如果自变量$x$取数列$\{x_n\}=\left\{\frac{2}{(2n+1)\pi}\Big|n=1, 2, 3,\ldots\right\}$的值趋于0时,$\sin\frac{1}{x}$在原点的任意邻域内无限次交替地取$-1, 1$这两个值,对于这些值,$|f(x)|=\frac{1}{x}=\frac{\pi}{2}(2n+1)$趋于无穷大.

但是当$x$取数列$\{x\}=\left\{\frac{1}{n\pi}\Big| n=1, 2, 3,\ldots\right\}$的
值趋于0时,由于
$\sin\frac{1}{x}=\sin (n\pi) =0$, 
故对于这些值,
$\Lim_{x'_n\to 0} f (x) =0$.

可见在原点的邻近不存在这样的$\delta>0$, 使得当$|x|<\delta$时,$|f(x)|>G$, 因此,当$x\to 0$时,$f(x)=\frac{1}{x}\sin\frac{1}{x}\quad (x\ne 0)$不趋于无穷大.

$y=f(x)$的图象位于两条双曲线$xy=\pm 1$之间,且在原点的邻近作无限多次振动,越靠近原点,曲线的振幅越大(图1.8).

\begin{figure}[htp]
    \centering
\begin{tikzpicture}[>=latex, scale=.8]
\draw[->](-5,0)--(5,0)node[right]{$x$};
\draw[->](0,-4)--(0,4)node[right]{$y$};
\draw[domain=-4.5:-.3, samples=100]plot(\x,{-1/\x}); 
\draw[domain=-4.5:-.3, samples=100]plot(\x,{1/\x}); 
\draw[domain=.3:4.5, samples=100]plot(\x,{-1/\x}); 
\draw[domain=.3:4.5, samples=100]plot(\x,{1/\x}); 
\draw [domain=.5:4.5, samples=200,  thick]plot(\x, {sin(180/\x*pi)/\x});
\draw [domain=.5:4.5, samples=200,  thick]plot(-\x, {sin(180/\x*pi)/\x});
\end{tikzpicture}
    \caption{}
\end{figure}

\end{proof}

如果对于任何$G>0$, 存在$\delta>0$, 当$0<|x-a|<\delta$时,有$f(x)>G$, 就说当$x\to a$时,函数$f(x)$趋于正无无穷大,记作
\[\lim_{x\to a}f(x)=+\infty\]

如果对于任何$G>0$, 存在$\delta>0$, 当$0<|x-a|<\delta$时,有$f(x)<-G$, 就说当$x\to a$时,函数$f(x)$趋于负无穷大,记作
\[\lim_{x\to a}f(x)=-\infty\]

例如,我们有:
\[\lim_{x\to 0}\frac{1}{x^2}=+\infty,\qquad \lim_{x\to 0}\frac{(-1)}{x^2}=-\infty\]

类似地,我们可以定义:
\[\lim_{x\to a^-}f(x)=+\infty,\quad \lim_{x\to a^+}f(x)=+\infty,\quad \lim_{x\to a^-}f(x)=-\infty,\quad \lim_{x\to a^+}f(x)=-\infty\]
的含义,这里不再写出。建议读者将这些定义严格地写出来。

例如,我们有:
\[\lim_{\theta\to \tfrac{\pi^+}{2}}\tan\theta =+\infty,\qquad \lim_{\theta\to \tfrac{\pi^-}{2}}\tan\theta =-\infty\]
\[\lim_{x\to 0^+}\log_a x =-\infty\quad (a>1),\qquad \lim_{x\to 0^+}\log_a x =+\infty\quad (0<a<1)\]

\begin{ex}
\begin{enumerate}
    \item 用函数极限定义证明:
    \begin{multicols}{2}
        \begin{enumerate}
            \item $\Lim_{x\to \infty}\frac{1}{2x+1}=0$
            \item $\Lim_{x\to 2}x^2=4$
            \item $\Lim_{x\to -1}\frac{x-3}{x^2-9}=\frac{1}{2}$
            \item $\Lim_{x\to 1}\frac{1}{x^2}=1$
            \item $\Lim_{x\to 1}\frac{x^3-x}{x-1}=2$
        \end{enumerate}
    \end{multicols}
    \item 说明:$\Lim_{x\to 3}\frac{x}{x^2-9}=\infty$
    \item 下面极限是否存在?
\begin{multicols}{2}
    \begin{enumerate}
        \item $\Lim_{x\to 1}\frac{2x|x-1|}{x-1}$
        \item $\Lim_{x\to 3}\frac{[x]^2-9}{x^2-9}$
    \end{enumerate}
\end{multicols}
\end{enumerate}
\end{ex}

\subsection{函数极限算法定理}
函数的极限算法定理与数列的极限算法定理类似,因为所谓$\Lim_{x\to a}u(x)=A$, $\Lim_{x\to a}v(x)=B$的意思就是对于任何一个各项都不同于$a$并且以$a$为极限的数列$x_n\to a$, 便有函数值数列$\{u(x_n)\}$, $\{v(x_n)\}$, 并且$\Lim_{n\to \infty}u(x_n)=A$, 
$\Lim_{n\to \infty} v(x_n)=B$, 因此,根据第四册下第三章的定理就可以直接得到相应的结果。现在给出函数的极限运算定理如下:

\begin{blk}{定理}
设$\Lim_{x\to a}u(x)=A$, $\Lim_{x\to a}v(x)=B$,那么
\begin{enumerate}
    \item $\Lim_{x\to a}[u(x)+v(x)]=\Lim_{x\to a}u(x)+\Lim_{x\to a}v(x)$
    \item $\Lim_{x\to a}[u(x)\cdot v(x)]=\Lim_{x\to a} u(x)\cdot \Lim_{x\to a}v(x)$
    \item $\Lim_{x\to a}[c\cdot u(x)]=c\cdot \Lim_{x\to a}u(x)$
    \item $\Lim_{x\to a}\frac{u(x)}{v(x)}=\frac{\Lim_{x\to a}u(x)}{\Lim_{x\to a}v(x)}$ 
    
    只要$v(x)$恒不为0,而且$\Lim_{x\to a}v(x)\ne 0$
    \item $\Lim_{x\to a}\sqrt[n]{u(x)}=\sqrt[n]{\Lim_{x\to a}u(x)}$
    \item 如$\Lim_{x\to a}u(x)=A=\Lim_{x\to a} v(x)$,而$|x-a|<\delta$时,$u(x)<f(x)<v(x)$,则:
    \[\Lim_{x\to a} f(x)=A\]
    即:如果$u(x)$与$v(x)$趋向同一极限$A$,且$f(x)$在$u(x)$与$v(x)$之间,那么,$f(x)$便也趋向那个极限$A$。
\end{enumerate}
\end{blk}

现在只有5需要补证.

设$\Lim_{x\to a}u(x)=A>0$, 则根据函数极限定义:对于$\varepsilon=\frac{A}{2}>0$, 存在$\delta>0$,使得$0<|x-a|<\delta$时,有
\begin{equation}
    u(x)>\frac{A}{2}>0
\end{equation}

于是,
\[\sqrt[n]{u(x)}-\sqrt[n]{A}=\frac{u(x)-A}{\sum^n_{k=1}[u(x)]^{\tfrac{n-k}{n}}A^{\tfrac{k-1}{n}}}\]
由(1.3)式
\[[u(x)]^{\tfrac{n-k}{n}}>\left(\frac{A}{2}\right)^{\tfrac{n-k}{n}}\]
故
\[\begin{split}
    \sum^n_{k=1}[u(x)]^{\tfrac{n-k}{n}}A^{\tfrac{k-1}{n}}&>\sum^n_{k=1}\left(\frac{A}{2}\right)^{\tfrac{n-k}{n}}A^{\tfrac{k-1}{n}}\\
    &=A^{\tfrac{n-1}{n}}\sum^n_{k=1}\left(\frac{1}{2}\right)^{\tfrac{n-k}{n}}\\
    &=A^{\tfrac{n-1}{n}}\left\{\left(\frac{1}{2}\right)^{\tfrac{n-1}{n}}+\left(\frac{1}{2}\right)^{\tfrac{n-2}{n}}+\cdots+\left(\frac{1}{2}\right)^{\tfrac{1}{n}}+1\right\}\\
    &=A^{\tfrac{n-1}{n}}\left\{\frac{1-\left(\frac{1}{2}\right)^{\tfrac{n-1}{n}}\cdot \left(\frac{1}{2}\right)^{\tfrac{1}{n}}}{1-\left(\frac{1}{2}\right)^{\tfrac{1}{n}}}\right\}\\
    &=A^{\tfrac{n-1}{n}}\cdot\frac{\frac{1}{2}}{1-\left(\frac{1}{2}\right)^{\tfrac{1}{n}}}=\frac{A^{\tfrac{n-1}{n}}}{2-2^{\tfrac{n-1}{n}}}\\
\end{split}\]
$\therefore\quad \left|\sqrt[n]{u(x)}-\sqrt[n]{A}\right|<|u(x)-A|\cdot \frac{2-2^{\tfrac{n-1}{n}}}{A^{\tfrac{n-1}{n}}}$

由题设$\Lim_{x\to a}u(x)=A$,而$\frac{2-2^{\tfrac{n-1}{n}}}{A^{\tfrac{n-1}{n}}}$是一个与$x$无关的常数,所以
\[\lim_{x\to a} \sqrt[n]{u(x)}=\sqrt[n]{A}=\sqrt[n]{\lim_{x\to a}u(x)}\]

下面我们来证明两个重要极限公式,为此先介绍一个引理。

\begin{blk}{引理}
     对于任意实数$\theta$, 都有$|\sin\theta|\le |\theta|$
\end{blk}

\begin{proof}
  由于对于任意实数。有$|\sin\theta|\le 1$, 那么仅考虑$\theta\in (0,\pi/2)$即可。
  
  在单位圆$O$上(图1.9),
 截取弧$\wideparen{P_0M}$使弧长$|\wideparen{P_0M}|$等于$\theta$, 其中$P_0$和$M$分别有坐标$(1, 0)$和$(x,y)$, 于是
 \[\sin\theta =y<\sqrt{y^2+ (1-x)^2} =|P_0M|<|\wideparen{P_0M}|=\theta\]
 
 显然,当$\theta=0$时,有$\sin\theta=\theta$. 因此
\[  |\sin\theta | \le |\theta|,\qquad  \theta\in  (0,\pi/2)\]

  如果$-\pi/2<\theta<0$, 则仍有$|MN|<\wideparen{P_0M}$,于是
\[|\sin\theta|<|\theta|,\qquad \theta\in (-\pi/2, 0) \]

  如果$|\theta|\ge \pi/2$, 则因为$\pi/2>1$与$|\sin\theta|\le 1$, 而同样地也得到$|\sin\theta|<|\theta|$. 
  
  因此,对于一切$\theta\ne 0$的值,有$|\sin\theta|<|\theta|$; 当$\theta=0$时,有$|\sin\theta|=|\theta|$。
\end{proof}



\begin{figure}[htp]\centering
    \begin{minipage}[t]{0.48\textwidth}
    \centering
\begin{tikzpicture}[>=latex]
\draw[->](-2.5,0)--(2.5,0)node[right]{$x$};
\draw[->](0,-2)--(0,2.5)node[right]{$y$};
\draw(0,0) circle(1.5);
\draw[very thick](0,0)--(45:1.5)node[above right]{$M(x,y)$}--(1.5,0)node[below right]{$P_0(1,0)$};
\draw[dashed](1.5/1.414,0)node[below]{$N$}--(1.5/1.414,1.5/1.414);
\node at (0,0)[below left]{$O$};
\end{tikzpicture}    
    \caption{}
    \end{minipage}
    \begin{minipage}[t]{0.48\textwidth}
    \centering
    \begin{tikzpicture}[>=latex, scale=1.3]

\draw[->](-1.5,0)--(2.5,0)node[right]{$x$};
\draw[->](0,-1.5)--(0,2)node[right]{$y$};
\draw(0,0) circle(1);
\draw[very thick](0,0)--(60:2)node[above right]{$C$}--(1,0)node[below right]{$A(1,0)$};
\draw[thick](.5,0)node[below]{$B$}--(60:1)node[above]{$D$};
\node at (0,0)[below left]{$O$};   
\draw(0.2,0) arc (0:60:.2)node[right]{$\theta$};   
    \end{tikzpicture}
    \caption{}
    \end{minipage}
    \end{figure}

\begin{blk}{定理}
\[\lim_{\theta\to 0}\frac{\sin\theta }{\theta}=1,\qquad \lim_{\theta\to 0}\frac{\cos\theta-1}{\theta}=0\]
\end{blk}

\begin{proof}
    让我们先证明$\Lim_{\theta\to 0}\frac{\sin\theta }{\theta}=1$。由图1.10容易看出:

假定$\theta$的单位是弧度,于是:
\[\begin{split}
    \triangle OBD\text{的面积}&=\frac{1}{2}OB\cdot BD=\frac{1}{2}\cos\theta\cdot \sin\theta\\
    \triangle OAC\text{的面积}&=\frac{1}{2}OA\cdot AC=\frac{1}{2}\cdot 1\cdot \tan\theta\\
    \text{扇形$OAD$的面积}&=\frac{1}{2}OA\cdot \wideparen{AD}=\frac{1}{2}\cdot 1^2\cdot \theta
\end{split}\]

假如角是用180等分平角的“度”作为单位,则扇形面积就是$\frac{1}{2}\cdot 1\cdot \frac{\pi}{100}\theta$。

因为上述扇形是夹在$\triangle OBD$和$\triangle OAC$之间,所以
\begin{equation}
    \frac{1}{2}\cos\theta\sin\theta<\frac{1}{2}\theta<\frac{1}{2}\tan\theta=\frac{1}{2}\frac{\sin\theta}{\cos\theta}
\end{equation}
即有,若$0<\theta<\frac{\pi}{2}$,则
\begin{equation}
    \cos\theta<\frac{\sin\theta}{\theta}<\frac{1}{\cos\theta}
\end{equation}
注意到:$\sin(-\theta)=-\sin\theta$和$\cos(-\theta)=\cos\theta$, 不等式(1.5)也蕴含,若$-\frac{\pi}{2}<\theta<0$, 则
\begin{equation}
    \cos(-\theta)<\frac{\sin(-\theta)}{-\theta}<\frac{1}{\cos(-\theta)}
\end{equation}
将不等式(1.5)和(1.6)合并为,若$0<|\theta|<\frac{\pi}{2}$,则
\begin{equation}
    \cos\theta<\frac{\sin\theta}{\theta}<\frac{1}{\cos\theta}
\end{equation}
又因为
\[0\le |1-\cos\theta|=\left|2\sin^2\frac{\theta}{2}\right|\le 2\left|\sin\frac{\theta}{2}\right|<|\theta|\]
从而
\[0\le \lim_{\theta\to 0}|1-\cos\theta|\le \lim_{\theta\to 0}|\theta|=0\]
所以
\[\lim_{\theta\to 0}|1-\cos\theta|=0\]
即:$\Lim_{\theta\to 0}\cos\theta=1$
由(1.7)知被夹逼在两者之间的$\frac{\sin\theta}{\theta}$极限值也就一定是1了!

以$\Lim_{\theta\to 0}\frac{\sin\theta}{\theta}=1$为基础,则:
\[\begin{split}
\lim_{\theta\to 0}\frac{\cos\theta-1}{\theta}&=\lim_{\theta\to 0}\left\{\frac{-2\left(\sin\frac{\theta}{2}\right)^2}{\theta}\right\}\\
&= \lim_{\theta\to 0}\left(-\sin\frac{\theta}{2}\right)\frac{\sin\frac{\theta}{2}}{\frac{\theta}{2}}    \\
&=\lim_{\theta\to 0}\left(-\sin\frac{\theta}{2}\right)\cdot \lim_{\theta\to 0}\frac{\sin\frac{\theta}{2}}{\frac{\theta}{2}}   \\
&=0\cdot 1=0
\end{split}   
\]
\end{proof}

\begin{example}
求:$\Lim_{x\to 2}\frac{x^2+3x-10}{3x^2-5x-2},\qquad \Lim_{x\to 0}\frac{\sqrt[3]{1+x^2}-1}{x^2}$
\end{example}

\begin{solution}
    如果应用前面的定理去分别求两式中的分子、分母的极限时,其分子、分母的极限值均为零,即我们得到$\frac{0}{0}$的不定形式。我们通常要对原代数式变形,找出分子、分母中具有$x-a$的因子,消去$x-a$的公共因子,一般问题就可解决。
\[\begin{split}
    \Lim_{x\to 2}\frac{x^2+3x-10}{3x^2-5x-2}&=\Lim_{x\to 2}\frac{(x+5)(x-2)}{(3x+1)(x-2)}\\
    &=\Lim_{x\to 2}\frac{x+5}{3x+1}\\
    &=\frac{2+5}{3\x 2+1}=1
\end{split}\]

\[\begin{split}
    \Lim_{x\to 0}\frac{\sqrt[3]{1+x^2}-1}{x^2}&=\Lim_{x\to 0}\frac{{(1+x^2)}-1}{x^2\left(\sqrt[3]{(1+x^2)^2}+\sqrt[3]{1+x^2}+1\right)}\\
    &=\Lim_{x\to 0}\frac{1}{\sqrt[3]{(1+x^2)^2}+\sqrt[3]{1+x^2}+1}=\frac{1}{3}
\end{split}\]
\end{solution}

\begin{example}
    求:$\Lim_{x\to 0}\frac{\cos x-1}{x^2},\qquad \Lim_{x\to \pi}\frac{\sin mx}{\sin nx}(m,n\in \mathbb{N})$
\end{example}

\begin{solution}
\[\begin{split}
    \Lim_{x\to 0}\frac{\cos x-1}{x^2}&=\Lim_{x\to 0}\frac{-2\left(\sin\frac{x}{2}\right)^2}{x^2}\\
    &=\frac{-2}{2^2}\Lim_{x\to 0}\frac{\sin\frac{x}{2}}{\frac{x}{2}}\cdot \Lim_{x\to 0}\frac{\sin\frac{x}{2}}{\frac{x}{2}}\\
    &=\left(-\frac{1}{2}\right)\cdot 1\cdot 1=-\frac{1}{2}
\end{split}\]

由三角函数诱导公式易知
\[\sin m(\pi-x)=\sin(m\pi-mx)=(-1)^{m-1}\sin mx\]
同样得到
\[\sin n(\pi-x)=(-1)^{n-1}\sin nx\]
因此:
\[(-1)^{m-n}\frac{\sin mx}{\sin nx}=\frac{\sin m(\pi-x)}{\sin n(\pi-x)}\]
两边乘以$(-1)^{m-n}$,得
\[\frac{\sin mx}{\sin nx}=(-1)^{m-n}\frac{\sin m(\pi-x)}{\sin n(\pi-x)}\]
设$y=\pi-x$, 则当$x\to \pi$, 有$y\to 0$, $my\to 0$, $ny\to 0$. 因此:
\[\begin{split}
    \Lim_{x\to \pi}\frac{\sin mx}{\sin nx}&=\lim_{y\to 0}(-1)^{m-n}\frac{\sin my}{\sin ny}\\
&=(-1)^{m-n}\lim_{y\to 0}\left(\frac{m}{n}\cdot \frac{\sin my}{my}\cdot \frac{ny}{\sin ny}\right)\\
&=(-1)^{m-n}\cdot \frac{m}{n}\cdot 1\cdot 1=(-1)^{m-n}\frac{m}{n}
\end{split}\]
\end{solution}

\begin{example}
由条件$\Lim_{x\to -\infty}\left(\sqrt[]{x^2-x+1}-ax-b\right)=0$,求出$a,b$的值。
\end{example}

\begin{solution}
$\because\quad \Lim_{x\to -\infty}\left(\sqrt{x^2-x+1}-ax-b\right)=0,\quad \Lim_{x\to -\infty}\frac{1}{x}=0$

$\therefore\quad \Lim_{x\to -\infty}\left(\sqrt{x^2-x+1}-ax-b\right)\cdot \frac{1}{x}=0$

即:
\[\Lim_{x\to -\infty}\left(\frac{\sqrt{x^2-x+1}}{x}-a-\frac{b}{x}\right)=0\]
又因为:当$x<0$时,
\[\sqrt{x^2-x+1}=\sqrt{x^2\left(1-\frac{1}{x}+\frac{1}{x^2}\right)}=-x\sqrt{1-\frac{1}{x}+\frac{1}{x^2}}\]
所以上式可写成
\[\lim_{x\to -\infty}\left(-\sqrt{1-\frac{1}{x}+\frac{1}{x^2}}-a-\frac{b}{x}\right)=0\]
即:$-1-a-0=0\quad \to \quad a=-1$

代入原条件,得
\[\lim_{x\to -\infty}\left(\sqrt{x^2-x+1}+x-b\right)=0\]
即:
\[\begin{split}
    b&=\lim_{x\to-\infty}\left(\sqrt{x^2-x+1}+x\right)\\
    &=\lim_{x\to-\infty} \frac{x^2-x+1-x^2}{\sqrt{x^2-x+1}-x}  \\
    &=\lim_{x\to-\infty}\frac{-x\left(1-\frac{1}{x}\right)}{-x\left(\sqrt{1-\frac{1}{x}+\frac{1}{x^2}}+1\right)}    \\
    &=\lim_{x\to-\infty}\frac{1-\frac{1}{x}}{\sqrt{1-\frac{1}{x}+\frac{1}{x^2}}+1} =\frac{1}{2}   \\
\end{split}\]

$\therefore\quad a=-1,\quad b=\frac{1}{2}$为所求。
\end{solution}

\begin{ex}
\begin{enumerate}
    \item 说明当$x\to \infty$时,函数$\frac{x^5-4}{x^3+x}$的变化趋势。
    \item 求极限
    \begin{enumerate}
\begin{multicols}{2}
    \item  $\Lim _{x \to 0} \frac{(x+2)^{2}}{x^{2}+4}$
    \item  $\Lim _{x \to 2} \frac{x^{2}-4}{x-2}$
    \item  $\Lim _{x \to 2} \frac{5 x^{2}+x-8}{x^{2}-4}$
    \item  $\Lim _{x \to 0} \frac{\left(4 x^{3}-3\right)(1-2 x)}{7 x^{3}-6 x+4}$
    \item  $\Lim _{x \to \infty} \frac{3 x^{5}}{x^{5}-x^{2}+1}$
    \item  $\Lim _{x \to \infty} \frac{\left(x^{2}-5\right)\left(x^{2}+7\right)}{x^{4}+35}$
    \item  $\Lim _{x \to 3} \frac{x^{2}-8 x+15}{x^{2}-7 x+12}$
    \item  $\Lim _{x \to -3} \frac{x^{2}-9}{x^{2}+9 x+18}$
    \item $\Lim _{x \to -1} \frac{x\left(x^{2}+4 x+3\right)}{x^{3}+3 x^{2}+5 x+3}$
    \item  $\Lim _{x \to 1} \frac{x^{3}+x^{2}-2}{x^{4}+2 x^{2}-2 x-1}$
    \item $\Lim _{x \to \infty} \frac{x+\sin x}{2 x+5}$
    \item  $\Lim _{x \to 0} \frac{(1+x)^{5}-(1+5 x)}{x^{2}+x^{5}}$
\end{multicols}
    \item  $\Lim _{x \to 0} \frac{(1+m x)^{n}-(1+n x)^{m}}{x^{2}}\qquad (m, n \in \mathbb{N})$
    \item  $\Lim _{x \to 1} \frac{x^{2}+x^{4}+\cdots+x^{2 n}-n}{x-1}$
\end{enumerate}

    \item 求极限  
\begin{enumerate}
\begin{multicols}{2}
\item $\Lim_{x\to 0^+}\sqrt{3x} $
\item $\Lim_{x\to 1^-}\sqrt{1-x} $
\item $\Lim_{x\to 3^-} 3[x]$
\item $\Lim_{x\to 1^+} [x+3]$
\item $\Lim_{x\to -3^+}\left(1+\sqrt{x+3}\right) $
\item $\Lim_{x\to 4^-} \left(\sqrt{4-x}+[x-1]\right)$
    \item $\Lim_{x\to 1^+}\frac{2x|x-1|}{x-1} $
    \item $\Lim_{x\to 4^-} ([x]-x)$
\end{multicols}
    \item $\Lim_{x\to 0^+}\frac{b}{x}\left[\frac{x}{a}\right] \qquad (a>0,\; b>0)$
    \item $\Lim_{x\to 0^+} \frac{x}{a}\left[\frac{b}{x}\right]\qquad (a>0,\; b>0)$
\end{enumerate}    

    \item 求极限  
\begin{enumerate}
\begin{multicols}{2}
\item $\Lim _{x \to  1} \frac{1-x}{\sqrt{1-x^{2}}}$;
\item  $\Lim _{x \to  1} \frac{1-\sqrt{x}}{1-x}$;
\item  $\Lim _{x \to  1} \frac{(2 x-3)(\sqrt{x}-1)}{2 x^{2}+x-3}$
\item  $\Lim _{x \to  2} \frac{x-2}{\sqrt{x^{2}-2}-\sqrt{2}}$
\item  $\Lim _{t \to +\infty}(\sqrt{1+t}-\sqrt{t})$;
\item  $\Lim _{x \to +\infty} \sqrt{x}(\sqrt{x+a}-\sqrt{x})$
\item  $\Lim _{x \to  5} \frac{\sqrt{x-1}-2}{x-5}$;
\item $\Lim _{x \to  0} \frac{\sqrt[3]{1+x}-\sqrt[3]{1-x}}{x}$
\item  $\Lim _{x \to  1} \frac{4 x+\sqrt{x-1}}{2 x-\sqrt{x+1}}$
\item  $\Lim _{x \to  1} \frac{x-1}{\sqrt{x^{2}-1}+\sqrt{x-1}}$
\item $\Lim _{x \to  a} \frac{\sqrt{x-a}+\sqrt{x}-\sqrt{a}}{\sqrt{x^{2}-a^{2}}}$
\item  $\Lim _{x \to -\infty} \frac{x-2}{\sqrt{x^{2}-4 x+3}}$
\item $\Lim _{x \to -\infty}\left[\frac{2(x-1)^{2}}{\sqrt{4 x^{2}+2 x+1}}+x\right]$
\end{multicols}
\item $\Lim _{x \to +\infty}\left(\sqrt{x+\sqrt{x+\sqrt{x}}}-\sqrt{x}\right)$
\item $\Lim _{x \to +\infty} x\left(\sqrt{x^{2}+2 x}-2 \sqrt{x^{2}+x}+x\right)$
\end{enumerate}   

\item \begin{enumerate}
    \item $f(x)=\begin{cases}
        0,& x>1\\
        1,& x=1\\
        x^2+2, &x<1
    \end{cases}$

    求$f(x)$在$x=1$的左右极限。

    \item $f(x)=\begin{cases}
        x\sin\frac{1}{x},& x>0\\
        1+x^2,& x<0
    \end{cases}$

    求$f(x)$在$x=0$的左右极限。

    \item $\Lim_{x\to 3}\frac{[x]^2-9}{x^2-9}$是否存在?
    \item $\Lim_{x\to 0}\frac{x}{|x|}$是否存在?
\end{enumerate}

\item 若$0<x<\frac{\pi}{2}$,求证:
\begin{multicols}{2}
    \begin{enumerate}
        \item $\tan x>x$
        \item $x-\frac{x^2}{4}<\sin x<x$
    \end{enumerate}
\end{multicols}

\item 求极限  
\begin{enumerate}
\begin{multicols}{2}
    \item $\Lim_{x\to 0} \frac{\sin ax}{x}$
    \item $\Lim_{x\to 0} \frac{\sin 7x}{4x}$
    \item $\Lim_{\theta\to 0} \frac{\sec2\theta\sin\theta}{\theta}$
    \item $\Lim_{\theta \to 0} \frac{\tan\theta}{\theta}$
    \item $\Lim_{t\to 0} \frac{15t}{\tan 6t}$
    \item $\Lim_{x\to 0} \frac{\tan 2x}{\sin 7x}$
    \item $\Lim_{\theta \to 0} \frac{\tan\theta-\sin\theta}{\theta^2}$
    \item $\Lim_{x\to 0} \frac{\csc x-\cot x}{x}$
    \item $\Lim_{x\to 0} \frac{\sin^2\frac{x}{2}}{x^2}$
    \item $\Lim_{x\to \pi} \frac{\sin x}{\pi-x}$
    \item $\Lim_{x\to 1} \frac{1+\cos\pi x}{\tan^2 \pi x}$
    \item $\Lim_{x\to \tfrac{\pi}{4}}\tan 2x\tan\left(\frac{\pi}{4}-x\right) $ 
    \item $\Lim_{x\to 1} (1-x)\tan\frac{\pi x}{2}$
    \item $\Lim_{x\to 0} \frac{\sin mx}{\sin nx}$
    \item $\Lim_{x\to 0} \frac{x+\sin x}{x+2\sin x}$
    \item $\Lim_{x\to 0} \frac{\sqrt{\cos x}-1}{x^2}$
\end{multicols}
\end{enumerate}   

\end{enumerate}
\end{ex}

\subsection{数$e$}
我们在这里要利用数列的极限来定义一个新的数,这一个数不论对于分析本身或者对于它的应用来说都是非常重要的。

考虑数列$a_n=\left(1+\frac{1}{n}\right)^n,\; n=1, 2, 3,\ldots$的极限.

首先我们计算一下数列$\{a_N\}$的数值:
\begin{center}
\begin{tabular}{llll}
   $a_1=2.0$ & $a_2=2.250$ & $a_3=2.370$& $a_4=2.441$\\
$a_5=2.488$ &$a_6=2.522$& $a_7=2.546$& $a_8=2.565$\\
$\cdots$ &$a_{256}=2.712$&$\cdots$&$a_{1024}=2.717$\\ 
\end{tabular}
\end{center}

观察这一串数,我们会猜测这个数列是递增的,可能收敛于一个极限值,现在我们来证明这个猜想成立。

依牛顿二项式定理,即有
\[\begin{split}
    \left(1+\frac{1}{n}\right)^n&=1+n\cdot\frac{1}{n}+\frac{n(n-1)}{1\cdot 2}\cdot \frac{1}{n^2}+\frac{n(n-1)(n-2)}{1\cdot 2\cdot 3}\cdot \frac{1}{n^3}\\
    &\quad +\cdots+\frac{n(n-1)(n-2)\cdots[n-(n-1)]}{1\cdot 2\cdot 3\cdots n}\cdot \frac{1}{n^n}
\end{split}\]
这个等式的右边的第二项等于1,其它各项可以变换如下:
\[\begin{split}
\frac{n(n-1)}{1\cdot 2}\cdot \frac{1}{n^2}&=\frac{n(n-1)}{n^2}\cdot\frac{1}{2!}\\&=\left(1-\frac{1}{n}\right) \cdot\frac{1}{2!}\\
\frac{n(n-1)(n-2)}{1\cdot 2\cdot 3}\cdot \frac{1}{n^3}&=\frac{n(n-1)(n-2)}{n^3}\cdot\frac{1}{3!}\\&=\left(1-\frac{1}{n}\right)\cdot \left(1-\frac{2}{n}\right)\cdot\frac{1}{3!}\\
\cdots & \cdots\cdots\\
\frac{n(n-1)(n-2)\cdots[n-(n-1)]}{1\cdot 2\cdot 3\cdots n}\cdot \frac{1}{n^n}&=\frac{n(n-1)(n-2)\cdots[n-(n-1)]}{n^n}\cdot \frac{1}{n!}\\
&=\left(1-\frac{1}{n}\right) \left(1-\frac{2}{n}\right)\cdots\left(1-\frac{n-1}{n}\right)\cdot\frac{1}{n!}
\end{split}\]
其中$n!=1\cdot 2\cdot 3\cdots n$。因此:
\begin{equation}
\begin{split}
\left(1+\frac{1}{n}\right)^n&=1+1+\left(1-\frac{1}{n}\right)\frac{1}{2!}+ \left(1-\frac{1}{n}\right)\left(1-\frac{2}{n}\right)\cdot\frac{1}{3!}\\
&+\cdots+ \left(1-\frac{1}{n}\right) \left(1-\frac{2}{n}\right)\cdots\left(1-\frac{n-1}{n}\right)\cdot\frac{1}{n!}  
\end{split}
\end{equation}

用$n+1$代替$n$代入此公式,求出:
\begin{equation}
\begin{split}
    \left(1+\frac{1}{n+1}\right)^{n+1}&=1+1+\left(1-\frac{1}{n+1}\right)\frac{1}{2!}+ \left(1-\frac{1}{n+1}\right)\left(1-\frac{2}{n+1}\right)\cdot\frac{1}{3!}\\
    &+\cdots+ \left(1-\frac{1}{n+1}\right) \left(1-\frac{2}{n+1}\right)\cdots\left(1-\frac{n}{n+1}\right)\cdot\frac{1}{(n+1)!}  
\end{split}
\end{equation}
(1.9)的展开式项数比(1.8)的展开式的项数多一项,此外(1.9)的展开式各项由第三项起大于(1.8)的展开式对应项,这是因为
\[1-\frac{1}{n+1}>1-\frac{1}{n},1-\frac{2}{n+1}>1-\frac{2}{n},\ldots, 1-\frac{n-1}{n+1}>1-\frac{n-1}{n}\]
所以
\[\left(1+\frac{1}{n+1}\right)^{n+1}>\left(1+\frac{1}{n}\right)^n\]
这就是说数列$a_n=\left(1+\frac{1}{n}\right)^n,\; n=1,2,3,\ldots$是递增的。

但是由等式(1.8)可知
\[\left(1+\frac{1}{n}\right)^n<1+1+\frac{1}{2!}+\frac{1}{3!}+\cdots+\frac{1}{n!}\]
又因为
\[\frac{1}{1\cdot 2\cdot 3}<\frac{1}{2^2},\frac{1}{1\cdot 2\cdot 3\cdot 4}<\frac{1}{2^3},\ldots,\frac{1}{1\cdot 2\cdot 3\cdots n}<\frac{1}{2^{n-1}}\]
所以:
\[\left(1+\frac{1}{n}\right)^n<2+\frac{1}{2}+\frac{1}{2^2}+\cdots+\frac{1}{2^{n-1}}\]
但是由第二项起的级数总和小于$\frac{\tfrac{1}{2}}{1-\tfrac{1}{2}}=1$,所以:
\[\left(1+\frac{1}{n}\right)^n<2+\frac{\frac{1}{2}}{1-\frac{1}{2}}=3\]
这就是说数列$\left\{\left(1+\frac{1}{n}\right)^n\right\}$是递增有上界的,所以它有极限,设它的极限用$e$表示,因而我们有数
\[e=\Lim_{n\to \infty} \left(1+\frac{1}{n}\right)^n\]
这个数$e$是无理数,取它的十进小数到十五位,其值是
\[e=2.718281328459045\cdots\]

现在要把变数$n$推广到实数$x$, 这是一个以后常要用到的重要极限。

\begin{blk}{定理}
\[\Lim_{x\to \infty} \left(1+\frac{1}{x}\right)^x=e\qquad (x\in\mathbb{R})\] 
\end{blk}

\begin{proof}
先证$\Lim_{x\to +\infty} \left(1+\frac{1}{x}\right)^x=e$

令$[x]=n$,于是$n\le x<n+1$,所以,当$x\to +\infty$时,$n\to +\infty$,而且
\[1+\frac{1}{n+1}<1+\frac{1}{x}\le 1+\frac{1}{n}\]
根据指数函数的单调性和幂函数在$x>0$半直线上的单调性,有:
\[\left(1+\frac{1}{n+1}\right)^n\le \left(1+\frac{1}{n+1}\right)^x<\left(1+\frac{1}{x}\right)^x\le \left(1+\frac{1}{n}\right)^x<\left(1+\frac{1}{n+1}\right)^{n+1}\]
因为:
\[\begin{split}
    \Lim_{n\to +\infty} \left(1+\frac{1}{n}\right)^{n+1}&=\Lim_{n\to +\infty} \left(1+\frac{1}{n}\right)^{n}\left(1+\frac{1}{n}\right)\\
&=\Lim_{n\to +\infty} \left(1+\frac{1}{n}\right)^{n}\cdot \Lim_{n\to +\infty} \left(1+\frac{1}{n}\right)=e
\end{split}\]
\[\begin{split}
    \Lim_{n\to +\infty} \left(1+\frac{1}{n+1}\right)^{n}&=\Lim_{n\to +\infty} \frac{\left(1+\frac{1}{n+1}\right)^{n+1}}{1+\frac{1}{n+1}}\\
    &=\frac{\Lim_{n\to +\infty}\left(1+\frac{1}{n+1}\right)^{n+1} }{\Lim_{n\to +\infty}\left(1+\frac{1}{n+1}\right) }=e
\end{split}\]
所以
\[e\le\Lim_{x\to +\infty}\left(1+\frac{1}{x}\right)^{x} \le e\]
即:$\Lim_{x\to +\infty}\left(1+\frac{1}{x}\right)^{x}=e$

当$x\to -\infty$时,令$x=-y$,那么$y\to +\infty$,这时
\[\begin{split}
    \left(1+\frac{1}{x}\right)^x&=\left(1-\frac{1}{y}\right)^{-y}=\left(\frac{y-1}{y}\right)^{-y}=\left(\frac{y}{y-1}\right)^y\\
    &=\left(1+\frac{1}{y-1}\right)^y=\left(1+\frac{1}{y-1}\right)^{y-1}\left(1+\frac{1}{y-1}\right)
\end{split}\]
所以
\[\begin{split}
    \Lim_{x\to -\infty}\left(1+\frac{1}{x}\right)^{x}&=\Lim_{y\to +\infty}\left(1+\frac{1}{y-1}\right)^{y-1}\left(1+\frac{1}{y-1}\right)\\
    &=\Lim_{y\to +\infty}\left(1+\frac{1}{y-1}\right)^{y-1}\cdot \Lim_{y\to +\infty}\left(1+\frac{1}{y-1}\right)=e
\end{split}\]
综合上面两个结果,即得
$$\Lim_{x\to \infty}\left(1+\frac{1}{x}\right)^{x}=e$$
\end{proof}

\begin{blk}{推论}
\[\Lim_{y\to 0}\left(1+y\right)^{\tfrac{1}{y}}=e\]
\end{blk}

事实上,令$x=\frac{1}{y}$,当$y\to 0$时,$x\to\infty$,于是:
\[\Lim_{y\to 0}\left(1+y\right)^{\tfrac{1}{y}}=\Lim_{x\to \infty}\left(1+\frac{1}{x}\right)^{x}=e\]

\begin{ex}
\begin{enumerate}
    \item 求下面变量的极限:
\begin{multicols}{2}
\begin{enumerate}
    \item $\Lim_{x\to\infty}\left(1+\frac{8}{x}\right)^x$
    \item $\Lim_{y\to 0}(1+y)^{\tfrac{1}{3}y}$
    \item $\Lim_{t\to\infty}\left(1-\frac{1}{t}\right)^t$
    \item $\Lim_{t\to\infty}\left(\frac{t}{1+t}\right)^t$
    \item $\Lim_{n\to+\infty}\left(1+\frac{x}{n}\right)^n$
    \item $\Lim_{n\to +\infty}\left(1+\frac{1}{n^2}\right)^n$
    \item $\Lim_{x\to+\infty}\left(\frac{2x+3}{2x+1}\right)^{x+1}$
\end{enumerate}
\end{multicols}

\item 证明:
\begin{multicols}{2}
\begin{enumerate}
    \item $\Lim_{x\to\infty}\frac{a^x}{x}=\infty$
    \item $\Lim_{x\to\infty}\frac{a^x}{x^a}=\infty$
\end{enumerate}
\end{multicols}
其中$a$为任意正常数。
\end{enumerate}
\end{ex}


\begin{example}
    
\end{example}


\begin{solution}
    
\end{solution}


\begin{example}
    
\end{example}

\begin{solution}
    
\end{solution}

\begin{example}
    
\end{example}

\begin{solution}
    
\end{solution}

\begin{solution}
    
\end{solution}






\begin{solution}
    
\end{solution}





























































































\chapter{变率和微商}
从本章起,我们开始学习单变量的微积分学的基本概念和基础理论。

微积分学是研究变量的数学,变量之间的关系就是函数,因此,函数是微积分学研究的主要对象。

在函数的基本性质中,有两个最基本、最重要的概念-变率与求和,为了解决求函数的变率与求函数$f$在$[a,b]$上的和的问题就相应地产生微分与积分运算,而这两种运算之间,也存在着一种自然的互逆关系,在本章中,我们由函数的变率问题引出函数的微商(导数)概念,并给出初等函数的一套求导法则,在下一章中,我们揭示微分运算与积分运算的互逆关系,这就是微积分学的基本定理。

总起来说,微分反映了函数的局部性质,或在某个点附近的性质;积分则反映了函数的整体性质,或某个区间的性质;函数的局部性质与整体性质之间的有机联系,恰恰反映了微分运算与积分运算之间的互逆关系。

\section{微商(导数)的定义}
函数关系$y=f(x)$就是确定变量$y$如何随着变量$x$的变动而变动的关系,对于给定的函数$y=f(x)$, 变量$y$在变量$x$的不同点附近的变动情况是不尽相同的,这就是说,在变量$x$的
某个值$x_1$外,当$x_1$略加变动时,相应的$y$的变动可能相当剧烈(急增,或急减);而在变量$x$的另一个值$x_2$处,$y$的变动就可能较为迟缓,但是,用这样的语言来表达函数在某一点处的变率是不精确的,我们需要用“数量”来确切地表达这个意思,这就是函数在某点(或菜时刻)的变率的问题,简称变率。

\subsection{直线函数的变率}
一次函数$f (x) =kx+b$
是一种最简单的函数,它的函数图象是一条斜率等于$k$的直线,如图2.1所示。
\begin{figure}[htp]
    \centering
\begin{tikzpicture}[>=latex]
    \draw[->](-2,0)--(4,0)node[right]{$x$};
    \draw[->] (0,-1)--(0,5)node[right]{$y$};
    \draw[very thick] (-1.5,-.5)--(4,5);
\draw[->](-.2,0) arc (0:45:.8)node[below]{$\theta$};
\draw(2,3)node[left]{$P_1$}--(2,0)node[below]{$x_1$};
\draw(3,4)node[left]{$P$}--(3,0)node[below]{$x$};
\draw[|<->|](3.2,4)--node[right]{$f(x)-f(x_1)$}(3.2,3);
\draw[|<->|](2,2.8)--node[below]{$x-x_1$}(3,2.8);
\draw(2,3)--(3,3);\node at (.25,-.25){$O$};
\end{tikzpicture}
    \caption{}
\end{figure}

设$x_1$, $f(x_1)$和$x$, $f(x)$分别是直线上点$P_1$和$P$的坐标,为了反映函数变化快慢的问题,无论$x<x_1$还是$x>x_1$,自然地考虑在点$x_1$邻近,函数与自变量的相应的改变量的比:
\[\frac{f(x)-f(x_1)}{x-x_1}=\frac{(kx+b)-(kx_1+b)}{x-x_1}=k=\tan\theta\]
上面的表达式称为函数的\textbf{差商},它表示函数在区间$[x_1,x]$或$[x,x_1]$上对于自变量$x$的\textbf{平均变化率}。由于$k$是不随变量$x$变动而变动的常数,因此,一次函数在自变量的任何一个区间内的平均变化率都是常数。

如果让自变量的变化区间的长度无限地缩短,也就是让$x$无限地接近于$x_1$时,平均变化率所趋向的极限
\[\lim_{x\to x_1} \frac{f (x) -f (x_1)}{x-x_1} =k\]

\subsection{平滑曲线的切线与变率}
一般的函数$y=f(x)$的图象通常不是直线,由于函数
和它的图象的多样性,为了讨论的方便起见,我们先把讨论的范围限制在“平滑”的曲线上,常用的函数$y=f(x)$的图象往往是“平滑的”,平滑性的直观内涵是:用愈高倍的显微镜去观察曲线的微段,就愈象直线段,比较明确的几何说法是:一条曲线在$P$点的平滑性就是存在唯一的一条过$P$点的切线,它无限地逼近曲线在$P$点邻近的微段,于是,当$y=f(x)$的图象$C$在$P$点存在唯一的一条切线时,我们就说曲线$C$在$P$点平滑,而$P$点叫做曲线的平滑点,一条在每点都平滑的曲线叫做平滑的曲线,一个函数的图象平滑曲线时,我们就称这种函数为平滑函数。

同学可能会问这样一个问题:在曲线的点$P$存在唯一的一条切线的含义是什么?因为迄今我们对于一般的曲线的切
线还未下过定义呢!

我们从图2.2和2.3注意到不能把切线定义为与曲线只有一个交点的直线。这样的定义限制得既太紧同时又太松。因
为,照此定义,图2.2所示的直线就不是过曲线上$P$点的切线了,实际上,尽管图2.2的直线与曲线还有一个交点$Q$, 但它在曲
线$P$点邻近却与曲线密合,故仍应该是过曲线$P$点的切线;又图2.3表明过抛物线上任何一点$P$与$y$轴平行的直线虽然与抛物线只有一个交点,但它的其余部分却远离$P$点邻近的弧,故它不应该是抛物线的切线,定义切线的可行途径是从割线开始,并应用极限的概念。


如图2.4所示,取曲线$y=f(x)$上点$P$附近的另一点$Q$, 通过这两点画一条直线,这直线叫做过曲线上$P$点的割线,让$Q$点沿曲线向点$P$移动,这条割线将达到极限位置,此极限位置与$Q$点从哪一侧趋向于$P$是无关的,我们称这个割线的极限位置为过曲线上$P$点的切线。

割线的这种极限位置的存在性这一假设,与曲线在点$P$具有唯一的一条切线或确定的方向的假设是等价的。

现在我们要对曲线$y=f(x)$用解析式子把割线的这种极限位置存在的过程表示出来。
设$\alpha$是割线$PQ$同正$x$轴构成的夹角,$\alpha_1$是过点$P$点的切线同正$x$轴构成的夹角,于是
\[\lim_{Q\to P}\alpha=\alpha_1\]
设$x_1,y_1$和$x,y$分别是点$P$和$Q$的坐标,这时,我们立即得到
\[\tan\alpha=\frac{y-y_1}{x-x_1}=\frac{f(x)-f(x_1)}{x-x_1}\]
因此,上述求极限的过程(不考虑垂直切线$\alpha_1=\frac{\pi}{2}$的情况)可由下式来表示:
\[\lim_{x\to x_1}\frac{f(x)-f(x_1)}{x-x_1}=\lim_{\alpha\to \alpha_1}\tan\alpha=\tan\alpha_1\]
这就是说过曲线$y=f(x)$上$P(x_1,y_1)$点的切线的斜率等于$y=f(x)$的差商当$x\to x_1$时的极限.

\begin{example}
    求抛物线$y=ax^2+bx+c$在$x_0$处的切线的斜率。
\end{example}

\begin{solution}
    解依题意$(x_0,f(x_0)=ax_0^2+bx_0+c)$在抛物线上,并设$(x_0+h,f(x_0+h))$是抛物线上点$(x_0,f(x_0))$的附近的一点,我们有
\[\begin{split}
&\qquad \lim_{h\to 0}\frac{f(x_0+h)-f(x_0)}{(x_0+h)-x_0}\\
&=\lim_{h\to 0}\frac{[a(x_0+h)^2+b(x_0+h)+c]-[ax^2_0+bx_0+c]}{h}    \\
&=\lim_{h\to 0}\frac{(2ax_0+b)h+h^2}{h}\\
&=\lim_{h\to 0}[(2ax_0+b)+h]=2ax_0+b
\end{split}\]
所以抛物线$y=ax^2+bx+c$在$x_0$处的切线的斜率是$2ax_0+b$.
\end{solution}

\begin{example}
    一质点沿一直线在$t$秒内移动的距离是$s=s(t)=t^2+4t$。
    
    求:质点的初速度;在两秒末的速度;前两秒内的平均速度。
\end{example}

\begin{analyze}
    如果质点从起点开始所走的距离$s$是时间$t$的线性函数,则由2.1知道该质点在每一时刻的速度都是常数,它的大小由平均速度来确定,即等于一次函数的斜率,此时我们说该质点作匀速运动,但是,如果运动不再是匀速的,即质点的速度每时每刻都是变的,那么我们将时刻$t$的速度(也叫做瞬时速度)理解成什么呢?为了回答这个问题,我们考察差商
\[\frac{\Delta s}{\Delta t}=\frac{s(t)-s(t_0)}{t-t_0}\]
或者写成\[\frac{s(t_0+\Delta t)-s(t_0)}{\Delta t}\]
这个差商称为在$t_0$和$t_0+\Delta t$之间的这段时间间隔上的质点的平均速度,对照着$s=s(t)=t^2+4t$的图象来看,这个平均速度也就是过曲线上的$P(t_0,s(t_0))$
点及它邻近一点$Q(t_0+\Delta t,s(t_0+\Delta t))$的割线的斜率(图2.5)。当$\Delta t$ 很小时,可以认为,从时刻$t_0$到$t_0+\Delta t$这段时间内,速度来不及有很大变化,可以近似地看成匀速运动,因而这段时间内的平均速度就可以看成时刻$t_0$的瞬时速度的近似值。

\begin{figure}[htp]
    \centering
    \begin{tikzpicture}[>=latex, scale=.5]
\draw[->](-5.5,0)--(3,0)node[right]{$t$};
\draw[->](0,-5)--(0,7)node[right]{$s$};

\draw[domain=-5.25:0, samples=300, dashed]plot(\x, {\x*\x+4*\x});
\draw[domain=0:1.25, samples=100, thick]plot(\x, {\x*\x+4*\x});
\foreach \x in {-4,-3,-2,-1,1,2}
{
    \draw(\x,0)--(\x,.2);
}
\node at (-2,0)[below]{$-2$};
\node at (1,0)[below]{$1$};
\node at (0,0)[below left]{$O$};
    \end{tikzpicture}

    \caption{}
\end{figure}



显然,从时刻$t_0$到时刻$t_0+\Delta t$, 质点走过的路程为
\[\begin{split}
    \Delta s&=s(t_0+\Delta t)-s(t_0)\\
    &=(t_0+\Delta t)^2+4(t_0+\Delta t)+(t^2_0+4t_0)\\
    &=(2t_0+4)\Delta t+(\Delta t)^2
\end{split}\]
所以这段时间内的平均速度为
\[\begin{split}
    \frac{\Delta s}{\Delta t}&=\frac{s(t_0+\Delta t)-s(t_0)}{\Delta t}\\
    &=\frac{(2t_0+4)\Delta t+(\Delta t)^2}{\Delta t}\\
    &=(2t_0+4)+\Delta t
\end{split}\]
$\Delta t$越小,这个平均速度就越接近时刻$t_0$的瞬时速度$v_0$, 我们自然令$\Delta t\to 0$, 求差商的极限值,得到
\[\begin{split}
    \lim_{\Delta t\to 0}\frac{s(t_0+\Delta t)-s(t_0)}{\Delta t}&= \lim_{\Delta t\to 0}[(2t_0+4)+\Delta t]\\
    &=2t_0+4
\end{split}\]
这样平均速度$\frac{s(t_0+\Delta t)-s(t_0)}{\Delta t}$,当$\Delta t\to 0$时的极限值
就表达了质点在时刻$t_0$的瞬时速度,把它记作
\[s'(t_0)=   \lim_{\Delta t\to 0}\frac{s(t_0+\Delta t)-s(t_0)}{\Delta t}\]
\end{analyze}

\begin{solution}
\begin{enumerate}
\item 质点的初速度
\[s'(0)=2\x0+4=4(\ms)\]
\item 质点在两秒末的速度
\[s'(2)=2\x2+4=8(\ms)\]
\item \[\text{质点在前两秒内的平均速度}=\frac{\text{在前两秒内所走距离}}{\text{时间}}=\frac{2^2+4\x2}{2}
=6(\ms)\]
\end{enumerate}
\end{solution}

从这个问题可以看出质点在$t_0$时刻的瞬时速度$s'(t_0)$的几何意义就是曲线$s=s(t)$在$P(t_0,s(t_0))$点的切线的斜率,所以函数在某点的变率有确定值与函数的图象在该点有唯一的一条切线是两个密切相关的概念。

\subsection{微商(导数)的定义}
从上面所举的两个例子来看,问题来自不同的领域:
\begin{enumerate}
    \item 求过曲线上一点的切线,
    \item 求函数在某点的变率.
\end{enumerate}
但解决的方法却完全一样,就是计算函数的差商的极限,这种极限反映了自然界中很多不同现象在量方面的共性,因此有必要从这些具体问题中把它抽象出来加以研究,再反过来去解决这类具体问题。

\begin{blk}{定义}
    设$y=f(x)$是定义在闭区间$[a,b]$上的一个函数,$x_0\in (a,b)$, 如果极限
\[\lim_{\Delta x\to 0}\frac{f (x_0+\Delta x) -f (x_0)}{\Delta x}\]
存在,我们就说$f(x)$在点$x_0$处\textbf{可微},并称这极限为函数$f(x)$在$x_0$点的\textbf{微商}(或导数),记为$f'(x_0)$或
$\frac{\dd y}{\dd x}\Big|_{x=x_0}$.
\end{blk}

显然,$f'(x_0)$的值与点$x_0$有关,当点$x_0$在开区间$(a,b)$内变化时,$f'(x_0)$也将跟着变化,因此,如果函数$f(x)$在开区间$(a,b)$内每点都可微(即存在有限导数),那么$f'(x)$便是一个新的函数,称为$f(x)$的\textbf{导函数}。

求已知函数的导函数$f'(x)$的运算,称为微商 运算,计算过程如下:
\begin{enumerate}
\item 设$\Delta x$为自变量某个值$x$的改变量:
\item 计算$f(x)$在点$x$的相应改变量
\[\Delta y=f (x+\Delta x) -f (x) \]
\item 计算$f(x)$在点$x$的差商
\[\frac{\Delta y}{\Delta x}=\frac{f (x+\Delta x) -f (x)}{\Delta x}\]
\item 计算
\[\lim_{\Delta x\to 0}\frac{f (x+\Delta x) -f (x)}{\Delta x}=f'(x)=\frac{\dd y}{\dd x}\]
\end{enumerate}

应当注意,这里$\frac{\dd y}{\dd x}$是一个独立的记号,它表示函数$f(x)$
在点$x$的导数,不能把它当成一个分数来看待,必须把它看成一个整体。

从微商的定义可以看出:
\begin{enumerate}
\item 曲线在一点的切线的斜率,就是函数在这一点的变率(微商或导数)。    
\item 微商所涉及的是函数的“局部”性质,也就是说,函数$y=f(x)$在一点$x_0$处是否可微只与函数$y=f(x)$在$x=x_0$处及其近旁的性质有关,而与其它地方无关。    
\item 如果$f(x)$在点$x$可微,按照极限存在的条件,必须且只须
\[\lim_{\Delta x\to 0^+}\frac{f (x+\Delta x) -f (x)}{\Delta x},\qquad \lim_{\Delta x\to 0^-}\frac{f (x+\Delta x) -f (x)}{\Delta x}\]
同时存在而且相等。

上面两式分别称为$f(x)$在点$x$的\textbf{右导数}和\textbf{左导数},记为$f'_+(x)$和$f'_-(x)$.
\end{enumerate}

\begin{example}
今有一个正在膨胀的肥皂泡,
\begin{enumerate}
\item 求肥皂泡的体积对于半径的增大率,
\item 如果肥皂泡的半径每秒增大0.1cm,问当半径为2cm时,体积的增大率是多少?
\end{enumerate}
\end{example}

\begin{solution}
    肥皂泡的体积$V$与半径$r$的函数关系是。
\[ V(r) =\frac{4}{3}\pi r^3\]
体积$V(r)$对于半径$r$的增大率,依导函数定义,就是$V(r)$对于$r$的导函数,故
\[\begin{split}
    V'(r)&=\lim_{\Delta r\to 0}\frac{V(r+\Delta r)-V(r)}{\Delta r}\\
    &=\lim_{\Delta r\to 0}\frac{\frac{4\pi }{3}[(r+\Delta r)^3-r^3]}{\Delta r}\\
    &=\frac{4\pi }{3}\lim_{\Delta r\to 0}[3r^2+3r(\Delta r)+(\Delta r)^2]\\
    &=4\pi r^2
\end{split}\]

因此,肥皂泡的体积$V$对于半径$r$的增大率是$4\pi r^2$. 当$r=2$时,\[V'(2)=4\pi \cdot 2^2=16\pi \]
上式表示在$r=2$时,肥皂泡体积对于半径的增大率是$16\pi$, 它
的含义也可以这样理解,体积增大的快慢在$r=2$时,是$16\pi$倍于半径增大的快慢。

由于半径的增大率是每秒0.1cm,故体积在半径等于2时的增大率是
\[16\pi \x0.1=1.6\pi ({\rm cm^3/s})\]
\end{solution}

\begin{ex}
\begin{enumerate}
    \item 设$y=\frac{1}{x},\quad (x\ne 0)$,求
\begin{enumerate}
    \item 当$x$取改变量$\Delta x$后,函数改变量$\Delta y$的表达式;
    \item 当$x=3$, $\Delta x=-1$时,$\Delta y$的值;
    \item 当$x=3$时,$\frac{\Delta y}{\Delta x}$的表达式和$\frac{\dd y}{\dd x}\Big|_{x=3}$的值.
\end{enumerate}

    \item 设$\phi(x)=\frac{2}{x^2},\quad (x\ne0)$, 求
\begin{enumerate}
\item 当$x=1$, $\Delta x=-0.5$时,$\Delta \phi(x)$的值;
\item 对于任何$x\; (x\ne 0)$, 在$\Delta x$的间隔内,$\phi(x)$的平均变率$\frac{\Delta \phi}{\Delta x}$的表达式;
\item $\phi(x)$在任何点$x\; (x\ne 0)$处的瞬时变率$\phi'(x)$.
\end{enumerate}

    \item 平均变化率$\frac{\Delta y}{\Delta x}=\frac{f (x+\Delta x) -f (x)}{\Delta x}$依赖于哪两个变量?在平均变化率取极限求瞬时变化率的过程中,$x$是变量还是常量?$\Delta x$是变量还是常量?
    \item 点$P(2, 8)$在抛物线$y=x^2+2x$上,点$Q$为抛物线上任何一点。
\begin{enumerate}
    \item 求割线$PQ$的斜率的表达式;
  \item 当$Q$点横坐标为$2.1, 1.9, 2.002, 1.998, 2+h$时,求割线斜率的值;
\item 求在$P(2, 8)$点处,抛物线$y=x^2+2x$的切线方程和法线方程。
\end{enumerate}    

\item 求圆面积对于它的半径的变率,对于它的直径的变率.
\item 圆的半径的变率为2cm/s, 求圆面积在半径等于4cm
时的变率。
\end{enumerate}
\end{ex}

\subsection{函数的可微性与连续性的关系}

\begin{blk}
    {定理} 如果函数$f(x)$在点$x_0$可微,那么$f(x)$在点$x_0$处连续。
\end{blk}

\begin{proof}
设$f(x)$在点$x_0$处是可微的,也就是$f(x)$在$x_0$处有导数,则
\[f'(x_0)=\lim_{x\to x_0}\frac{f(x)-f(x_0)}{x-x_0}\]
所以
\[\begin{split}
    \lim_{x\to x_0}[f(x)-f(x_0)]&=\lim_{x\to x_0}\frac{f(x)-f(x_0)}{x-x_0}(x-x_0)\\
    &=\lim_{x\to x_0}\frac{f(x)-f(x_0)}{x-x_0}\cdot \lim_{x\to x_0}(x-x_0)\\
&=f'(x_0)\cdot 0=0
\end{split}\]
\end{proof}

上面的定理说明函数在其有导数的点一定连续,在其不
连续的点一定没有导数。但是它的逆命题不一定成立,即连续的函数不一定有导数。前面我们曾指出一个函数可微的充分必要条件是它的左导数和右导数都存在而且相等,下面给出在某些点不可微的函数的例子。

\begin{example}
    求函数$f(x)=|x^2-1|$的导函数的定义域。
\end{example}

    
\begin{solution}
函数$f(x)=|x^2-1|$是一个到处连续的函数,它的图象如图2.6所示.

去掉函数式中的绝对值的符号,$f(x)$便可以写成分段函数式:
\[f(x)=\begin{cases}
    x^2-1,& x\le -1\quad \text{或}\quad x\ge 1\\
    -(x^2-1), & -1\le x\le 1
\end{cases}\]
所以
\[f'(x)=\begin{cases}
    2x ,&  x< -1\quad \text{或}\quad x> 1\\
    -2x,& -1<x<1
\end{cases}\]
但是当$x=-1$和1时,函数$f(x)$的左导数和右导数不相等,即此时导数不存在。因为当$x=-1$时,函数$f(x)$在$-1$点左邻域的表达式是
\[f (x) =x^2-1\]
而在$-1$点右邻域的表达式是
\[f (x) =- (x^2-1)\]
所以$f(x)$在$x=-1$点的左导数是
\[\begin{split}
    f'_-(-1)&=\lim_{\Delta x\to 0^-}\frac{f(-1+\Delta x)-f(-1)}{\Delta x}\\
    &=\lim_{\Delta x\to 0^-}\frac{[(-1+\Delta x)^2-1]-0}{\Delta x}\\
    &=\lim_{\Delta x\to 0^-}(-2+\Delta x)=-2
\end{split}\]
$f(x)$在$x=-1$点的右导数是
\[\begin{split}
    f'_+(-1)&=\lim_{\Delta x\to 0^+}\frac{f(-1+\Delta x)-f(-1)}{\Delta x}\\
    &=\lim_{\Delta x\to 0^+}\frac{-[(-1+\Delta x)^2-1]-0}{\Delta x}\\
    &=\lim_{\Delta x\to 0^+}(2-\Delta x)=2 
\end{split}\]
由于$f'_-(-1)\ne f'_+(-1)$, 所以我们说$f(x)$在$x=-1$处的导数不存在,同样可得
\[-2=f'_- (1)\ne f'_+(1)=2\]
故$f(x)$在$x=1$处也不可微.

因此,$f(x)=|x^2-1|$的导函数的定义域是$x\ne\pm 1$的点的集合,它的函数值如下:
\[f'(x)=\begin{cases}
    2x,& |x|>1\\
    -2x,& |x|<1
\end{cases}\]
\end{solution}

对照图2.6来看,尽管函数$f(x)=|x^2-1|$处处连续,但是在$x=-1$和1这两点,曲线的特征是当动点由左侧趋于定
点1(或$-1$)时,割线的极限位置存在;当动点由右侧趋于定点1(或$-1$)时,割线的极限位置也存在,这两个极限位置分别叫1(或$-1$)点的\textbf{左、右切线},并且它们之间的夹角不为\textbf{平角},点$-1$和1分别是曲线的一种\textbf{角点}.
    
图2.7的情形是曲线$y=f(x)$在$P(x_0,y_0)$点连续并且在此点有平行于$y$轴的切线.现在我们来讨论图2.7中所示各函数在点$x_0$处的导数:

图(a)表示的函数在点$x_0$的邻近递增,因此过曲线上$P(x_0,y_0)$和$Q(x_0+\Delta x,y_0+\Delta y)$两点的割线,无论$Q$点在$P$点的哪一侧,$\Delta x$与$\Delta y$都有相同正、负号,于是
\[\frac{\Delta y}{\Delta x}=\frac{f(x_0+\Delta x)-f(x_0)}{\Delta x}>0\]
而且当$\Delta x\to 0$时,割线的倾斜角$\varphi$($0^{\circ}\le \varphi\le 180^{\circ}$)便以$\pi/2$为极限,从而
\[\lim_{\Delta x\to 0}\frac{f(x_0+\Delta x)-f(x_0)}{\Delta x}=\lim_{\varphi\to \tfrac{\pi^-}{2}}\tan\varphi=+\infty\]
可见$f(x)$在点$x_0$不可微(不存在有限导数),但是我们常常把这种情况:当$\Delta x\to 0$时,$\frac{\Delta y}{\Delta x}\to\infty$简略地叙述为函
数$f(x)$在$x_0$的导数为正无穷大,记作$f'(x_0)=+\infty$。

图(b)表示的函数$f$在点$x_0$的邻近递减,因此过曲线上$P(x_0,y_0)$和$Q(x_0+\Delta x,y_0+\Delta y)$两点的割线,无论$Q$点在$P$点的哪一侧,$\Delta x$与$\Delta y$都有相反的正、负号,于是
\[\frac{\Delta y}{\Delta x}=\frac{f(x_0+\Delta x)-f(x_0)}{\Delta x}<0\]
而且当$\Delta x\to 0$时,$\varphi\to \frac{\pi^+}{2}$,从而
\[\lim_{\Delta x\to 0}\frac{f(x_0+\Delta x)-f(x_0)}{\Delta x}=\lim_{\varphi\to \tfrac{\pi^+}{2}}\tan\varphi=-\infty\]
因此$f(x)$在点$x_0$处不可微,但是我们常常把这种情况叙述为函数$f(x)$在$x_0$处的导数为负无穷大,记作$f'(x_0)=-\infty$。

图2.7(c)表示的函数在点$x_0$处连续,让$x$由$x_0$变动到
$x_0+\Delta x$,于是由$\Delta x$所引起的相应的函数的改变量$\Delta y$的情形是:
\begin{itemize}
    \item 当$\Delta x<0$时,有$\Delta y<0$,从而$\frac{\Delta y}{\Delta x}>0$;
    \item 当$\Delta x>0$时,有$\Delta y<  0$,从而$\frac{\Delta y}{\Delta x}<0$。
\end{itemize}
于是:
\begin{itemize}
    \item 当$\Delta x\to 0^-$时,$\varphi\to \frac{\pi^-}{2}$,从而$\frac{\Delta y}{\Delta x}=\tan\varphi\to +\infty$;
    \item 当$\Delta x\to 0^+$时,$\varphi\to \frac{\pi^+}{2}$,从而$\frac{\Delta y}{\Delta x}=\tan\varphi\to -\infty$。
\end{itemize}
因此$f(x)$在点$x_0$不可微,但是我们常把这种情况叙述为$f(x)$在点$x_0$的左导数$f'_-(x_0)=+\infty$, 而它的右导数为$f'_+(x_0)=-\infty$, 曲线在$x_0$处由上升转为下降有一个尖点,它的切线垂直于$x$轴。

图(d)表示的函数在点$x_0$处连续,但是
\begin{itemize}
    \item 当$\Delta x\to 0^-$时,$\frac{\Delta y}{\Delta x}=\tan\varphi\to -\infty$;
    \item 当$\Delta x\to 0^+$时,$\frac{\Delta y}{\Delta x}=\tan\varphi\to +\infty$。
\end{itemize}
因此,$f(x)$在点$x_0$不可微,我们常常把这种情况叙述为$f'_-(x_0)=-\infty$和$f'_+(x_0)=+\infty$, 点$x_0$是曲线的一个尖点,它的切线垂直于$x$轴。

\begin{example}
设$f(x)=\begin{cases}
    0,& x=0\\
    x\sin\frac{1}{x},& x\ne 0
\end{cases}$

讨论它在0点的导数。
\end{example}

\begin{solution}
在第一章,我们已经说明了
\[\lim_{x\to 0}x\sin\frac{1}{x}=0=f(0)\]
因此,$f(x)$在点$x=0$连续,很明显$f(x)$在其它各点处也都连续,所以说$f(x)$是一个到处连续的函数,我们将说明它在原点不存在导数。

因为
\[\frac{f(0+\Delta x)-f(0)}{\Delta x}=\frac{\Delta x\sin\frac{1}{\Delta x}}{\Delta x}=\sin\frac{1}{\Delta x}\]
而当$\Delta x\to 0$时,$\sin\frac{1}{\Delta x}$
没有极限,故$f'(0)$不存在,就
其几何意义来说,当动点沿着曲线$y=x\sin\frac{1}{x}$趋于原点$O$
时,割线$OQ$不断地在$-\frac{\pi}{4}\le \theta\le \frac{\pi}{4}$这个幅度之内摆动,而不趋于任何极限位置,即切线不存在(图2.8)。
\end{solution}

通过以上的例子,我们知道函数的连续点未必是它的可以微分点。几种常见的不可微分点如例2.4的角点,平行于$y$轴的切线的切点,特别是曲线上的尖点和在例2.5的曲线上的不存在割线的极限位置的点。

假如区间$[a,b]$上的函数$f(x)$, 在开区间$(a,b)$中存在着导数$f'(x)$, 并且$f'_+(a)$和$f'_-(b)$都存在,则称$f(x)$在闭区间上可以微分(可导)。

\begin{ex}
\begin{enumerate}
    \item 说明下面函数在点$x=0$不可微:
\begin{multicols}{2}
\begin{enumerate}
    \item $f(x)=|x|$
    \item $f(x)=\begin{cases}
        x^2,& x\le 0\\
        x,& x\ge 0
    \end{cases}$
    \item $f(x)=\sqrt[3]{x}$
    \item $f(x)=\sqrt{|x|}$
\end{enumerate}
\end{multicols}
\item 说明函数$f(x)=\begin{cases}
    x^2,& x\ge 0\\
    -x^2,& x<0
\end{cases}$
    处处可微。
\item \begin{enumerate}
\item 假设$g(x)=f(x+c)$. 从定义出发证明$g'(x)=f'(x+c)$, 并绘图说明。    
\item 设$f$可微并有周期$\varphi$, 证明$f'$也是有周期的.
\end{enumerate}

\end{enumerate}
\end{ex}

\section{微商运算的基本法则}
\subsection{几个基本函数的微商(导数)}
\subsubsection{常数函数的导数恒等于0}

\begin{proof}
设$f(x)=c$ ($c$是常数),则根据常数函数的性质,由自变数$x$的改变量$\Delta x$引起的相应的函数的改变量为:
\[\Delta y=f (x+\Delta x) -f (x)=c-c=0\]
显然差商$\frac{\Delta y}{\Delta x}=\frac{0}{\Delta x}=0$, 故
\[f' (x) =\lim_{\Delta x\to 0}\frac{\Delta y}{\Delta x}=
\lim_{\Delta x\to 0} 0=0\]
这个事实是明显的,因为当自变数$x$有增、减时,函数值并无增、减,因此,它的变化率为零。
\end{proof}

\subsubsection{一次函数(直线函数)的导数是常数}

若$f(x)=kx+b$, 则$f'(x)=k$. 

\subsubsection{若$f(x)=x^n$ ($n$是正整数),则$f'(x)=nx^{n-1}$}

\begin{proof}
  由自变量的改变量$\Delta x$引起的函数改变量为
\[\Delta y=f (x+\Delta x) -f (x) = (x+\Delta x)^n-x^n\] 
按牛顿二项式定理展开,得  
\[\begin{split}
\Delta y&= x^n+nx^{n-1}\Delta x+\frac{n(n-1)}{1\cdot 2}x^{n-2}(\Delta x)^2+\cdots+nx(\Delta x)^{n-1}+(\Delta x)^n-x^n\\
&=nx^{n-1}\Delta x+\frac{n(n-1)}{1\cdot 2}x^{n-2}(\Delta x)^2+\cdots+nx(\Delta x)^{n-1}+(\Delta x)^n
\end{split}\]
其中差商
\[\frac{\Delta y}{\Delta x}=nx^{n-1}+\frac{n(n-1)}{1\cdot 2}x^{n-2}\Delta x+\cdots+nx(\Delta x)^{n-2}+(\Delta x)^{n-1}\]
式中除第一项外,其余各项随$\Delta x\to 0$而趋于0,因此:
\[f'(x)=\lim_{\Delta x\to 0}\frac{\Delta y}{\Delta x}=nx^{n-1}\]
\end{proof}

\subsubsection{若$f(x)=x^{\mu}$ (其中$\mu$是任意实数,函数的定义域依赖于$\mu$), 则$f'(x)=\mu x^{\mu-1}$}

\begin{proof}
    \[\frac{\Delta y}{\Delta x}=\frac{(x+\Delta x)^{\mu}-x^{\mu}}{\Delta x}=x^{\mu-1}\cdot \frac{\left(1+\frac{\Delta x}{x}\right)^{\mu-1}}{\frac{\Delta x}{x}}\]
利用第一章中已经算出的极限
\[\lim_{x\to 0}\frac{(1+x)^{\mu}-1}{x}=\mu\]
就得到
\[f'(x)=\lim_{\Delta x\to 0}\frac{\Delta y}{\Delta x}=\mu x^{\mu-1}\]
\end{proof}

特殊情形:
\begin{enumerate}
    \item 若$f(x)=\frac{1}{x}=x^{-1},\; (x\ne 0)$,则$f'(x)=(-1)x^{-2}=-\frac{1}{x^2}$
    \item 若$f(x)=\sqrt{x}=x^{\tfrac{1}{2}},\; (x\ge 0)$,则$f'(x)=\frac{1}{2}x^{-\tfrac{1}{2}}=\frac{1}{2\sqrt{x}}$
\end{enumerate}

\subsubsection{若$f(x)=\log_a x\; (0<a\ne 1,\; 0<x<+\infty)$,则$f'(x)=\frac{\log_a e}{x}$}

\begin{proof}
\[\begin{split}
    \frac{\Delta y}{\Delta x}&=\frac{\log_a(x+\Delta x)-\log_a x}{\Delta x}\\
    &=\frac{1}{x}\cdot \frac{\log_a\left(1+\frac{\Delta x}{x}\right)}{\frac{\Delta x}{x}}\\
    &=\frac{1}{x}\cdot \log_a\left(1+\frac{\Delta x}{x}\right)^{\tfrac{1}{\Delta x/x}}
\end{split}\]  

利用极限$\Lim_{t\to 0}(1+t)^{1/t}=e$,以及对数函数是连续函数,得到
\[\begin{split}
    f'(x)&=\lim_{\Delta x\to 0}\frac{\Delta y}{\Delta x}=\frac{1}{x}\lim_{\tfrac{\Delta x}{x}\to 0}\log_a\left(1+\frac{\Delta x}{x}\right)^{\tfrac{1}{\Delta x/x}}\\
    &=\frac{1}{x}\log_a\left[\lim_{\tfrac{\Delta x}{x}\to 0}\left(1+\frac{\Delta x}{x}\right)^{\tfrac{1}{\Delta x/x}}\right]\\
    &=\frac{1}{x}\log_a e=\frac{\log_a e}{x}
\end{split}\]

特别:当$f(x)=\ln x$时,$f'(x)=\frac{1}{x}$。
\end{proof}

上面的结果表明对数函数($a>1$时)的增大速度是与自
变数的值成反比的,当自变数无限增大时,增大速度就保持正值而趋向于0. 由于自然对数的导数比较简单,所以在理论研究中常常采用自然对数。

\subsubsection{若$f(x)=a^x\; (0<a\ne 1,\; -\infty<x<+\infty)$,则$f'(x)=a^x\cdot \ln a$}

\begin{proof}
    \[\frac{\Delta y}{\Delta x}=\frac{a^{x+\Delta x}-a^x}{\Delta x}=a^x\cdot \frac{a^{\Delta x}-1}{\Delta x}\]
利用$\Lim_{\Delta x\to 0}a^{\Delta x}=1$,则可设$a^{\Delta x}=1+\alpha$,且当$\Delta x\to 0$时,有$\alpha\to 0$. 在此等式的两边取对数,得:
\[\Delta x=\log_a(1+\alpha)\]
于是
\[\frac{\Delta y}{\Delta x}=a^x\cdot \frac{\alpha}{\log_a(1+\alpha)}=a^x\cdot \frac{1}{\log_a(1+\alpha)^{1/\alpha}}\]
因此
\[\begin{split}
    f'(x)&=\lim_{\Delta x\to 0}\frac{\Delta y}{\Delta x}=a^x\cdot \frac{1}{\Lim_{\alpha\to 0}\log_a(1+\alpha)^{1/\alpha}}\\
    &=a^x\cdot \frac{1}{\log_a\left[\Lim_{\alpha\to 0}(1+\alpha)^{1/\alpha}\right]}\\
    &=a^x\cdot \frac{1}{\log_a e}
\end{split}\]
又因为$\ln a=\frac{1}{\log_a e}$,所以
\[f'(x)=a^x\cdot \ln a\]

特别,若$f(x)=e^x$,则$f'(x)=e^x$。
\end{proof}

这个事实表明指数函数(当$a>1$时)的增大速度与函数值成正比例,当底数为$e$时,导数的结果特别简单。


\subsubsection{若$f(x)=\sin x$, 则$f'(x)=\cos x$}
\begin{proof}
    \[\begin{split}
\frac{\Delta y}{\Delta x}&=\frac{\sin(x+\Delta x)-\sin x}{\Delta x}\\
&=\frac{2\cos\left(x+\frac{\Delta x}{2}\right)\sin\frac{\Delta x}{2}}{\Delta x}\\
&=\cos\left(x+\frac{\Delta x}{2}\right)\cdot \frac{\sin\frac{\Delta x}{2}}{\frac{\Delta x}{2}}
    \end{split}\]
因此
\[\begin{split}
    \lim_{\Delta x\to 0}\frac{\Delta y}{\Delta x}&=\lim_{\Delta x\to 0}\left[\cos\left(x+\frac{\Delta x}{2}\right)\cdot \frac{\sin\frac{\Delta x}{2}}{\frac{\Delta x}{2}}\right]\\
    &=\lim_{\Delta x\to 0}\cos\left(x+\frac{\Delta x}{2}\right)\cdot \lim_{\Delta x/2\to 0}\frac{\sin\frac{\Delta x}{2}}{\frac{\Delta x}{2}}
\end{split}\]
根据$\cos x$的连续性与极限$\Lim_{t\to 0}\frac{\sin t}{t}=1$, 得到
$f' (x) =\cos x$.
\end{proof}

\subsubsection{若$f(x)=\cos x$, 则$f'(x)=-\sin x$}
\begin{proof}
\[\begin{split}
\frac{\Delta y}{\Delta x}&=\frac{\cos(x+\Delta x)-\cos x}{\Delta x}\\
&=\frac{-2\sin\left(x+\frac{\Delta x}{2}\right)\cdot \sin\frac{\Delta x}{2}}{\Delta x}\\
&=-\sin\left(x+\frac{\Delta x}{2}\right)\cdot \frac{\sin\frac{\Delta x}{2}}{\frac{\Delta x}{2}}
    \end{split}\]
因此
\[\begin{split}
    \lim_{\Delta x\to 0}\frac{\Delta y}{\Delta x}&=-\lim_{\Delta x\to 0}\left[\sin\left(x+\frac{\Delta x}{2}\right)\cdot \frac{\sin\frac{\Delta x}{2}}{\frac{\Delta x}{2}}\right]\\
    &=-\lim_{\Delta x\to 0}\sin\left(x+\frac{\Delta x}{2}\right)\cdot \lim_{\Delta x/2\to 0}\frac{\sin\frac{\Delta x}{2}}{\frac{\Delta x}{2}}\\
    &=-\sin  x
\end{split}\]
\end{proof}

现在将上面的导数公式列成下表:
\begin{enumerate}
    \item $\frac{\dd c}{\dd x}=0$
    \item $\frac{\dd(kx+b)}{\dd x}=k$
    \item $\frac{\dd x^n}{\dd x}=nx^{n-1}$\quad ($n$为自然数)
    \item $\frac{\dd x^\mu}{\dd x}=\mu x^{\mu-1}$\quad ($\mu$为任意实数)
    \item $\frac{\dd\log_a x}{\dd x}=\frac{\log_a e}{x},\qquad \frac{\dd \ln x}{\dd x}=\frac{1}{x}$
    \item $\frac{\dd a^x}{\dd x}=a^x\cdot \ln a,\qquad \frac{\dd e^x}{\dd x}=e^x$
    \item $\frac{\d\sin x}{\dd x}=\cos x,\qquad \frac{\dd \cos x}{\dd x}=-\sin x$
\end{enumerate}

\subsection{求导的基本法则}

现在要建立一些求导数的公式,用这些公式等函数的导数。

\begin{blk}
{定理1} 函数和的导数等于各函数的导数的和,即:
\[[f (x) +g (x) ] '=f' (x) +g' (x)\] 
\end{blk}

\begin{proof}
    设$u(x)=f(x)+g(x)$, 任意固定$x$, 作取非零数$\Delta x$,有 
\[\begin{split}
    \frac{\Delta u(x)}{\Delta x}&=\frac{[f(x+\Delta x)+g(x+\Delta x)]-[f(x)+g(x)]}{\Delta x}\\
    &=\frac{f(x+\Delta x)-f(x)}{\Delta x}+\frac{g(x+\Delta x)-g(x)}{\Delta x}\\
    &=\frac{\Delta f(x)}{\Delta x}+\frac{\Delta g(x)}{\Delta x}
\end{split}\]
令$\Delta x\to 0$,即得:
\[\begin{split}
    [f(x)+g(x)]'&=\lim_{\Delta x\to 0}\frac{\Delta u(x)}{\Delta x}\\
    &=\lim_{\Delta x\to 0}\frac{\Delta f(x)}{\Delta x}+\lim_{\Delta x\to 0}\frac{\Delta g(x)}{\Delta x}\\
    &=f'(x)+g'(x)
\end{split}\]
\end{proof}

这个法则可以推广到任意有限个函数。

\begin{blk}
{定理2} $[A\varphi (x)]'=A\varphi'(x)$, 其中$A$是常数,即常因数不因微分法而改变。    
\end{blk}

\begin{proof}
设$f(x)=A\varphi(x)$, 
\[\begin{split}
    [A\varphi(x)]'&=f'(x)=\lim_{\Delta x\to 0}\left[A\frac{\varphi(x+\Delta x)-\varphi(x)}{\Delta x}\right]\\
    &=A\lim_{\Delta x\to 0}\frac{\varphi(x+\Delta x)-\varphi(x)}{\Delta x}\\
    &=A\varphi'(x)
\end{split}\]
\end{proof}



\begin{example}
    求$3x^2-7x-1$的导数。
\end{example}

\begin{solution}
\[  (3x^2-7x-1)'=(3x^2)'+(-7x)'+(-1)'=6x-7\]
\end{solution}    

\begin{blk}{推论}
设$f(x)=a_nx^n+a_{n-1}x^{n-1}+\cdots+a_1x+a_0\; (a_n\ne 0)$,则
\[f'(x)=na_nx^{n-1}+(n-1)a_{n-1}x^{n-2}+\cdots+a_1\]
\end{blk}

\begin{example}
求过抛物线$y=3x^2+6$外一点$P(2,-9)$所引抛物线的两条切线的方程。
\end{example}

\begin{solution}
在点$x$处的抛物线的切线的斜率为
\[k=y'_x=6x\]
故过抛物线上一点$T(x_1,y_1)$的切线方程为
\[y-y_1=6x_1 (x-x_1)\] 

$\because\quad y_1=3x^2_1+6$

$\therefore\quad $切线方程可化简为
\[    y=6x_1x-3x_1^2+6\]
设切线通过$P(2,-9)$点,于是
\[9=12x_1-3x_1^2+6\]
即:$x_1^2-4x_1-5=0$

$\therefore\quad x_1=-1$或$5$,从而$y_1=9$或81.

故所求切线方程为:
\[y-9=-6(x+1)\quad\text{和}\quad y-81=30(x-5)\]
即:
\[y+6x-3=0\quad\text{和}\quad  y-30x+69=0\]
\end{solution}

\begin{example}
求三次函数$f(x)=ax^3+3bx^2+3cx+d$的图象与$x$轴相切的条件。
\end{example}

    
\begin{solution}
由函数$f(x)=ax^3+3bx^2+3cx+d$求得它的图象任意一点$x=x_0$处的切线斜率为
\[f' (x) =3ax^2_0+6bx_0+3c=3 (ax^2_0+2bx_0+c) \]
    
要此曲线与$x$轴相切于$(x_0,0)$点,$x_0$必须满足条件
\begin{numcases}{}
    ax^3_0+3bx^2_0+3cx_0+d=0\\
    ax^2_0+2bx_0+c=0
\end{numcases}
将(2.1)改写成
\begin{equation}
    x_0(ax^2_0+2bx_0+c)+bx_0^2+2cx_0+d=0
\end{equation}
(2.2)代入得
\begin{equation}
    bx_0^2+2cx_0+d=0
\end{equation}
于是解(2.1)和(2.2)即解(2.2)和(2.4),要方程(2.2)和(2.4)有公共解,必须
\[D=\begin{vmatrix}
    a&2b\\b&2c
\end{vmatrix}=2(ac-b^2)\ne 0\]
于是
\[x^2_0=\frac{c^2-bd}{b^2-ac},\qquad x_0=\frac{ad-bc}{2(b^2-ac)}\; (b^2-ac\ne 0)\]
所以,$a,b,c,d$必须适合条件
\[\begin{cases}
    b^2-ac\ne 0\\
    \frac{c^2-bd}{b^2-ac}=\frac{(ad-bc)^2}{4(b^2-ac)^2}
\end{cases}\]
即
\[\begin{cases}
    b^2-ac\ne 0\\
   (ad-bc)^2 =4(b^2-ac)(c^2-bd)
\end{cases}\]
\end{solution}

\begin{blk}
    {定理3} 二函数之积的导数为两项之和,其中第一项是第一个因子的导数与第二个因子的乘积,而第二项是第二个因子的导数与第一个因子的乘积,即
\[[f (x) \cdot g (x) ] '=f' (x) \cdot g (x) +f (x) \cdot g' (x) \]
\end{blk}

\begin{proof}
    设 $h(x)=f(x)\cdot g(x)$, 任取非零数$\Delta x$, 则
\[\begin{split}
    [f(x)\cdot g(x)]'&=\lim_{\Delta x\to 0}\frac{\Delta h(x)}{\Delta x}\\
    &=\lim_{\Delta x\to 0}\frac{f(x+\Delta x)g(x+\Delta x)-f(x)g(x)}{\Delta x}\\
    &=\lim_{\Delta x\to 0}\frac{1}{\Delta x}\Bigl[f(x+\Delta x)g(x+\Delta x)-f(x)g(x+\Delta x)\\
    &\qquad\qquad +f(x)g(x+\Delta x)-f(x)g(x) \Bigr]\\
    &=\lim_{\Delta x\to 0}\frac{f(x+\Delta x)-f(x)}{\Delta x}\cdot g(x+\Delta x)\\
    &\qquad +\lim_{\Delta x\to 0}f(x)\cdot \frac{g(x+\Delta x)-g(x)}{\Delta x}\\
    &=f' (x) \cdot g (x) +f (x) \cdot g' (x) 
\end{split}\]
\end{proof}

重复应用这个定理,我们便可以求两个以上的函数之积的导数,例如
\[\frac{\dd }{\dd x}(uvw)=(vw)\frac{\dd u}{\dd x}+u\frac{\dd }{\dd x}(vw)\]
但是
\[\frac{\dd }{\dd x}(vw)=w\frac{\dd v}{\dd x}+v\frac{\dd w}{\dd x}\]
代入上式便得到
\[\frac{\dd }{\dd x}(uvw)=vw\frac{\dd u}{\dd x}+uw\frac{\dd v}{\dd x}+uv\frac{\dd w}{\dd x}\]


\begin{example}
    设$f(x)=\sqrt{x}(x^2-3)$,求$f'(x)$
\end{example}


\begin{solution}
    \[\begin{split}
f'(x)&=\left[\sqrt{x}(x^2-3)\right]'\\
&=\left(\sqrt{x}\right)'(x^2-3)+\sqrt{x}(x^2-3)'\\
&=\frac{7}{2\sqrt{x}}(x^2-3)+\sqrt{x}\cdot 2x = \frac{5x^2-3}{2\sqrt{x}}
    \end{split}\]
\end{solution}

\begin{example}
    设$f(x)=\frac{1}{2}\sin^2 x+x\sin x+\frac{7}{x^2}$,求$f'(x)$
\end{example}

\begin{solution}
\[\begin{split}
    f'(x)&=\left(\frac{1}{2}\sin^2 x\right)'+(x\sin x)'+\left(\frac{7}{x^2}\right)'\\
    &=\frac{1}{2}\left[(\sin x)'\sin x+\sin x(\sin x)'\right]+(\sin x+x\cos x)-\frac{14}{x^3}\\
    &=\sin x\cos x+\sin x+x\cos x-\frac{14}{x^3}
\end{split}\]
\end{solution}

\begin{blk}
    {定理4} 两个函数的商的导数,等于分子的导数与分母的积,减去分母的导数与分子的积,再除以分母的平方,即
\[\left(\frac{u(x)}{v(x)}\right)'=\frac{v(x)\cdot u'(x)-u(x)\cdot v'(x)}{[v(x)]^2}\]
\end{blk}

\begin{proof}
    设$f(x)=\frac{u(x)}{v(x)}$,任取非零数$\Delta x$,则
\[\begin{split}
    \left(\frac{u(x)}{v(x)}\right)'&=\lim_{\Delta x\to 0}\frac{\Delta f(x)}{\Delta x}=\lim_{\Delta x\to 0}\frac{\frac{u(x+\Delta x)}{v(x+\Delta x)}-\frac{u(x)}{v(x)}}{\Delta x}\\
    &=\lim_{\Delta x\to 0}\frac{1}{\Delta x}\left\{\frac{u(x+\Delta x)\cdot v(x)-u(x)\cdot v(x+\Delta x)}{v(x+\Delta x)\cdot v(x)}\right\}\\
    &=\lim_{\Delta x\to 0}\frac{1}{\Delta x}\left\{\frac{v(x)[u(x+\Delta x)-u(x)]}{v(x)v(x+\Delta x)}-\frac{u(x)[v(x+\Delta x)-v(x)]}{v(x)v(x+\Delta x)}\right\}\\
    &=\frac{v(x)\Lim_{\Delta x\to 0}\frac{u(x+\Delta x)-u(x)}{\Delta x}}{v(x)\Lim_{\Delta x\to 0}v(x+\Delta x)}-\frac{u(x)\Lim_{\Delta x\to 0}\frac{v(x+\Delta x)-v(x)}{\Delta x}}{v(x)\Lim_{\Delta x\to 0}v(x+\Delta x)}\\
    &=\frac{v(x) u'(x)-u(x) v'(x)}{v^2(x)}
\end{split}\]
\end{proof}


\begin{example}
求证:$(\tan x)'=\sec^2 x,\qquad (\cot x)'=-csc^2 x$
\end{example}

\begin{proof}    
\[\begin{split}
    (\tan x)'&=\left(\frac{\sin x}{\cos x}\right)'\\
    &=\frac{\cos x(\sin x)'-\sin x(\cos x)'}{\cos^2 x}\\
    &=\frac{\cos^2 x+\sin^2 x}{\cos^2x}\\
    &=\frac{1}{\cos^2 x}=\sec^2 x
\end{split}\]
\[(\cot x)'=\left(\frac{\cos x}{\sin x}\right)'=\frac{-\sin^2 x-\cos^2 x}{\sin^2 x}=-\csc^2 x\]    
\end{proof}

\begin{example}
    求$\left(\frac{2x-1}{x^2+1}\right)'$
\end{example}

\begin{solution}
\[\begin{split}
    \left(\frac{2x-1}{x^2+1}\right)'&=\frac{(x^2+1)(2x-1)'-(2x-1)(x^2+1)'}{(x^2+1)^2}\\
&=\frac{(x^2+1)\cdot 2-(2x-1)(2x)}{(x^2+1)^2}\\
&=\frac{2(x^2+1)-(4x^2-2x)}{(x^2+1)^2}\\
&=\frac{-2(x^2-x-1)}{(x^2+1)^2}
\end{split}\]
\end{solution}

\begin{ex}
\begin{enumerate}
    \item  求下面各函数的导函数:
\begin{multicols}{2}
\begin{enumerate}
    \item $3 x^{3}-1$
    \item  $x+2 x^{2}+3 x^{3}$
    \item  $1-\frac{1}{x^{2}}$
    \item $\frac{x^{2}+1}{x}$
    \item  $1+\frac{1}{x}-\frac{1}{x^{2}}$
    \item  $\frac{2 x^{5}-7 x^{3}-3 x^{2}}{6 x^{2}}$
    \item  $(x+2)\left(x^{2}+1\right)$
    \item  $x+\sqrt{x}$
    \item $x-\frac{1}{\sqrt{x}}$
    \item  $\frac{x+1}{\sqrt{x}}$
    \item  $\frac{2 x \sqrt{x}-x^{\tfrac{5}{2}}+3 x^{\tfrac{1}{2}}}{-\sqrt{x}}$ \item  $e^{x}-e^{-x}$
    \item $\frac{x^{n}-x^{-n}}{x} $
    \item $\sin x+\cos x$
    \item $\sin x-\cos x$
\end{enumerate}
\end{multicols}
\item  求下面各函数的导函数:
\begin{multicols}{2}
\begin{enumerate}
    \item $\left(x^{2}+1\right)^{3} \sqrt{x}$
    \item $\frac{x+1}{\sqrt{x}}$
    \item  $x \sin x$
    \item $x \cot  x$
    \item  $\frac{\tan  x}{x}$;
    \item  $x \log _{2} x+x^{2} \tan  x$
    \item  $\frac{x^{2}}{x^{3}+c^{3}}$
    \item  $x^{3} e^{-x}$;
    \item  $a^{-x} \sin x$ 
    \item $a^{x}\{(x+1) \cos x+\log_2 x\}$
    \item $\frac{\sqrt{x}+1}{2 x+1}$ 
    \item $\frac{2 x-1}{x^{2}+1}$
    \item $\frac{(x-3)(x-4)}{(x-5)}$
    \item  $\frac{x^{2}\left(x^{2}-1\right)\left(x^{3}-1\right)}{x+1}$
    \item  $\frac{x}{x+1}-\frac{1}{x-1}$
    \item  $x \sin x \cos x$
    \item $\sec x $ 
    \item $\csc x$
    \item $\frac{\sin x}{1+\tan  x}$
\end{enumerate}
\end{multicols}

\item 已知抛物线$y=x^2-x$上一点的切线平行于直线$y=x$。求此点的坐标。
\item 求曲线$y=x^3-x^2$上这样的点,使过该点的切线与$x$轴平行。
\item 若抛物线$y=ax^2+bx+c$通过原点,且过该点的切线的斜率等于2, 又抛物线过$(1, 1)$点,求$a$、$b$、$c$.
\item 对于函数$f(x)=x^3+ax^2+bx+c$, 方程$f(x)+\frac{1}{2}x=0$
有不相等的三根$2,\alpha,\beta$, 若$f'(\alpha)=f'(\beta)$, $f'(1)=0$, 求$a,b,c$.
\item 求三次函数$y=f(x)$, 使它同时满足下面的条件:
\begin{enumerate}
    \item 用$x+1$去除$f(x)$与用$(x-1)(x-2)$去除,$f(x)$所得余式相同;
    \item 过曲线$y=f(x)$上的点$(1,f(1))$的切线方程是
$y=-2x+1$
\end{enumerate}
\item 
过$(2, 0)$点求与曲线$y=\frac{1}{x}$相切的直线方程。
\item 
\begin{enumerate}
\item 在抛物线$y=x^2$上求这样一点,使过该点的切线与过$(a,a^2)$点的切线垂直;
\item 求抛物线两条正交切线交点的轨迹方程.
\end{enumerate}

\item 设抛物线方程为$y=x^2+ax+b$, 试问点$(x_0,y_0)$位于
何处时,可以从点$(x_0,y_0)$对此抛物线作出两条切线或一条切线,或作不出切线?
\item 求抛物线$y=x^2+ax$, $y=x^2+bx\; (a\ne b)$的公切线的方
程。
\end{enumerate} 
\end{ex}

\subsection{复合函数的求导法则}
\begin{blk}
    {定理} 假设函数$y=g(x)$在点$x$可导,而函数$z=f(y)$在点$g(x)$可导,那么复合函数$z=\varphi(x)=f(g(x))$在点$x$可导,并且
\[[f(g(x))]'=f'(g(x))\cdot g'(x)\]
简写成$[f(g(x))]'=f'_g\cdot g'_x$,或者写成
\[\frac{\dd z}{\dd x}=\frac{\dd z}{\dd y}\cdot \frac{\dd y}{\dd x}\]
\end{blk}

\begin{proof}
    设$D_g$, $D_f$分别是函数$g(x),f(y)$的定义域,令$x\in D_g$, $g(x)\in D_f$, 任取非零数$\Delta x$, 且使$g(x+\Delta x)\in D_f$,于是$\Delta g=g(x+\Delta x)-g(x)$, 又根据$g(x)$在点$x$连续,故当$x\to 0$时,$\Delta y\to 0$. 因为$g(x)$在点$x$可导,可以设
\begin{equation}
    \alpha=\frac{\Delta y}{\Delta x}-g'(x)
\end{equation}
而且由上式知道
\[\lim_{\Delta x\to 0}\alpha=\lim_{\Delta x\to 0}\left(\frac{\Delta y}{\Delta x}-g'(x)\right)=g'(x)-g'(x)=0\]
把(2.5)改写成
\begin{equation}
    \Delta y=g'(x)\Delta x+\alpha\Delta x
\end{equation}
其中$\alpha$随$\Delta x$一同趋向于零,且$\alpha \Delta x$是比$\Delta x$更小的量。

\begin{enumerate}
    \item 若$g'(x)\ne 0$,只要$\Delta x$够小,由(2.6)知$\Delta y\ne 0$,那么
\[\begin{split}
    \frac{\varphi(x+\Delta x)-\varphi(x)}{\Delta x}&=\frac{f(g(x+\Delta x))-f(g(x))}{g(x+\Delta x)-g(x)}\cdot \frac{g(x+\Delta x)-g(x)}{\Delta x}\\
    &=\frac{f(g(x)+\Delta y)-f(g(x))}{\Delta y}\cdot \frac{\Delta y}{\Delta x}
\end{split}\]
令$\Delta x\to 0$,由上式得
\[\varphi'(x)=f'(g(x))\cdot g'(x)\]
即
\[[f(g(x))]'=f'(g(x))\cdot g'(x)\]

\item 如果$g'(x)=0$,则当$\Delta x\to 0$时,有两种情形:
\begin{enumerate}
    \item 若$\Delta y=g(x+\Delta x)-g(x)=0$,则
\[\begin{split}
    \frac{\varphi(x+\Delta x)-\varphi(x)}{\Delta x}&=\frac{f(g(x+\Delta x))-f(g(x))}{\Delta x}\\
    &=\frac{f(g(x))-f(g(x))}{\Delta x}=0
\end{split}\]
\item 若$\Delta y\ne 0$,则
\[\frac{\varphi(x+\Delta x)-\varphi(x)}{\Delta x}=\frac{f(g(x+\Delta y))-f(g(x))}{\Delta y}\cdot \frac{\Delta y}{\Delta x}\]
令$\Delta x\to 0$,得到
\[\varphi'(x)f'(g(x))\cdot 0=0\]
\end{enumerate}
合并上述两种情形,都有
\[\varphi'(x)=[f(g(x))]'=0\]
于是定理得证。
\end{enumerate}
\end{proof}

这个定理是说:复合函数对自变量的导数,等于已知函数对中间变量的导数,乘以中间变量对自变量的导数。


\begin{example}
    设$\varphi (x)=2^{\sin x}$, 求$\varphi'(x)$。
\end{example}

    
\begin{solution}
    把$\varphi (x)$看成$u=\sin x$和$f(u)=2^u$的复合函数,于是$\varphi  (x) =2^{\sin x}=f (\sin x)$.

依上述定理,得
\[\begin{split}
  \varphi' (x) &= [f (u) ]'\cdot u'_x= (2^u)'\cdot (\sin x)'\\
&=2^u\ln2 \cdot \cos x\\
&=2^{\sin x}\cdot \cos x \cdot \ln2  
\end{split}\]
\end{solution}

\begin{example}
    设$y=\sin(x^2+x+1)\cdot \cos(x^3+2x^2)$, 求$y'_x$.
\end{example}

\begin{solution}
    设$u=x^2+x+1$, $v=x^3+2x^2$, 则
\[\begin{split}
  y'&= (\sin u)'\cdot u'_x\cdot \cos v+\sin u\cdot (\cos v)'\cdot v'_x\\
&=\cos u\cdot u_x'\cdot \cos v-\sin u\cdot \sin v\cdot  v'_x\\
&= (2x+1) \cos (x^2+x+1) \cos (x^3+2x^2)\\
&\qquad - (3x^2+4x) \sin (x^2+x+1) \sin (x^3+2x^2)  
\end{split}\]
\end{solution}
    
\begin{example}
设$f(x)=\frac{x}{\sqrt{9+x^2}}$,求$f'(x)$.
\end{example}


\begin{solution}
\[\begin{split}
    f'(x)&=\frac{\sqrt{9+x^2}(x)'-x\left(\sqrt{9+x^2}\right)'}{9+x^2}\\
    &=\frac{\sqrt{9+x^2}-x\cdot \frac{1}{2}(9+x^2)^{-\tfrac{1}{2}}(9+x^2)'}{9+x^2}\\
    &=\frac{\sqrt{9+x^2}-\frac{x^2}{\sqrt{9+x^2}}}{9+x^2}=\frac{9}{\left(9+x^2\right)^{3/2}}
\end{split}\]
\end{solution}
    
\begin{blk}
{推论1} 如果函数有两个以上的中间变量,上面的求导法则可以推广使用。
\end{blk}

例如,设$y=f(u)$, $u=g(v)$, $v=h(x)$, 那么对于复合函数。$y=\varphi (x)=f\{g[h(x)]\}$的导数可以这样来求:
\[\frac{\dd y}{\dd x}=\frac{\dd y}{\dd u}\cdot \frac{\dd u}{\dd v}\cdot \frac{\dd v}{\dd x}\]
即:$y'_x=y'_u\cdot u'_v\cdot v'_x$

事实上,依前定理有
\[\frac{\dd y}{\dd x}=\frac{\dd y}{\dd u}\cdot \frac{\dd u}{\dd x}\]
这里$u=g[h(x)]$。再用同一定理
\[\frac{\dd u}{\dd x}=\frac{\dd u}{\dd v}\cdot \frac{\dd v}{\dd x}\]
所以
\[\frac{\dd y}{\dd x}=\frac{\dd y}{\dd u}\cdot \frac{\dd u}{\dd v}\cdot \frac{\dd v}{\dd x}\]


\begin{example}
    设$y=\ln\sqrt{5-2x+3x^4},\quad (5-2x+3x^4>0)$,求$y'_x$.
\end{example}

    
\begin{solution}
    设$y=\ln u$,$u=\sqrt{v}$,$v=5-2x+3x^4$,于是:
\[\begin{split}
    \frac{\dd y}{\dd x}&=\frac{\dd y}{\dd u}\cdot \frac{\dd u}{\dd v}\cdot \frac{\dd v}{\dd x}\\
    &=\frac{1}{u}\cdot \left(\frac{1}{2}v^{-\tfrac{1}{2}}\right)\cdot (-2+12x^3)\\
    &=\frac{1}{\sqrt{5-2x+3x^4}}\cdot \frac{1}{2}\frac{1}{\sqrt{5-2x+3x^4}}\cdot (-2+12x^3)\\
    &=\frac{6x^3-1}{5-2x+3x^4}
\end{split}\]
\end{solution}

求复合函数的导数,关键在于分析清楚函数的复合关系,适当选定中间变量,根据复合函数求导法则由外向里逐层求导,直到最后一个中间变量对自变量求导为止,每次求导时,必须明确是哪个变量对哪个变量求导,所有这些导数的乘积就是复合函数的导数。

\begin{blk}{推论}
    \[(\ln|g(x)|)'=\frac{g'(x)}{g(x)}\quad (g(x)\ne 0)\]
\end{blk}

\begin{proof}
  \[\ln|g(x)|\text{有意义}\quad \Longleftrightarrow\quad g(x)\ne 0\]  
又
\[|g(x)|'\text{存在}\quad \Longleftrightarrow\quad g(x)\ne 0\]
所以由
\[|g(x)|'=\begin{cases}
    g'(x) ,  & g(x)>0\\
    [-g(x)]'=-g'(x), & g(x)<0
\end{cases}\]
和复合函数求导法则,便得
\[[\ln|g(x)|]'=\frac{1}{|g(x)|}\cdot |g(x)|=\frac{g'(x)}{g(x)}\quad (g(x)\ne 0)\]
\end{proof}

\begin{example}
    求$(\ln|x^2-5x+4|)'$
\end{example}

\begin{solution}
\[(\ln|x^2-5x+4|)'=\frac{(x^2-5x+4)'}{x^2-5x+4}=\frac{2x-5}{x^2-5x+4}\]
其中:$x^2-5x+4\ne 0$,即$x\ne 1$和$x\ne 4$.
\end{solution}



\begin{example}
求$\left[\frac{(x^2+1)^{\tfrac{1}{2}}(x-1)^2}{(x+1)^{\tfrac{3}{2}}}\right]'$
\end{example}

    
\begin{solution}
设$y=\frac{(x^2+1)^{\tfrac{1}{2}}(x-1)^2}{(x+1)^{\tfrac{3}{2}}}$,则
\[\begin{split}
    \ln|y|&=\frac{1}{2}\ln(x^2+1)+2\ln|x-1|-\frac{3}{2}\ln|x+1|\\
    \frac{1}{y}\cdot y'&=\frac{1}{2}\frac{2x}{x^2+1}+\frac{2}{x-1}-\frac{2}{2(x+1)}\\
    &=\frac{3x^3+7x^2-x+7}{2(x^2+1)(x-1)(x+1)}
\end{split}\]    
\[\begin{split}
    y'&=\frac{3x^3+7x^2-x+7}{2(x^2+1)(x-1)(x+1)}\cdot \frac{(x^2+1)^{\tfrac{1}{2}}(x-1)^2}{(x+1)^{\tfrac{3}{2}}}\\
    &=\frac{(3x^3+7x^2-x+7)(x-1)}{2(x^2+1)^{\tfrac{1}{2}}(x+1)^{\tfrac{5}{2}}}
\end{split}\]
    
\end{solution}

\begin{example}
设$f(x)=u(x)^{v(x)}\quad (u(x)>0)$,求$f'(x)$.
\end{example}

\begin{solution}
先求此函数的对数,有
\[\ln f(x)=v(x)\ln u(x)\]
两边对$x$求导,即得
\[-\frac{1}{f(x)}\cdot f'(x)=v'(x)\ln u(x)+v(x)\cdot\frac{1}{u(x)}\cdot u'(x)\]
    所以
\[f'(x)=u(x)^{v(x)}\left[v'(x)\ln u(x)+\frac{v(x)u'(x)}{u(x)}\right]\]
\end{solution}

\begin{example}
一人在放风筝,且知风筝在80米的高度飘行,放线速度为2米/秒,求当线长为100米时,风筝和此人之间的水平距离的变率。
\end{example}


\begin{solution}
    设在某时刻$t$, 线长为$\ell(t)$米,水平距离为$x(t)$米,根据勾股定理得到绳长和水平距离的关系式
\[\ell^2 (t) =x^2 (t) +6400\]
两边对时间$t$求导数,得到
\[\frac{\dd \ell^2}{\dd \ell}\cdot \frac{\dd \ell}{\dd t}=\frac{\dd(x^2+6400)}{\dd x}\cdot \frac{\dd x}{\dd t}\]
即
\[\ell\cdot \frac{\dd \ell}{\dd t}=x\cdot \frac{\dd x}{\dd t}\]
但是已知$\frac{\dd\ell}{\dd t}=2({\rm m/s})$,又当$\ell=100$时,
\[x=\sqrt{100^2-6400}=\sqrt{3600}=60\]
于是
\[60\frac{\dd x}{\dd t}=100\times 2\]

$\therefore\quad \frac{\dd x}{\dd t}=\frac{10}{3}=3\frac{1}{3}({\rm m/s})$

答:当$\ell=100$米时,风筝和人之间的水平距离的变率为$3\frac{1}{3}$(米/秒)。
\end{solution}

\begin{ex}
\begin{enumerate}
    \item 求下面函数对于 $x$ 的导数:
\begin{multicols}{2}
\begin{enumerate}
    \item $(x+1)^{2} $ 
\item $(1-3 x)^{3} $
\item  $\sqrt{x+1}$
\item  $\frac{1}{\sqrt{x+1}}$
\item  $\sqrt{x^{2}+1}$
\item  $\frac{1}{\sqrt{x^{2}+1}}$
\item  $\sqrt{2 x^{2}+x-1}$
\item  $\frac{1}{\sqrt{\tan x}}$
\item  $\frac{1}{\sin ^{3} x} $
\item  $\cos (a x+b)$
\item  $\cos ^{3}(a x+b) $
\item  $\cot \sqrt{x} $
\item  $\sec \left(\frac{1}{x}\right)$
\item  $x \sqrt{\sin x}$
\item  $x^{2} \sqrt{\sec 2x}$
\item  $\ln \left(1+x^{2}\right)$
\item $\log _{a}(\sin x)$
\item $\log _{a}(\sec x)$
\item  $\sqrt{\ln x}$
\item $e^{a x^{2}}$
\item $e^{1+x-x^{2}}$
\item  $a^{\sqrt{2x+1}}$
\end{enumerate}
\end{multicols}

 \item    对下面各函数先取对数再求导数:
 \begin{multicols}{2}
    \begin{enumerate}
        \item   $\frac{x-1}{(x+1)^{2}}$
        \item   $\frac{\sqrt{x-1}(x+1)}{x^{2}+1}$
        \item   $\frac{(x-3)(x-4)}{(x-5)}$
        \item   $\sqrt{\frac{x-1}{x+1}}$ 
        \item  $(x+2)^{2}(x+3)^{3}(x+4)^{4}$
\end{enumerate}
\end{multicols}
\item  求下列函数对于 $x$ 的微商:
\begin{enumerate}
\begin{multicols}{2}
  \item $e^{2 x}(2 \cos 3 x+3 \sin 3 x)$
\item $\ln \left(x+\sqrt{x^{2} \pm a^{2}}\right)$
\item $\frac{2 x+1}{\sqrt{3-4 x-4 x^{2}}}$  
\item $\cos[\ln(x^2+9)]$
\item $\ln(1-\tan^2 x)$
\item $x^x+(\ln x)^x$
\end{multicols}
\item $(1-x)^{\tfrac{1}{2}}(1+3x)^{\tfrac{3}{20}}(1-2x)^{-\tfrac{2}{5}}$
\end{enumerate}

\item 若$y=e^{ax}\cos(bx+c)$,求证:
\[\frac{1}{\sqrt{a^2+b^2}}\frac{\dd y}{\dd x}=e^{ax}\cos(bx+c+r)\]
其中:$\tan r=b/a$.

\item 设$f(x)=\begin{cases}
    x^2\sin\frac{1}{x},& x\ne 0\\
    0,& x=0
\end{cases}$,求$f(x)$的导函数。

(提示:对于$x\ne 0$,可用本节的方法得出,而$f'(0)$需直接由定义得出)。

\item 有一个长度为5米的梯子贴靠在铅直的墙上,假设其下端
沿地板以3米/秒的速率离开墙脚而滑动,则
\begin{enumerate}
    \item 当其下端离开墙脚1.4米时,梯子上端下滑之速率为多少?
    \item 何时梯子的上下端能以相同的速率移动?
    \item 何时其上端下滑之速度为4米/秒?
\end{enumerate}

\item 求曲线$y=\cos^2\sqrt{x}$在点$x=0$处的切线方程.
\item 在$\triangle ABC$中,$a=7$cm, $b=5$cm, $c=8$cm, $P$为$AB$边上
一动点,从$A$往$B$以每秒2厘米的速度运动.当$P$点在
$AB$边中点时,$CP$长度的变率是多少?
\item 求曲线$y=2\left(e^x+\frac{1}{5}e^{-3x}\right)$在横坐标$x_0=0$的点的切线方程.
\item 试由公式$1+x+x^2+\cdots+x^n=\frac{x^{n+1}-1}{x-1}$分别导出$1+2x+3x^2+\cdots+nx^{n-1}$与$x+2^2x^2+3^2x^3+\cdots +n^2x^n$的公式.
\end{enumerate}  
\end{ex}


\subsection{反函数求导法则}

\begin{blk}
    {定理} 若函数$y=f(x)$在$x_0$的某一邻域内连续,且严格单调,又函数$f(x)$在点$x_0=f^{-1}(y_0)$的导数$f'(x_0)$存在并且异于零,则反函数$x=f^{-1}(y)$在对应点$y_0=f(x_0)$
的导数存在,并且等于$\frac{1}{f'(x_0)}=\frac{1}{f'(f^{-1}(y_0))}$
,即
\[[f^{-1}(y_0)]'=\frac{1}{f'(x_0)}=\frac{1}{f'[f^{-1}(y_0)]}\]
\end{blk}

\begin{proof}
给数值$y=y_0$以任意增量$\Delta y$, 则反函数$x=f^{-1}(y)$也得到对应增量$\Delta x$, 当$\Delta y\ne 0$时,由于$y=f(x)$的严格单调性,从而$x=f^{-1}(y)$也是严格单调的,也有$\Delta x\ne 0$, 于是有
\[\begin{split}
    \frac{f^{-1}(y_0+\Delta y)-f^{-1}(y_0)}{\Delta y}&=\frac{\Delta x}{f(x_0+\Delta x)-f(x_0)}\\
    &=\frac{1}{\frac{f(x_0+\Delta x)-f(x_0)}{\Delta x}}
\end{split}\]
现在令$\Delta y\to 0$,则由于反函数$x=f^{-1}(y)$也是连续的,故增量$\Delta x\to 0$,于是
\[\lim_{\Delta y\to 0}\frac{f^{-1}(y_0+\Delta y)-f^{-1}(y_0)}{\Delta y}=\frac{1}{\Lim_{\Delta y\to 0}\frac{f(x_0+\Delta x)-f(x_0)}{\Delta x}}\]
上面等式右边的分母就趋于极限$f'(x_0)\ne 0$,而左边按定义就是$[f^{-1}(y_0)]'$,因此
\[[f^{-1}(y_0)]'=\frac{1}{f'(x_0)}\]
再将$x_0=f^{-1}(y_0)$代入上式,得到
\[[f^{-1}(y_0)]'=\frac{1}{f'[f^{-1}(y_0)]}\]
\end{proof}

\begin{blk}{推论}
\[\begin{split}
    (\arcsin x)'&=\frac{1}{\sqrt{1-x^2}},\qquad -1<x<1\\
    (\arccos x)'&=\frac{-1}{\sqrt{1-x^2}},\qquad -1<x<1\\
    (\arctan x)'&=\frac{1}{{1+x^2}},\qquad -\infty<x<\infty\\
    (\arccot x)'&=\frac{-1}{{1+x^2}},\qquad -\infty<x<\infty\\
\end{split}\]
\end{blk}

\begin{proof}
如果我们限制正弦函数$x=\sin y$的定义域为$\left(-\frac{\pi}{2},\frac{\pi}{2}\right)$,那么$x=\sin y$的反函数存在,即
\[    y=\arcsin x,\qquad x\in (-1, 1)\]
于是
\[    (\arcsin x)'=\frac{1}{(\sin y)'}=\frac{1}{\cos y}=\frac{1}{\cos(\arcsin x)}\]
由于$\arcsin x\in \left(-\frac{\pi}{2},\frac{\pi}{2}\right)$,
所以$\cos y>0$, 从而
\[    \cos (\arcsin x) =\sqrt{1-\sin^2 (\arcsin x)} =\sqrt{1-x^2}\]
    最后得到
\[  (\arcsin x)'=\frac{1}{\sqrt{1-x^2}},\qquad x\in (-1, 1) \]

我们除去了数值$x=\pm 1$, 因为在它的对应值$y=\pm\frac{\pi}{2}$
处导数$(\sin y)'=\cos y=0$.

在第四册, 我们已经证明$\arcsin x+\arccos x=\frac{\pi}{2}$,所以
\[(\arccos x)'=\left(\frac{\pi}{2}-\arcsin x\right)'=-\frac{1}{\sqrt{1-x^2}},\qquad -1<x<1\]
其余同理可证。
\end{proof}

\begin{example}
求$(\arcsec x)'$.
\end{example}


\begin{solution}
函数$x=\sec y$在区间$\left(0,\frac{\pi}{2}\right)$上,由1递增到$+\infty$,而在区间$\left(\frac{\pi}{2},\pi\right)$上,由$-\infty$递增到$-1$。因此,$x=\sec y$,$y\in\left(0,\frac{\pi}{2}\right)\cup \left(\frac{\pi}{2},\pi\right)$有反函数存在,即:
\[y=\arcsec x,\qquad x\in (-\infty,-1)\cup(1,+\infty)\]

依反函数求导法则,有
\[\begin{split}
    (\arcsec x)'&=\frac{1}{(\sec y)'}=\frac{1}{\sec y\cdot \tan y}\\
    &=\frac{1}{\sec(\arcsec x)\cdot \tan(\arcsec x)}\\
    &=\begin{cases}
        \frac{1}{x\sqrt{\sec^2(\arcsec x)-1}}, & 0<\arcsec x<\frac{\pi}{2}\\
        \frac{1}{x\left(-\sqrt{\sec^2(\arcsec x)-1}\right)}, & \frac{\pi}{2}<\arcsec x<\pi \\
    \end{cases}\\
    &=\begin{cases}
        \frac{1}{x\sqrt{x^2-1}}, & x>1\\
       - \frac{1}{x\sqrt{x^2-1}}, & x<-1
    \end{cases}
\end{split}\]
\end{solution}

\begin{example}
若已知$(\log_a x)'=\frac{\log_a e}{x},\quad x\in(0,+\infty)$,试利用反函数求导法则求$(a^x)'$    
\end{example}

\begin{solution}
    因为$y=a^x,\quad (a>0,\; a\ne 1,\; -\infty<x<+\infty)$是$x=\log_a y\; (y>0)$的反函数,所以
\[\begin{split}
    (a^x)'&=\frac{1}{(\log_a y)'}=\frac{1}{\frac{\log_a e}{y}}=\frac{y}{\log_a e}\\
    &=\ln a\cdot (a^x)=a^x\ln a
\end{split}\]
\end{solution}

\begin{example}
求函数$f(x)=2x\arctan 2x-\ln\sqrt{1+4x^2}$的导数。
\end{example}

\begin{solution}
\[\begin{split}
    f'(x)&=\left[(2x)'\arctan 2x+2x\cdot \frac{1}{1+(2x)^2}\cdot (2x)'\right]\\
    &\qquad -\frac{1}{\sqrt{1+4x^2}}\cdot \frac{1}{2\sqrt{1+4x^2}}\cdot (1+4x^2)'\\
    &=2\arctan 2x+\frac{4x}{1+4x^2}-\frac{4x}{1+4x^2}\\
    &=2\arctan 2x
\end{split}\]
\end{solution}


\begin{example}
求曲线$y=\arcsin x$和$y=\arccos x$的交角。
\end{example}

\begin{solution}
设这两条曲线交于$T(x_0,y_0)$点,则
\[y_0=\arcsin x_0=\arccos x_0,\qquad 0<y_0<\frac{\pi}{2}\]
  由此得
\begin{equation}
    \sin(\arcsin x_0)=\sin(\arccos x_0)
\end{equation}  
即
\[x_0=\sqrt{1-x^2_0}\]
解得
\[x_0=\pm\frac{1}{\sqrt{2}}\]
因为$0<y_0<\frac{\pi}{2}$,所以$x_0>0$. 如此应取$x_0=\frac{1}{\sqrt{2}}$. 由于
\[(\arcsin x)'=\frac{1}{\sqrt{1-x^2}},\qquad (\arccos x)'=\frac{-1}{\sqrt{1-x^2}}\]
所以在点$x=\frac{1}{\sqrt{2}}$处,$y=\arcsin x$的切线的斜率为
\[\tan\alpha\Big|_{x=\tfrac{1}{\sqrt{2}}}=\frac{1}{\sqrt{1-\frac{1}{2}}}=\sqrt{2}\]
又$y=\arccos x$的切线的斜率是
\[\tan\beta\Big|_{x=\tfrac{1}{\sqrt{2}}}=\frac{-1}{\sqrt{1-\frac{1}{2}}}=-\sqrt{2}\]

设$\theta$是这两条曲线在交点$\left(\frac{1}{\sqrt{2}},\frac{\pi}{4}\right)$的夹角,也就是过$T$点的两条切线的交角,于是
\[\tan\theta=\frac{\tan\beta-\tan\alpha}{1+\tan\beta\tan\alpha}=\frac{-\sqrt{2}-\sqrt{2}}{1+\left(-\sqrt{2}\right)\sqrt{2}}=\frac{-2\sqrt{2}}{-1}=2\sqrt{2}\]
因此,这两条曲线的交角$\theta=\arctan 2\sqrt{2}$. 
\end{solution}

\begin{example}
一飞机在离地面2公里的高度,以每小时200公里的速度飞临某目标的上空,以便进行航空摄影,试求飞机飞至该目标正上方时摄影机转动的角速度(图2.9)。
\end{example}

\begin{solution}
我们把该目标取为坐标原点,设飞机和目标的水平距离为$x$, 显然它是时间$t$的函数$x=x(t)$. 现在要求出飞机在目标正上方时,$\frac{\dd \theta}{\dd t}$之值,
为此,先找出$\theta$和$x$的关系,从图2.9中可以看出
\[\tan\theta=\frac{2}{x},\quad  0<\theta\le \frac{2}{\pi}\]

$\therefore\quad \theta=\arctan \frac{2}{x}$
于是
\[\frac{\dd \theta}{\dd t}=\frac{-2}{x^2+4}\frac{\dd x}{\dd t}\]
但根据题意知道$\frac{\dd x}{\dd t}=-200$, 这里负号表示$x$在减小,故得
\[\frac{\dd \theta}{\dd t}=\frac{400}{x^2+4}\]
当飞机在目标正上方时,即$x=0$时,
$\frac{\dd \theta}{\dd t}=100\text{弧度/时}$,将弧度化为角度即为$\frac{\dd \theta}{\dd t}=\frac{5}{\pi}\text{度/秒}$
\end{solution}

\begin{figure}[htp]
    \centering
    \begin{minipage}[t]{0.48\textwidth}
    \centering
\begin{tikzpicture}[>=latex, scale=.9]
\draw[->](-1,0)--(5,0)node[right]{$x$};
\draw[->](0,-1)--(0,4.5)node[right]{$y$};
\node at (0,0)[below left]{$O$};
\draw[dashed](0,3.5)--(4,3.5);
\draw[very thick](4,3.5)--(0,0);
\draw[<->](4,3.5)--node[fill=white]{2}(4,0);
\draw[<->](1,0) arc (0:41.2:1)node[below right]{$\theta$};
\draw[<->](3,3.5) arc (180:180+41.2:1)node[above left]{$\theta$};
\draw[ultra thick, <-](4,3.5)--(4.6,3.5);
\end{tikzpicture}
    \caption{}
    \end{minipage}
    \begin{minipage}[t]{0.48\textwidth}
    \centering
    \begin{tikzpicture}[>=stealth, scale=1.5]
        \draw[very thick](-1,2.5)--(-.25,2.5)--(-.25,2)--(-1.5,0)--(1.5,0)--(.25,2)--(.25,2.5)--(1,2.5);
\draw[|<->|](1.75,0)--node[fill=white]{12cm}(1.75,2.5);
\draw[dashed](1,2.5)--(1.75,2.5);
\draw[|<->|](0,-.15)--node[below]{$R=5$cm}(1.5,-.15);
\draw(-1.75/2,1)--(1.75/2,1);
\draw[<->](0,0)--node[fill=white]{$y$}(0,1);
\draw[|<->|](0,.8)--node[fill=white]{$r$}(1.75/2,.8);
    \end{tikzpicture}
    \caption{}
    \end{minipage}
    \end{figure}

\begin{example}
有一底半径为5cm, 高为12cm的正圆锥形容器(图2.10), 以每秒2$\rm cm^3$的速度往容器中倒水,试求容器水位等于锥高一半时,水面上升速度。
\end{example}


\begin{solution}
    设在某时刻$t$, 容器水高为$y$, 则此时水的体积$V$为
 \[\begin{split}
     V&=\frac{\pi R^2h}{3}-\frac{\pi r^2}{3}(h-y)\\
     &=\frac{\pi}{3}\cdot 5^2\cdot 12-\frac{\pi r^2}{3}(12y)\\
     &=100\pi-\frac{\pi r^2}{3}(12-y)
 \end{split}\]
而
\[\frac{r}{R}=\frac{h-y}{h}\]
即
\[r=\frac{R}{h}(h-y)=\frac{5}{12}(12-y)\]
代入上式,得
\[\begin{split}
    V&=100\pi-\frac{\pi}{3}\cdot \frac{5^2}{12^2}(12-y)^3\\
    &=100\pi-\frac{25\pi}{432}(12-y)^3
\end{split}\]
又由题设知:$V=2t$,所以
\[2t=100\pi-\frac{25\pi}{432}(12-y)^3\]
即
\begin{equation}
    t=50\pi-\frac{25\pi}{864}(12-y)^3
\end{equation}
这就确定了水的高度与时间$t$的关系,水面上升速度就是
$y$关于时间$t$的变化率,也就是$\frac{\dd y}{\dd t}$。根据反函数求导法则,有
\[\frac{\dd y}{\dd t}=\frac{1}{\frac{\dd t}{\dd y}}\]
由(2.8)得
\[\frac{\dd t}{\dd y}=\frac{25\pi}{864}\cdot 3(12-y)^2(12-y)'=\frac{25\pi}{288}(12-y)^2\]

$\therefore\quad \frac{\dd y}{\dd t}=\frac{288}{25\pi(12-y)^2}$

当$y=\frac{1}{2}h=6$时,
\[\left.\frac{\dd y}{\dd t}\right|_{y=6}=\frac{288}{25\x 36\pi}\approx 0.102({\rm cm/s})\]

答:水面上升速度为每秒0.102厘米.
\end{solution}

\begin{ex}
\begin{enumerate}
    \item 求下列函数的微商:
\begin{multicols}{2}
\begin{enumerate}
  \item $\arcsin \frac{x}{4}$
\item  $\arccos x^{2}$
\item  $\arctan \frac{1}{x}$
\item  $\arcsin \sqrt{x}$
\item  $(\arcsin x)^{2}$
\item  $\arctan (\tan  x)$
\item $\arctan \frac{2 x}{1-x^{2}}$
\item  $\arctan (2 \tan  x)$
\item $x^{2} \arccot \frac{x}{a}$
\item $\ln \sqrt{1+4 x^{2}} \cdot \arccot \frac{1-x}{1+x}$
\item  $\frac{1-\arccos 2 x}{\arctan x}$
\item  ${\rm arccsc}\, x$
\end{enumerate}
\end{multicols}

\item 证明
   \[
    \frac{\dd}{\dd x}\left\{\arctan \left(\frac{\sin x}{\cos 2 x}\right)\right\}=\frac{3 \cos x-\cos 3 x}{2(1-\sin x \sin 3 x)}
   \]
    \item 求下列函数的导数:
\begin{enumerate}
\item $\arcsin \cdot \frac{6 x^{2}}{\sqrt{4+81 x^{8}}}$
\item $\arctan \frac{1}{1-3 x^{2}}-\arctan \frac{1}{1+3 x^{2}}$
\item $\arccot \frac{4 \sqrt{x}}{1-4 x}$
\end{enumerate}

\item 一圆锥形容器,深10尺,上顶圆半径为4尺;求
\begin{enumerate}
    \item 灌入水时,水的体积$V$对水面高度$h$的变化率;
    \item 体积$V$对容器截面圆半径$R$的变化率.
\end{enumerate}

\item 一圆锥形容器底面朝上放着,它的顶角为$2\arctan\frac{3}{4}$, 
今向里面倒进某种液体,试求
\begin{enumerate}
\item 当液面半径$r$为3, 半径增加的速度$\frac{\dd r}{\dd t}$为$\frac{1}{4}$时,
体积增加的速度$\frac{\dd V}{\dd t}$;
\item 当液面半径为6, 体积增加的速度为24时,半径增加的速度。
\end{enumerate}

\item 水从高为18厘米,底半径为6厘米的圆锥形漏斗中流入半
径为5厘米的圆柱形筒内.已知此漏斗中水深为12厘米时,漏斗中水面的下降速度为1厘米/分,求此时圆筒中水面的上升速度.
\item 底数$a$为什么值时,直线$y=x$才能与对数曲线$y=\log_a x$相切?在何处相切?
\item 求曲线$y=e^{-x}$上的一点,使过该点的切线与直线$y=-ex$平行,并写出该点的法线方程。
\item 求曲线$y=\frac{1}{2}(1+2x^2\pm \sqrt{1+4x^2})$上横坐标$x=u$的点
处的切线方程。这切线还与曲线交于何处?$y$的反函数为何?
\end{enumerate}    
\end{ex}

\subsection{隐函数的导数}

到现在为止,我们所讨论的函数都是可以由自变量明显地表示出来的函数,例如,方程
$y=x^3+1$
就定义了一个明显的函数$f$, 这里$f(x)=x^3+1$. 这种函数叫做显函数。但是并不是所有的函数都可以用这种明显的表达式来定义的,例如由下面的$x,y$的方程
\[x^3-x=y^3-y^2+24\]
就不容易解出用$x$表示$y$的解,或用$y$表示$x$的解。但是,却有可能存在一个函数$f$, 使得方程
\[x^3-x=f^3(x)-f^2(x)+24\]
对于函数$f$的定义域中的一切$x$值都成立,这样一个函数,如果存在的话,就叫做被这个方程定义的隐函数。最简的情形是二元一次方程:
\[Ax+By+C=0\]
当$B\ne 0$时,就定义一个隐函数$f:\; \mathbb{R}\mapsto \mathbb{R}$, 这里$f(x)=-\frac{A}{B}x-\frac{C}{B}$. 

应该注意:
\begin{enumerate}
\item 方程$F(x,y)=0$的解有可能不存在,例如方程$x^2+y^2+1=0$没有实数解,因此,由它不能定义一个隐函数。
\item 由方程$F(x,y)=0$可能定义几个函数.例如,由方程
$x^2+y^2=16$
就定义了两个函数:
\[f_1 (x) =\sqrt{16-x^2},\qquad x\in [-4, 4] \]
和
\[f_2 (x) =-\sqrt{16-x^2},\qquad x\in  [-4, 4] \]
\end{enumerate}

\begin{blk}
    {定义} 方程
    \begin{equation}
 F (x,y) =0       
    \end{equation}
给出了变元$x,y$之间的一个关系。如果存在一个函数$y=
f(x)$, 或$x=g(y)$使得对于函数的定义域中的一切值,等式$F(x,f(x))=0$或$F(g(x),y)=0$恒成立,我们就说方程(2.8)定义了一个\textbf{隐函数}:
\[y=f(x)\qquad \text{或}\qquad x=g(y)\]
\end{blk}

要对方程(2.8)给出的隐函数求导数,通常的办法是直接对这个方程求导数,而不是先解出显函数,再来求导数。


\begin{example}
设$xy+x^2-1=0$,求$y'_x$.
\end{example}

\begin{solution}
先把$y$看成$x$的函数,即$y=f(x)$, 然后,在等式
$xf (x) +x^2-1=0$
的两边对$x$求导数,得到
\[\begin{split}
    (x\cdot f(x))'+(x^2)'&=0\\
xf'(x)+f(x)+2x&=0
\end{split}\]

$\therefore\quad f (x) =-\frac{2x+f (x)}{x}=-\frac{2x+\frac{1-x^2}{x}}{x}=-\frac{1+x^2}{x^2}$    
\end{solution}

\begin{example}
    方程$x^2+y^2=16$定义了两个可微函数:
$y=\sqrt{16-x^2}$和$y=-\sqrt{16-x^2},\; x\in [-4, 4]$. 求每一个函数的导数。
\end{example}


\begin{solution}
    我们用$f(x)$代表上面两个函数中的任何一个,则
\[x^2+f^2 (x) =16,\qquad  x\in  [-4, 4] \]
对上面等式的两边微分,得到
\[\begin{split}
    2x+2f (x) \cdot f' (x) &=0\\
    f'(x)&=-\frac{x}{f(x)}
\end{split} \]
即
\[\begin{split}
    \left(\sqrt{16-x^2}\right)'&=-\frac{x}{\sqrt{16-x^2}}\\
    \left(-\sqrt{16-x^2}\right)'&=\frac{x}{\sqrt{16-x^2}},\qquad x\in[-4,4]
\end{split}\]
\end{solution}

\begin{rmk}
应用这个方法求隐函数的导数时,我们作了两点假设:
\begin{enumerate}
    \item 存在一个由方程$F(x,y)=0$定义的函数;
    \item 这样定义的函数是可微的.
\end{enumerate} 
至于在什么条件下,这种假设是合理的问题,已经超出了本书的范围,故不在此讨论。
\end{rmk}









\begin{example}
    
\end{example}


\begin{solution}
    
\end{solution}


\begin{example}
    
\end{example}


\begin{solution}
    
\end{solution}




\begin{example}
    
\end{example}


\begin{solution}
    
\end{solution}





\begin{example}
    
\end{example}


\begin{solution}
    
\end{solution}






\begin{example}
    
\end{example}


\begin{solution}
    
\end{solution}



\begin{example}
    
\end{example}


\begin{solution}
    
\end{solution}


\begin{example}
    
\end{example}


\begin{solution}
    
\end{solution}








\begin{example}
    
\end{example}


\begin{solution}
    
\end{solution}




\begin{example}
    
\end{example}


\begin{solution}
    
\end{solution}




\begin{example}
    
\end{example}


\begin{solution}
    
\end{solution}



\begin{example}
    
\end{example}


\begin{solution}
    
\end{solution}







\begin{example}
    
\end{example}


\begin{solution}
    
\end{solution}






\begin{example}
    
\end{example}


\begin{solution}
    
\end{solution}




















































\chapter{求函数从$a$到$b$的和与积分}
函数有两个基本性质——变率与和,在前一章,我们研究了函数的瞬时变率的概念,即
\[f'(x_0) =\lim_{h\to 0} \frac{f (x_0+h) -f (x_0)}{h}\]
以及它的应用,在本章我们将研究对于一个给定函数$f(x)$“求从$a$到$b$的和”这个概念.在给出定义之前,我们先举几个实例来看一看.

\paragraph{速率与距离} 
一列行驶中的火车,它的行进速率$v$是时间$t$的函数,即
$v=f(t)$, 如图3.1, 我们可以从速率表读得当时的速率,很自然地,我们想知道火车在从$t=a$到$t=b$这一段时间间隔内一共走了多少路程.从函数的观点看,所谓求在时间$[a,b]$内所走过的距离就是求速率函数$v=f(t)$由$t=a$到$t=b$的和.

\begin{figure}[htp]
    \centering
\begin{tikzpicture}[>=latex]
\draw[->](-.5,0)--(6,0)node[right]{$t$};
\draw[->](0,-.5)--(0,3)node[right]{$v=f(t)$};
\draw[very thick](0,0)--(1,1.5)--(1.6,1.5);
\draw[dashed](1,1.5)--(1,0)node[below]{$a$};
\node at (0,0) {};
\draw plot[smooth] coordinates{
(1.6,1.5)(2.5,2.5)(3.5,2.2)(4,1.2)(4.3,.8)(4.8,.6)(5,0)
};
\draw[dashed](3.5,2.2)--(3.5,0)node[below]{$b$};

\end{tikzpicture}
    \caption{}
\end{figure}

让我们用$D(f,[a,b])$
表示所走过的距离,这个记号强调$D$依赖于$f$和区间$[a,b]$.

\paragraph{变力所作的功}

假定某物体在一个平行于$OX$轴的力$P$的作 用下沿直线
$OX$运动,力$P$的方向与物体运动的方向一致,并且力的大小随离开$O$点的距离而改变,即变力$P$是所在点的横坐标$x$的函数$P=P(x)$, 如图3.2.假定物体在这个变力$P$作用之下,从直线$OX$的一点$a$移到另一点$P$, 那么力$P$所作的功就是变力函数$P(x)$由$x=a$到$x=b$的和,我们用
$W (P, [a,b])$
表示力$P$所作的功,它表明$W$依赖于$P(x)$和$[a,b]$.

\begin{figure}[htp]
    \centering
    \begin{minipage}[t]{0.48\textwidth}
    \centering
    \begin{tikzpicture}[>=latex, scale=1]
\draw[->](-.5,0)--(4,0)node[right]{$x$};
\draw[->](0,-.5)--(0,3)node[right]{$P(x)$};
\draw[domain=0:3.5, samples=100, very thick]plot(\x, {3/(\x+1)-.5});
\draw(1,0)node[below]{$a$}--(1,1);
\draw(3,0)node[below]{$b$}--(3,.25);
\node at (0,0)[below left]{$O$};
    \end{tikzpicture}
    \caption{}
    \end{minipage}
    \begin{minipage}[t]{0.48\textwidth}
    \centering
    \begin{tikzpicture}[>=latex, scale=1]
\draw[->](-.5,0)--(4,0)node[right]{$x$};
\draw[->](0,-.5)--(0,3)node[right]{$y$};
\draw[domain=0:3.5, samples=100, very thick]plot(\x, {.3*(\x-2.2)^2+.5})node[above right]{$y=f(x)$};
\node at (0,0)[below left]{$O$};
\draw(.5,0)node[below]{$a$}--(.5,1.367)node[above right]{$M_0$};
\draw(3,0)node[below]{$b$}--(3,.692)node[above]{$M$};
\fill[pattern=north east lines, domain=.5:3, samples=100, very thick]plot(\x, {.3*(\x-2.2)^2+.5})--(3,.692)--(3,0)--(.5,0)--(.5,1.367);


    \end{tikzpicture}
    \caption{}
    \end{minipage}
\end{figure}

\paragraph{曲边梯形的面积}

令$y=f(x)$是函数$f$的图象,表示一条曲线,如图3.3. 我们要求曲线上的一段弧$M_0M$ 与其两端的纵坐标线及$x$轴上的线段$[a,b]$所围成的图形的面积.这样的图形(它有三条边是直线,其中两条互相平行,第三条与前两条
互相垂直,而第四条边是曲线)叫做曲边梯形.显然曲边梯形的面积$A$依赖于$y=f(x)$和$x$轴上的线段$[a,b]$, 记这一面积为$A (f, [a,b])$.

在上一章,我们利用函数$f(x)$和它的图象之间的对应关系,就可以把函数的变率和它的图象的切线的斜率相对应地一并分析讨论,这样,一方面可以把函数的变率这种“数量”的概念用切线来形象化,便于想象;而另一方面又可以把“切线”这种几何概念数量化,便于计算,在这一章,我们也要利用函数$f(x)$和它的图象之间的对应关系,把函数$f(x)$由$x=a$到$x=b$的“和”与曲线$y=f(x)$的曲边梯形的“面积”相对应地一并分析讨论,并且说明函数的“求从$a$到$b$的和”恰好对应于求曲边梯形的面积.这也就是为什么把函数“求从$a$到$b$的和”这种基本运算叫做积分的道理.

\section{{``}和{''}与{``}面积{''}}

对于任意曲线围成的图形,我们还没有规定它的“面积”的意义,和它密切相关的“函数从$a$到$b$的和”的概念至今也没有明确的解析的定义.在这一节我们要把这两个概念由“直观的定性理解”推进到“数量化的定量定义”.唯有确立了它们的“解析的定义”,它们才真正地成为能算好用的量.

\subsection{{``}和{''}与{``}面积{''}的基本性质}

现在让我们先从“函数的和”与“曲线形的面积”的直观内涵来分析一下它们分别所应有的基本性质.

\subsubsection{{``}曲线形的面积{''}的基本性质}

从面积的直观内涵容易看出下列两点:
\begin{enumerate}
\item (单调性)设区域$R_1$包含在$R_2$之内,即$R_1\subseteq R_2$, 则:
$R_1$的面积$\le R_2$的面积.(图3.4)
\item (可加性)设区域$R$可以用一条曲线分割成两块区
域$R_1+R_2$, 则有:
$R$的面积$=R_1$的面积$+R_2$的面积.(图3.5)
\end{enumerate}

\begin{figure}[htp]
    \centering
    \begin{minipage}[t]{0.48\textwidth}
    \centering
    \begin{tikzpicture}[>=latex, scale=1]
\draw[very thick](0,0) ellipse [x radius=2, y radius =1];
\draw[very thick](-.5,0) ellipse [x radius=1, y radius =.6];
\node at (0,1.3){$R$};
\node at (-.5,0){$R_1$};
\node at (1.3,0){$R_2$};
    \end{tikzpicture}
    \caption{}
    \end{minipage}
    \begin{minipage}[t]{0.48\textwidth}
    \centering
    \begin{tikzpicture}[>=latex, scale=1]
\draw[very thick](0,0) ellipse [x radius=2, y radius =1];
\node at (0,1.3){$R$};
\node at (-1,0){$R_1$};
\node at (1,0){$R_2$};
\draw[very thick](0,1) to [bend left](.3,.6)to [bend left](0,-.3)to [bend right](0,-1);
    \end{tikzpicture}
    \caption{}
    \end{minipage}
\end{figure}

\subsubsection{{``}函数的和{''}的基本性质}

同样地,从“函数从$a$到$b$的和”的直观内涵容易看出下列两个基本性质,即

\begin{blk}{性质1:单调性}
设函数$f(t)$和$g(t)$在$a\le t\le b$上,$f(t)\le g(t)$恒成立,那么:

“$f(t)$由$t=a$到$t=b$的和”$\le $“$g(t)$由$t=a$到$t=b$
的和”.
\end{blk}

例如,两部车子在$t=a$到$t=b$的时间内,甲车的速率$f(t)\le $乙车的速率$g(t)$, 则甲车在上述时间内所经过的里程$\le $乙车在上述时间内所经过的里程.

\begin{blk}{性质2:可加性}
设$a<b<c$,那么,有:

“$f(t)$由$t=a$到$t=b$的和”$+$“$f(t)$由$t=b$到$t=c$的和”$=$“$f(t)$由$t=a$到$t=c$的和”.
\end{blk}

上述基本性质是一目了然的,让我们先用来说明一些简单的基本事实.

\begin{example}
    设函数$y=f(t)=k$(常数),则由和的直观内涵,显然应有:
\[\text{常数函数从$a$到$b$的和}=k(b-a)\]

例如,以等速率每小时$k$公里行驶的火车从$t=a$小时到$t=b$小时内所走的总里程应该是$k(b-a)$公里.

现在让我们利用函数的图象来观察上述数值$k(b-a)$的几何意义.

\begin{figure}[htp]
    \centering
    \begin{minipage}[t]{0.48\textwidth}
    \centering
    \begin{tikzpicture}[>=latex, scale=.9]
\draw[->](-.5,0)--(5,0)node[right]{$t$};
\draw[->](0,-.5)--(0,3)node[right]{$y$};
\fill[pattern=north east lines](1,0) rectangle (4,1.5);
\draw[dashed](0,1.5)--(1,1.5)--(1,0)node[below]{$t=a$};
\draw[dashed](4,1.5)--(4,0)node[below]{$t=b$};
\node at (0,0)[below left]{$O$};
\draw[very thick](1,1.5)--(4,1.5);
\node at (0,.75)[left]{$k>0$};
\node at (2.5,.75){\Large $+$};

    \end{tikzpicture}
    \caption{}
    \end{minipage}
    \begin{minipage}[t]{0.48\textwidth}
    \centering
    \begin{tikzpicture}[>=latex, scale=.9]
\draw[->](-.5,0)--(5,0)node[right]{$t$};
\draw[->](0,-2)--(0,1.5)node[right]{$y$};
\fill[pattern=north east lines](1,0) rectangle (4,-1.5);
\draw[dashed](0,-1.5)--(1,-1.5)--(1,0)node[above]{$t=a$};
\draw[dashed](4,-1.5)--(4,0)node[above]{$t=b$};
\node at (0,0)[below left]{$O$};
\draw[very thick](1,-1.5)--(4,-1.5);
\node at (0,-.9)[left]{$k<0$};
\node at (2.5,-.75){\Large $-$};
    \end{tikzpicture}
    \caption{}
    \end{minipage}
\end{figure}

\begin{enumerate}
    \item 当$k>0$时,函数$f(t)=k$的图象在$t$轴上方,$k(b-a)$对应于一个矩形的面积,如图3.6中阴影部分的面积.
    \item 当$k<0$时,函数$f(t)=k$的图象在$t$轴下方,$k(b-a)$对应于一个矩形的面积的负值,如图3.7中阴影部分的面积的负值.
\end{enumerate}

假如我们把$t$轴之上的面积定义为正的,而把$t$轴之下的面积定义为负的,则当$y=k$(常数)时,常数函数从$a$到$b$的和$=k(b-a)=$上述的有号面积.
\end{example}

\begin{example}
设$y=f(t)$是一个阶梯函数,即分段地是常数函数:
\[0=t_0<t_1<t_2<\cdots<t_{i-1}<t_i<\cdots <t_n=b\]
当$t_{i-1}\le t<t_i$时,$f(t)=k_i,\; i=1,2,\ldots,n$.由性质2,我们可以分段地用常数函数求和,就可以得到阶梯函数$f(t)$从$a$到$b$的和:
\[k_1 (t_1-t_0) +k_2 (t_2-t_1) +\cdots +k_n (t_n-t_{n-1})\]

作出阶梯函数的图象(图3.8),上述函数和就等于下列逐段由矩形并起来的区域的“有号面积”.
\begin{figure}[htp]
    \centering
\begin{tikzpicture}[>=latex, yscale=.7]
\draw[->](-5,0)--(7,0)node[right]{$t$};
\draw[->](0,-2)--(0,4)node[right]{$y=f(t)$};
\node at (0,0)[above right]{$O$};

\foreach \x/\xtext in {-3/t_1, -1/t_2, 1.5/t_3, 2.5/t_4, 3.5/t_5, 4.3/t_6, 5.5/t_{n-1},-4.5/{t_0=a},6.5/{t_n=b}}
{
    \node at (\x,0)[below]{$\xtext$};
}

\draw[pattern=north east lines, thick](-4.5,0) rectangle(-3,1.5);
\draw[pattern=north east lines, thick](-3,1) rectangle(-1,0);
\draw[pattern=north east lines, thick](-1,0) rectangle(1.5,-1.5);
\draw[pattern=north east lines, thick](1.5,0) rectangle(2.5,1);
\draw[pattern=north east lines, thick](2.5,0) rectangle(3.5,2);
\draw[pattern=north east lines, thick](3.5,0) rectangle(4.3,-1);
\draw[pattern=north east lines, thick](5.5,0) rectangle(6.5,1);
\foreach \x in {-3.75, -2, 2, 3, 6}
{
    \node at (\x,.5){$+$};
}
\foreach \x in {.75, 3.9}
{
    \node at (\x,-.5){$-$};
}
\node at (4.8,.5){…};\node at (4.8,-.5){…};

\end{tikzpicture}
    \caption{}
\end{figure}
\end{example}

\subsection{逼近法求和}

对于一个比较一般的函数,例如$y=kt+c$, $y=t^2$, $y=\sin t$等等,我们必须给函数的和下一个适当的定义,同时要提供计算这个和的方法,这里我们将要再一次地用逼近法的观点去解答上述问题,回顾上一章讨论变率时,我们的基本想法是用折线函数去逼近一般的平滑函数,所以折线函数是讨论函数变率的简单好用的基本函数,因为它们的变率十分简单,分段地是个常数而且无限逼近函数在某一点的变率,由上一段的讨论;我们知道阶梯函数的“从$a$到$b$的和”是十分简明的,它们在“函数求从$a$到$b$的和”这方面是不是也扮演着
这种既简单又基本的角色呢?这就得看一看能否用阶梯函数的从$a$到$b$的和去无限逼近一般性的“函数从$a$到$b$的和”了.

现在让我们使用几个简单的实例来试试看.

\begin{example}
假设物体以初速度$u$, 加速度$a>0$在直线上运动,于是物体在任何时刻$t$的速度是$v=f (t) =u+at$, 求物体从$t=0$到$t=T$时所经过的距离,即求速度函数$f(t)$在时间间隔$[0,T]$上的和.
\end{example}

\begin{analyze}
    一个自然的作法是用两个阶梯函数$g_n(t)$和$G_n(t)$把上述函数$f(t)=u+at$夹逼在中间,即
    \[g_n(t)\le f(t)\le G_n(t)\]
对于任何$0\le t\le T$都成立,那么由性质1,$g_n(t)$从0到$t$的和$\le f(t)$从$0$到$T$的和$\le G_n(t)$从0到$T$的和.

如果能使$g_n(t)$从0到$T$的和与$G_n(t)$从0到$T$的和,当$n\to\infty$时的极限相同,则由上述夹逼关系就可以看出$f(t)$从0到$T$的和必须等于这个共同极限.
\end{analyze}

\begin{figure}[htp]
    \centering
\begin{tikzpicture}[>=latex]
\draw[->](-.5,1)--(8,1)node[right]{$t$};
\draw[|<->|](-.2,1)--node[fill=white]{$u$}(-.2,2);
\draw[->](0,0)--(0,6)node[right]{$v=f(t)=u+at,\; (0\le t\le T)$};
\node at (0,1)[below left]{$O$};
\draw[ultra thick](0,2)node[above left]{$A$}--(7,5.5)node[above]{$B$};
\draw[<->](7.3,1)node[below]{$C$}--node[fill=white]{$u$}(7.3,2);
\draw[|<->|](7.3,5.5)--node[fill=white]{$aT$}(7.3,2);
\foreach \x in {1,2,3,...,6,7}
{
    \draw[dashed](\x,1)node[below]{$t_{\x}$}--(\x,2+\x*.5);
    \draw[very thick](\x-1, 1.5+\x*0.5)--(\x, 1.5+ \x*0.5)--(\x, 2+\x*0.5);
    \draw[thick](\x-1, 1.5+\x*0.5)--(\x-1, 2+\x*0.5)--(\x, 2+ \x*0.5);
}
\draw[dashed](0,2)--(7,2);
\draw(7,0)--(7,1);
\draw[<->](0,.2)--node[fill=white]{$T$}(7,.2);
\end{tikzpicture}
    \caption{}\label{fig:chapter3_sec1_line_segmentation}
\end{figure}

下面就是把上述想法付诸实践的具体做法之一.

\begin{solution}
\begin{enumerate}
    \item 把时间间隔$n$等分,取分点$t_i=\frac{i}{n}T,\; i=0,1, 2,\ldots,n$, 于是在每个分点$t_i=\frac{i}{n}T$处的速度分别是
\[f (t_0) =f (0) =u,\;  f(t_1) =u+at_1,\; f (t_3) =u+at_2,\ldots,f (t_n) =f (T) =u+aT\]
\item 用阶梯函数近似代替$v=f(t)=u+at$. 

假设物体在时间的每个小区间$[t_i,t_{i+1}],\; i=0,1, 2,\ldots,(n-1)$内,以小区间起点处的速度作匀速运动,我们得到一个速度的阶梯函数
\[g_n (t)=f(t_i)=u+at_i,\quad t_i\le t<t_{i+1},\quad i=0,1,2,\ldots,(n-1)\]
因为$f(t)=u+at$是严格递增的,所以$g_n(t)$有以下性质:
\[g_n (t)=f(t_i)\le f(t),\quad t_i\le t<t_{i+1},\quad i=0,1,2,\ldots,(n-1)\]
从而,对于任何$0\le t\le T$,有$g_n(t)\le f(t)$.

假设物体在时间的每个小区间$[t,t+1],\; i=0,1, 2,\ldots,(n-1)$内,以小区间终点处的速度作匀速运动,我们得到另一个速度的阶梯函数
\[G_n(t)=f(t_{i+1})=u+at_{i+1},\quad t_i< t\le t_{i+1},\quad i=0,1,2,\ldots,(n-1)\]

由于$f(t)=u+at$是递增的,所以$G(t)$有以下性质
\[G_n (t)=f(t_{i+1})\ge f(t),\quad t_i< t\le t_{i+1},\quad i=0,1,2,\ldots,(n-1)\]
从而对于任何$0\le t\le T$,有
\[G_n(t)\ge f(t)\]
这样,我们就得到了$v=f(t)$的夹逼阶梯函数:
\[ g_n(t) \le f (t) \le G_n (t),\quad 0\le t\le T\]

(在图\ref{fig:chapter3_sec1_line_segmentation}中,绘出了$f(t)=u+at$的图象和当$n=7$时阶梯函数$g_7(t)$和$G_7(t)$的图象.)

\item 求阶梯函数的总和

应用基本性质2, 我们得到$g_n(t)$从0到$T$的和
\[f(t_0)(t_1-t_0)+f(t_1)(t_2-t_1)+\cdots+f (t_{n-1}) (t_n-t_{n-1})\]

$\because\quad t_1-t_0=t_2-t_1=\cdots=t_n-t_{n-1}=\frac{T}{n}$

因此:
\begin{align*}
\text{$g_n(t)$从0到$T$的和}&=\left[f(t_0)+f(t_1)+\cdots+f(t_{n-1})\right]\cdot \frac{T}{n}\\
&=\left[u+(u+at_1)+(u+at_2)+\cdots +(u+at_{n-1})\right]\cdot \frac{T}{n}\\
&=\left[nu+a(t_1+t_2+\cdots+t_{n-1})\right]\cdot \frac{T}{n}\\
&=\left[nu+a\left(\frac{T}{n}+\frac{2T}{n}+\cdots+\frac{(n-1)T}{n}\right)\right]\cdot \frac{T}{n}\\
&=uT+a\cdot [1+2+\cdots+(n-1)]\left(\frac{T}{n}\right)^2\\
&=uT+a\cdot \frac{n(n-1)}{2}\cdot \frac{T^2}{n^2}\\
&=uT+\frac{1}{2}aT^2\left(1-\frac{1}{n}\right)
\end{align*}

\begin{align*}
    \text{$G_n(t)$从0到$T$的和}&=f(t_1)(t_1-t_0)+f(t_2)(t_2-t_1)+\cdots+f(t_n)(t_n-t_{n-1})  \\
   &= \left[f(t_1)+f(t_2)+\cdots+f(t_{n})\right]\cdot \frac{T}{n}\\
    &=\left[(u+at_1)+(u+at_2)+\cdots +(u+at_{n})\right]\cdot \frac{T}{n}\\
    &=\left[nu+a\left(\frac{T}{n}+\frac{2T}{n}+\cdots+\frac{nT}{n}\right)\right]\cdot \frac{T}{n}\\
    &=uT+a\cdot (1+2+\cdots+n)\cdot \left(\frac{T}{n}\right)^2\\
    &=uT+\frac{1}{2}aT^2\left(1+\frac{1}{n}\right)
    \end{align*}
综合上述计算和基本性质1, 即得:$g_n(t)$从0到$T$的和$\le f(t)$从0到$T$的和$\le G_n(t)$从0到$T$的
和,即
\[uT+\frac{1}{2}aT^2\left(1-\frac{1}{n}\right)\le f(t)
\text{从0到$T$的和}\le uT+\frac{1}{2}aT^2\left(1+\frac{1}{n}\right)\]

\item 求阶梯函数和式的极限.

因为
\begin{align*}
    \lim_{n\to\infty}\left\{uT+\frac{1}{2}aT^2\left(1-\frac{1}{n}\right)\right\}&=\lim_{n\to\infty}\left\{uT+\frac{1}{2}aT^2\left(1+\frac{1}{n}\right)\right\}\\
    &=uT+\frac{1}{2}aT^2
\end{align*}
所以,这个共同的极限值$uT+\frac{1}{2}aT^2$就是物体在变速$v=f(t)=u+at$运动下,从$t=0$到$t=T$所走的距离,也就是函数$v=f(t)=u+at$从$t=0$到$t=T$的和.
\end{enumerate}

从几何上看,上述极限值$uT+\frac{1}{2}aT^2=\frac{u+(u+aT)T}{2}$就是在图\ref{fig:chapter3_sec1_line_segmentation}中的梯形$OABC$的面积.
\end{solution}    

\begin{example}
    设函数$f(x)=x^2$, $a=0$, $b>0$, 求$f(x)$由0到$b$的和,相应地,求抛物线$y=x^2$在线段$[0,b]$上所盖的曲边三角形$OBC$的面积(图\ref{fig:chapter3_sec1_y=x^2}).
\end{example}

\begin{figure}[htp]
    \centering
\begin{tikzpicture}[>=latex, scale=.5]
\draw[->](-5,0)--(5,0)node[right]{$x$};
\draw[->](0,-1.5)--(0,11)node[right]{$y$};
\draw[domain=-3.25:3.25, samples=100, very thick]plot(\x, {\x*\x});
\node at (0,0)[below left]{$O$};
\foreach \x in {.5,1,...,3}
{
    \draw[thick](\x-.5,0) rectangle (\x, \x*\x);
}
\foreach \x in {.5,1,...,2.5}
{
    \draw(\x,\x*\x) -- (\x+.5,\x*\x);
}
\node at (3,9)[right]{$C$};
\node at (3,0)[below]{$B$};
\node at (-2.5,6)[left]{$y=x^2$};
\end{tikzpicture}
    \caption{}\label{fig:chapter3_sec1_y=x^2}
\end{figure}

\begin{solution}
\begin{enumerate}
    \item 把线段$[0,b]$ $n$等分,等分点的坐标是
   $ x_i=\frac{b}{n}i\;  (i=0, 1, 2,\ldots,n)$ 
   \item 作阶梯函数$g_n(x)$和
    $G_n (x)$.

    根据$f(x)=x^2$在$x>0$时递增,定义
\begin{align*}
g_n(x)&=f(x_i)=x^2_i=\left(\frac{b}{n}\right)^2i^2,\quad x_i\le x<x_{i+1}\\
G_n(x)&=f(x_{i+1})=x^2_{i+1}=\left(\frac{b}{n}\right)^2(i+1)^2,\quad x_i<x\le x_{i+1}\\
\end{align*}
其中,$i=0,1,2,\ldots,(n-1)$.于是,$g_n(x)\le f(x)\le G_n(x),\quad 0\le x\le b$

\item 求阶梯函数的和
\begin{align*}
s_n&=g_n(x)\text{从0到$b$的和}\\
&=f\left(x_{0}\right) \cdot \frac{b}{n}+f\left(x_{1}\right) \cdot \frac{b}{n}+\cdots+f\left(x_{n-1}\right) \cdot \frac{b}{n}\\
&=\left\{f\left(x_{0}\right)+f\left(x_{1}\right)+\cdots+f\left(x_{n-1}\right)\right\} \cdot \frac{b}{n} \\
&=\left\{0^{2}+1^{2}+2^{2}+\cdots+(n-1)^{2}\right\} \cdot\left(\frac{b}{n}\right)^3\\
&=\frac{1}{6}n(n-1)(2 n-1) \cdot\left(\frac{b}{n}\right)^{3} =\frac{b^3}{6}\left(1-\frac{1}{n}\right)\left(2-\frac{1}{n}\right) 
\end{align*}
\begin{align*}
   S_{n}= G_{n}(x) \text { 从0到$b$的和}
   &= \left\{f\left(x_{1}\right)+f\left(x_{2}\right)+\cdots+f\left(x_{n}\right)\right\} \cdot \frac{b}{n} \\
   &=\left\{1^2+2^{2}+\cdots+n^{2}\right\} \cdot\left(\frac{b}{n}\right)^{3} \\
   &=\frac{1}{6}n(n+1)(2 n+1) \cdot\left(\frac{b}{n}\right)^{3} \\
   &=\frac{b^3}{6}\left(1+\frac{1}{n}\right)\left(2+\frac{1}{n}\right) 
\end{align*}

综合上述计算和性质1,有:
\begin{align*}
\frac{b^3}{6}\left(1-\frac{1}{n}\right)\left(2-\frac{1}{n}\right)&=s_n\le \begin{bmatrix}
\text{函数$f(x)=x^2$从0到$b$的和,}\\
\text{相应的曲边三角形$OBC$的面积}
\end{bmatrix}\\
&\le S_n=\frac{b^3}{6}\left(1+\frac{1}{n}\right)\left(2+\frac{1}{n}\right)
\end{align*}

\item 令$n\to\infty$,求阶梯函数和式的极限.因为
\begin{align*}
    \frac{b^3}{6} \left(1-\frac{1}{n}\right)\left(2-\frac{1}{n}\right)\to \frac{b^3}{3}\qquad (n\to\infty)\\
    \frac{b^3}{6} \left(1+\frac{1}{n}\right)\left(2+\frac{1}{n}\right)\to \frac{b^3}{3}\qquad (n\to\infty)\\
\end{align*}
所以$\frac{1}{3}b^3$是能被$s_n$、$S_n$左、右夹逼的唯一实数,而所求的“$f(x)=x^2$从0到$b$的和”以及“曲边三角形$OBC$的面积,”这两个值都是夹逼在$s_n$和$S_n$中间的实数,故它们都等于$\frac{1}{3}b^3$.
\end{enumerate}
\end{solution}

\begin{blk}
    {推论} $f(x)=x^2$从$a$到$b$的和$=\frac{1}{3}b^3-\frac{1}{3}a^3\quad  (b> a> 0)$.
\end{blk}


\begin{example}
设$y=f(x)$是一个定义在$[a,b]$上的递增连续函数,试说明它从$a$到$b$的和是可以确定的.
\end{example}

\begin{figure}[htp]
    \centering
\begin{tikzpicture}[>=latex]
\draw[->](-.5,0)--(5,0)node[right]{$x$};
\draw[->](0,-.5)--(0,4)node[right]{$y$};
\draw[domain=.5:4, samples=50, very thick]plot(\x, {.2*\x*\x+.5});
\draw[dashed](1,.7)--(1,0)node[below]{$a$};
\draw[dashed](4,3.7)--(4,0)node[below]{$b$};
\draw[dashed](2,1.3)--(2,0);
\draw[dashed](3,2.3)node[left]{$y=f(x)$}--(3,0);
\node at (0,0)[below left]{$O$};
\end{tikzpicture}
    \caption{}\label{fig:chapter3_sec1_monotone f}
\end{figure}

\begin{solution}
\begin{enumerate}
    \item     取$[a,b]$之间的$n$等分点,将它分成$n$段,即有
\[x_0=a<x_1<x_2<\cdots<x_n=b\]
它的第$i$段是$[x_{i-1},x_i]$,且$x_i-x_{i-1}=\frac{b-a}{n}$,$x_i=a+\frac{i}{n}(b-a)$

\item 定义$f(x)$的上、下夹逼阶梯函数如下:
\[
\begin{split}
    g_n(x)&=f(x_{i-1}),\qquad x_{i-1}\le x\le x_i \\
    G_n(x)&=f(x_i),\qquad x_{i-1}\le x\le x_i 
\end{split} \quad (i=1,2,3,\ldots,n)
\]
由$f(x)$的递增性,对于任何一个$x\in[a,b]$,有
\[g_1(x)\le g_2(x)\le\cdots \le g_n(x)\le\cdots \le f(x)\le \cdots\le G_n(x)\le\cdots\le G_2(x)\le G_1(x)\]
简写成
\[g_n(x)\le f(x)\le G_n(x),\qquad a\le x\le b\]
\item 求相应的阶梯函数从$a$到$b$的和
\begin{equation}
\begin{split}
    s_n=g_n(x)\text{从$a$到$b$的和}
    &=\frac{b-a}{n}\left[f(x_0)+f(x_1)+\cdots+f(x_{n-1})\right]\\
    &=\frac{b-a}{n}\sum^n_{i=1}f(x_{i-1})
\end{split}
\end{equation}
\begin{equation}
    \begin{split}
    S_n=G_n(x)\text{从$a$到$b$的和}
    &=\frac{b-a}{n}\left[f(x_1)+f(x_2)+\cdots+f(x_{n})\right]\\
    &=\frac{b-a}{n}\sum^n_{i=1}f(x_{i})   
    \end{split}
    \end{equation}
\end{enumerate}
由性质1, 有:
\[s_1<s_2<\cdots<s_n<\cdots<f(x)\text{从$a$到$b$的和}<\cdots<S_n<\cdots<S_2<S_1\]
简写成
\begin{equation}
    s_n<f(x)\text{从$a$到$b$的和}<S_n
\end{equation}
在例3.3, 例3.4中,$f(x)$有明确的解析式,我们可以用求和公式直接求出$s_n$和$S_n$的表达式,从而可以得出它们的共同极限值,在这里,我们只知道$f(x)$的递增性,当然无法将和式$s_n$和$S_n$进一步化简,但是我们可以说明,当$n\to\infty$时,$S_n-s_n\to 0$.

事实上,
\begin{align*}
    S_n-s_n&=\frac{b-a}{n}\sum^n_{i=1}f(x_{i})   -\frac{b-a}{n}\sum^n_{i=1}f(x_{i-1}) \\
    &=\frac{b-a}{n}\sum^n_{i=1}\left[f(x_{i})-f(x_{i-1})\right]   \\
    &=\frac{b-a}{n}\left[f(x_{n})-f(x_{0})\right]  =\frac{b-a}{n}\left[f(b)-f(a)\right]   
\end{align*}
因此,当$n\to\infty$时,$S_n-s_n\to 0$.

又$f(x)$从$a$到$b$的和是介于$s_n$和$S_n$中间的唯一实数,
此
\[\lim_{n\to\infty} s_n=\lim_{n\to\infty} S_n=f(x)\text{ 从$a$到$b$的和}\]

以上简单明了的分析,说明了下面两点互相关联的事实:
\begin{enumerate}
    \item 当函数$f(x)$在$a\le x\le b$上递增时,则它从$a$到$b$的和可以用上述两个夹逼阶梯函数从$a$到$b$的和去无限逼近.
    \item 由于$\Lim_{n\to\infty}s_n$与$\Lim_{n\to\infty}S_n$存在,且$\Lim_{n\to\infty}s_n=\Lim_{n\to\infty}S_n$, 我们可以用$f(x)$的上、下夹逼阶梯函数序列,即:$g_1 (x)\le g_2(x)\le \cdots\le g_n(x)\le \cdots\le f(x)\le \cdots\le G_1(x)\le \cdots\le G_2(x)\le G_1(x),\; (a\le x\le b)$的从$a$到$b$的和的极限$\Lim_{n\to\infty}s_n=\Lim_{n\to\infty}S_n$作为上述$f(x)$从$a$到$b$的和的数量化定义.
\end{enumerate}
\end{solution}

\begin{blk}
    {定义1}若$f(x)$是$[a,b]$上的递增连续函数,将$[a,b]$等分成$n$段,分点坐标$x_i=a+\frac{i}{n}(b-a),\; i=0,1, 2,\ldots,n$,则存在两个阶梯函数:
\[    g_n (x) =f (x_{i-1}) ,\quad x_{i-1}\le  x< x_i\]
    和
\[    G_n (x)=f(x_i),\quad x_{i-1}<x\le x_i,\quad (i=1,2,\ldots,n)\]
满足下面的性质:
\begin{enumerate}
    \item 对于任何$x\in[a,b]$,$g_n(x)\le f(x)\le G_n(x)$
    \item 相应的阶梯函数从$a$到$b$的和
\[s_n=\frac{b-a}{n}\sum^n_{i=1}f(x_{i-1}),\qquad S_n=\frac{b-a}{n}\sum^n_{i=1}f(x_{i})  \]
\end{enumerate}
当$n\to\infty$时,$S_n-s_n\to 0$,那么$f(x)$从$a$到$b$的和是$s_n$与$S_n$的共同极限,即:
\begin{align*}
f(x)\text{从$a$到$b$的和}&=\frac{b-a}{n}\lim_{n\to\infty}\sum^n_{i=1}f(x_{i-1})\\
&=\frac{b-a}{n}\lim_{n\to\infty}\sum^n_{i=1}f(x_{i})
\end{align*}

同样地,如果$f(x)$是$[a,b]$上的递减连续函数.那么将$[a,b]$ $n$等分,分点坐标$x_i=a+\frac{i}{n}(b-a)$, 这时有两个阶梯函数
\[    g_n (x) =f (x_{i}) ,\quad x_{i-1}<  x\le x_i\]
    和
\[    G_n (x)=f(x_{i-1}),\quad x_{i-1}\le x< x_i,\quad (i=1,2,\ldots,n)\]
满足下面的性质:
\begin{enumerate}
    \item 对于任何$x\in[a,b]$,$g_n(x)\le f(x)\le G_n(x)$
    \item 相应的阶梯函数$g_n(x)$与$G_n(x)$相应的和
\begin{align*}
    s_n&=g_n\text{从$a$到$b$的和}=\frac{b-a}{n}\sum^n_{i=1}f(x_{i})\\
    S_n&=G_n\text{从$a$到$b$的和}=\frac{b-a}{n}\sum^n_{i=1}f(x_{i-1})  
\end{align*}
\end{enumerate}
当$n\to\infty$时,$S_n-s_n=\frac{b-a}{n}\left[f(b)-f(a)\right]\to 0$,即:
\[\Lim_{n\to\infty}s_n=\Lim_{n\to\infty}S_n\]
\end{blk}

\begin{blk}
    {定义2} 递减连续函数从$a$到$b$的和为上述夹逼阶梯函数$g_n(x)$和$G_n(x)$从$a$到$b$的和的共同极限,即  
\begin{align*}
    f(x)\text{从$a$到$b$的和}&=\lim_{n\to\infty} \frac{b-a}{n}\sum^n_{i=1}f(x_i)\\
    &=\lim_{n\to\infty} \frac{b-a}{n}\sum^n_{i=1}f(x_{i-1})\\
\end{align*}
\end{blk}

通常我们把递增或递减的函数合称为单调函数,常见的
函数$f(x)$都是分段单调连续的,例如,$y=\sin x$本身虽然不是单调的,但是它在$\left[-\frac{\pi}{2}+2k\pi,\frac{\pi}{2}+2k\pi\right]\; (k\in\mathbb{Z})$这些区间的每一段是递增的;而在$\left[\frac{\pi}{2}+2k\pi,\frac{3\pi}{2}+2k\pi\right]\; (k\in\mathbb{Z})$这些区间的每一段是递减的.因此,对于一般在$[a,b]$上连续的函数,如果存在有限个分点使得$f(x)$在每个分段上都是单调的,我们可以逐段取上、下夹逼阶梯函数,合起来作为定义在$[a,b]$上的函数$f(x)$的阶梯函数,于是得出
\[g_n (x)\le f(x)\le G_n(x),\qquad x\in [a,b]\]
又因为有限个在分段上趋于0的量的和仍趋于0, 所以
\[\text{“$G_n(x)$从$a$到$b$的和”}-\text{“$g_n(x)$从$a$到$b$的和”}\to 0\]

总结以上讨论,我们叙述为存在定理和定义如下:

\begin{blk}
    {定理} 设$y=f(x)$是一个定义在$[a,b]$上分段单调函数,则存在满足下列性质的两系列阶梯函数:
    \[g_n(x)\le f(x)\le G_n(x),\qquad x\in[a,b]\]
而且当$n\to 0$时,“$G_n(x)$从$a$到$b$的和,”与“$g_n(x)$从$a$到$b$的和”趋于共同极限.
\end{blk}


\begin{blk}
    {定义3} 设$f(x)$是$[a,b]$上的连续函数,如果 存在有限个分点:
\[a=a_0<a_1<\cdots<a_k<\cdots<a_1=b\]
使得$f(x)$在每个分段$[a_{k-1},a_k]$都是单调的,再将每个分段
    $[a_{k-1},a_k]$都$n$等分,则当$n\to\infty$时,有
\begin{align*}
f(x)\text{从$a$到$b$的和}&=\lim_{n\to\infty}\frac{a_1-a_0}{n}\sum^n_{i=1}f(x_i)+\lim_{n\to\infty}\frac{a_2-a_1}{n}\sum^{2n}_{i=n+1}f(x_i)+\\
&\cdots +\lim_{n\to\infty}\frac{a_\ell-a_{\ell-1}}{n}\sum^{\ell n}_{i=\ell(n-1)+1}f(x_i)
\end{align*}
\end{blk}

为了说明上述定义是合理的,我们就得证明上述$f(x)$从$a$到$b$的和与夹逼的阶梯函数列$\{G_n(x)\}$和$\{g_n(x)\}$的选取无关,其证明如下:

\begin{proof}
    设$\{\bar G_m(x)\}$和$\{\bar g_m(x)\}$是另外一组满足存在定理的上、下夹逼函数列,则由下述不等式
\[g_n(x)\le f(x)\le G_n(x),\quad \bar g_m(x)\le f(x)\le \bar G_m(x),\qquad a\le x\le b\]
即有
\[g_n(x)\le \bar G_m(x),\qquad \bar g_m(x)\le G_n(x)\]
所以由基本性质1,有
\begin{align*}
    s_n&=g_n(x)\text{从$a$到$b$的和}\le \bar G_m(x)\text{从$a$到$b$的和}=\bar S_m\\
    \bar s_m&=\bar g_m(x)\text{从$a$到$b$的和}\le G_n(x)\text{从$a$到$b$的和}=S_n\\
\end{align*}
所以
\[\lim_{n\to\infty}s_n=\lim_{m\to\infty}\bar S_m,\qquad \lim_{m\to\infty}\bar s_m\le \lim_{n\to\infty}S_n\]
但是
\[\lim_{n\to\infty}s_n\le \lim_{m\to\infty}\bar S_m=\lim_{m\to\infty}\bar s_m\le \lim_{n\to\infty} S_n=\lim_{n\to\infty}s_n\]
所以上述极限必须相等,即:
\[\lim_{n\to\infty}s_n= \lim_{m\to\infty}\bar S_m=\lim_{m\to\infty}\bar s_m=\lim_{n\to\infty} S_n\]
\end{proof}

\begin{example}
求$f(x)=2x-x^2$从0到2的和.
\end{example}

\begin{solution}
    由$f'(x)=2-2x$, $f''(x)=2$知$f(1)=2-1=1$是
$f(x)$的极大值,并且$y=2x-x^2$在$[0, 1]$上递增,在$[1,2]$上递减,于是,我们把区间$[0, 1]$和$[1, 2]$都$n$等分,设分点坐标$x_i=\frac{i}{n},\; i=0, 1, 2,\ldots,2n$, 即有
\[0=x_0<x_1<x_2<\cdots <x_n=1<x_{n+1}<x_{n+2}<\cdots <x_{2n}=2\]
且$\Delta x_i=x_i-x_{i-1}=\frac{1}{n}$,由定义3得到:
\begin{align*}
    f(x)&=2x-x^2\text{从0到2的和}\\
    &=\lim_{n\to\infty}\frac{1}{n}\sum^n_{i=1}\left[2\left(\frac{i}{n}\right)-\left(\frac{i}{n}\right)^2\right]+\lim_{n\to\infty}\frac{1}{n}\sum^{2n}_{i=n+1}\left[2\left(\frac{i}{n}\right)-\left(\frac{i}{n}\right)^2\right] \\
&=\lim_{n\to\infty}\frac{1}{n}\sum^{2n}_{i=1}2\left(\frac{i}{n}\right)-\lim_{n\to\infty}\sum^{2n}_{i=1}\left(\frac{i}{n}\right)^2\\
&=2\lim_{n\to\infty}\frac{1}{n^2}\sum^{2n}_{i=1}i-\lim_{n\to\infty}\frac{1}{n^3}\sum^{2n}_{i=1}i^2\\
&=2\lim_{n\to\infty}\frac{1}{n^2}\cdot \frac{(2n+1)2n}{2}-\lim_{n\to\infty}\frac{1}{n^3}\cdot \frac{2n(2n+1)(4n+1)}{6}\\
&=2\lim_{n\to\infty}\left(2+\frac{1}{n}\right)-\lim_{n\to\infty}\frac{1}{3}\left(2+\frac{1}{n}\right)\left(4+\frac{1}{n}\right)\\
&=4-\frac{8}{3}=\frac{4}{3}
\end{align*}
\end{solution}

\begin{ex}
\begin{enumerate}
    \item 已知质点的运动速度$v=t+4$, 试求质点在前10秒内所走的路程.
    \item 求$f(x)=x^3$在$1\le x\le 2$上的和.
\end{enumerate}
\end{ex}


\section{定积分的定义和基本性质}
\subsection{定积分定义}

设函数$f(x)$在$[a,b]$上连续,如果存在有限个分点
\[a=a_0<a_1<a_2<\cdots<a_{\ell-1}<a_{\ell}=b\]
使得$f(x)$在每个分段$[a_{k-1},a_k]\; (k=1, 2,\ldots,\ell)$上都是单调的,我们把$f(x)$从$a$到$b$的和叫做$f(x)$从$a$到$b$的\textbf{定积分},并记作
\[\int^b_a f(x) \dd x\]

这里,积分符号所使用的是长$S$形的求和号的变形,而从部分区间长$\Delta x_i=\frac{a_k-a_{k-1}}{n}$过渡到极限,则通过字母$\dd$来表示.

我们把数$a$与$b$称为\textbf{积分限}($a$称为\textbf{下限},$b$称为\textbf{上限}).区间$[a,b]$称为\textbf{积分区间},函数$f(x)$称为\textbf{被积函数},乘积$f(x)\dd x$称为\textbf{被积表达式},定积分符号下出现的字母$x$叫做\textbf{积分变量}.

上述定积分定义用定积分符号表示就是:
\[\int^b_a f(x) \dd x=\sum^t_{k=1}\int^{a_k}_{a_{k-1}}f(x)\dd x\]
其中$\Int^{a_k}_{a_{k-1}}f(x)\dd x$是$f(x)$为单调的第$k$个分段$[a_{k-1},a_k]$上的夹逼阶梯函数$g_n(x)$和$G_n(x)$从$a_{k-1}$到$a_k$的和的共同极限,即:
\begin{align*}
    \int^{a_k}_{a_{k-1}}f(x)\dd x&=\lim_{n\to \infty}\frac{a_k-a_{k-1}}{n}\sum^n_{i=1}f(x_{i-1})\\
    &=\lim_{n\to \infty}\frac{a_k-a_{k-1}}{n}\sum^n_{i=1}f(x_{i})
\end{align*}

定积分定义的另一种表述形式是:设函数$f(x)$在$a\le x\le b$上分段单调连续,如果存在两系列上、下夹逼阶梯函数$\{g_n(x)\}$, $\{G_n(x)\}$, 使得
\[g_n (x)\le f(x)\le G_n(x),\qquad a\le x\le b\]
并且$g_n(x)$从$a$到$b$的和$s_n$与$G_n(x)$从$a$到$b$的和$S_n$具有相同的
极限,这个极限叫做$f(x)$从$a$到$b$的定积分$\Int^b_a f(x) \dd x$.

综上所述,在积分符号中,我们只对$a<b$, 即积分下限小于积分
上限的情形给出了定义,若$a>b$,我们定义
\[\int^a_b f(x) \dd x=-\int^b_a f(x) \dd x\]
此外,由于定积分可以解释为曲边梯形的面积,自然可以定
\[\int^a_a f(x) \dd x=0\]
作了这样的规定之后,不论$a<b$, $a>b$或$a=b$, 定积分都有意义了.


\begin{ex}
\begin{enumerate}
    \item 求证 $\Int_{a}^{b} k x\dd x=\frac{k b^{2}}{2}-\frac{k a^{2}}{2} \quad(a<b)$
    \item  求 $\Int_{8}^{0}\left(x^{2}-4 x\right) \dd x$
    \item  求 $\Int_{-1}^{2}\left(x^{3}-3 x\right) \dd x$
\end{enumerate}
\end{ex}

\subsection{逼近法求曲线形的面积}

任意一条曲线围成的图形(图\ref{fig:chapter3_sec2_arbitrary_shape})常常可以用两组互相垂直的直线把它分成若干部分,每一部分都是一个曲边梯形(图\ref{fig:chapter3_sec2_good_piece}), 在这里并不排除下述情形(图\ref{fig:chapter3_sec2_degenerated_piece}):两条平行的边中有一条缩成了一点,因而曲边梯形变成了曲边三角形,这样一来,我们的问题就化成了求曲边梯形面积的问题.

\begin{figure}[htp]
    \centering
    \begin{minipage}[t]{0.45\textwidth}
        \centering
        \begin{tikzpicture}[>=latex, scale=1]
    \draw(-2,3)--(2,3);
    \draw(-2,2)--(2,2); 
    \draw(-.5,0)--(-.5,4);
    \draw(.5,0)--(.5,4); 
    \node at (0,0){};
    \draw plot[smooth] coordinates{
 (-0.32,3.56)(-0.82,3.36)(-1.12,2.9)(-1.42,1.72)(-1.26,0.9)(-1.08,0.78)(-0.64,0.64)(-0.56,0.64)(-0.5,0.62)(-0.42,0.62)(-0.36,0.6)(-0.3,0.58)(0.1,0.5)(0.18,0.5)(0.24,0.52)(0.48,0.7)(0.64,0.86)(0.7,0.9)(1.36,1.62)(1.42,1.7)(1.36,3.16)(0.96,3.56)(0.84,3.66)(0.78,3.7)(0.6,3.8)(0.54,3.82)(0.34,3.82)(-0.32,3.56)
};
        \end{tikzpicture}
        \caption{}\label{fig:chapter3_sec2_arbitrary_shape}
        \end{minipage}
    \begin{minipage}[t]{0.25\textwidth}
    \centering
    \begin{tikzpicture}[>=latex, scale=1]
    \draw(0,1.3)--(0,0)--(2.5,0)--(2.5,1.8);
    \node at (0,0) {};
    \draw plot[smooth] coordinates{(0,1.3)(1.4,2)(2.5,1.8)};
    \end{tikzpicture}
    \caption{}\label{fig:chapter3_sec2_good_piece}
    \end{minipage}
    \begin{minipage}[t]{0.25\textwidth}
    \centering
    \begin{tikzpicture}[>=latex, scale=1]
    \draw(0,2)--(0,0)--(1.5,0);
    \node at (0,0){};
    \draw plot[smooth] coordinates{(0,2)(0.5,1.8)(1,1.3)(1.5,0)};
    \end{tikzpicture}
    \caption{}\label{fig:chapter3_sec2_degenerated_piece}
    \end{minipage}
  \end{figure}

\begin{blk}
  {命题} 设$y=f(x)$是定义在$a\le x\le b$的分段单调连续函数,而且$f(x)\ge 0$, 则区域$R=\{a\le x\le b,\; 0\le y\le f(x)\}$的
面积等于
\[\int^b_a f (x) \dd x \]  
\end{blk}

\begin{figure}[htp]
    \centering
\begin{tikzpicture}[>=latex]
    \draw[->, thick](-.5,0)--(7,0)node[right]{$x$};
    \draw[->, thick](0,-.5)--(0,5)node[right]{$y$};
\draw[domain=.5:6, samples=100, very thick]plot(\x, {1.5*sin(1.2*(\x-1) r)+3});
\draw(.5,0)node[below]{$a$} rectangle  (1.1, .18+3);
\draw(1.1,0) rectangle  (1.8, 1.23+3);
\draw(1.8,0) rectangle  (2.5, 1.46+3);
\draw[pattern=north east lines](2.5,0) rectangle  (3.5, .212+3);
\draw[pattern=north east lines](3.5,0) rectangle  (4.2, -.965+3);
\draw[pattern=north east lines](4.2,0) rectangle  (5, -1.49+3);
\draw(5,0) rectangle  (5.5, -1.16+3);
\draw(5.5,0) rectangle (6, -.419+3);

\draw[pattern=north east lines](.5,0) rectangle  (1.1, -.847+3);
\draw[pattern=north east lines](1.1,0) rectangle  (1.8, .18+3);
\draw[pattern=north east lines](1.8,0) rectangle  (2.5, 1.23+3);
\draw(2.5,0) rectangle  (3.5, 1.46+3);
\draw(3.5,0) rectangle  (4.2, .212+3);
\draw(4.2,0) rectangle  (5, -.965+3);
\draw[pattern=north east lines](5,0) rectangle  (5.5,-1.49+3);
\draw[pattern=north east lines](5.5,0) rectangle (6, -1.16+3);

\node at (6,0)[below]{$b$};
\node at (0,0)[below left]{$O$};
\node at (4.5,3.2)[above]{$y=f(x)$};

\end{tikzpicture}
    \caption{}\label{fig:chapter3_sec2_riemann_sum}
\end{figure}


\begin{proof}
    由$\Int^b_a f (x) \dd x$的定义得知,存在两系列阶梯函数
$\{g_n(x)\}$和$\{G_n(x)\}$, 满足下面的性质:
\[g_n(x)\le f (x) \le G_n(x)\]
而且$G_n(x)$从$a$到$b$的和与$g_n(x)$从$a$到$b$的和趋于共同的极限$\Int^b_a f (x) \dd x$. 令
\begin{align*}
    R_n&=\{a\le x\le b,\; 0\le y\le g_n(x)\}\\
    R'_n&=\{a\le x\le b,\; 0\le y\le G_n(x)\}
\end{align*}
它们都是由高高低低的狭长方形所组成的区域(图\ref{fig:chapter3_sec2_riemann_sum}),则:
\[R_n\subset \text{曲边梯形区域 }R \subset R'_n\]
而且
\[R_n\text{的面积}=g_n(x)\text{从$a$到$b$的和}s_n,\qquad R'_n\text{的面积}=G_n(x)\text{从$a$到$b$的和}S_n\]
即有
\[s_n<\text{曲边梯形区域$R$的面积}A<S_n\]
由定积分定义有:
\[\lim_{n\to\infty}s_n=\lim_{n\to\infty}S_n=\Int^b_a f (x) \dd x\]
又曲边梯形面积$A=(f,\; [a,b])$也是夹逼在$s_n$和$S_n$中间的实数,所以
\[A=(f,\; [a,b])=\Int^b_a f (x) \dd x\]
\end{proof}

\begin{ex}
\begin{enumerate}
    \item 求$\Int^1_{-2}e^x\dd x$. 

提示:$\frac{e^{\Delta x}-1}{\Delta x}=1$.    

\item 求在$x=0$与$x=\pi$间正弦曲线$y=\sin x$与$Ox$轴所包的面积.
\end{enumerate}
\end{ex}    

\subsection{定积分的基本性质}

为简单起见,我们约定以下被积函数在积分区间上连续并且可以分段单调,即$f(x)$是在$[a,b]$上连续并且只有有限个极大值和极小值的函数.

\begin{blk}{性质1}
若$a<b<c$,那么,
\[\int^b_a f(x)\dd x+\int^c_b f(x)\dd x=\int^c_a f(x)\dd x\]
\end{blk}

\begin{proof}
因为$\Int^b_a f(x)\dd x$存在,故对于在$[a,b]$上的
$f(x)$, 存在一组上、下夹逼函数列$\{g_n(x)\}$和$\{G_n(x)\}$, 使得
\[g_n (x) \le f (x) \le G_n (x),\qquad x\in [a,b]\]
且相应的阶梯函数在$[a,b]$上的和$s_n$与$S_n$适合
\begin{equation}
    s_n<\int^b_a f(x)\dd x<S_n,\qquad S_n-s_n\to 0
\end{equation}
对于在$[b,c]$上的$f(x)$, 同样得到
\[\bar g_n (x) \le f (x) \le \bar G_n (x),\qquad x\in [b,c]\]
且
\begin{equation}
    \bar  s_n<\int^c_b f(x)\dd x<\bar S_n,\qquad \bar S_n-\bar s_n\to 0
\end{equation}
$(3.4)+(3.5)$得到
\begin{equation}
    s_n+ \bar  s_n<\int^b_a f(x)\dd x+\int^c_b f(x)\dd x<S_n+\bar S_n
\end{equation}

另一方面,对于函数$f(x)\; (a\le x\le c)$存在下面一组阶梯函数列:
\[\tilde g_n(x)=\begin{cases}
    g_n(x),  &  a\le x\le b\\
    \max\left(g_n(b), \bar g_n(b)\right),  &  x=b\\
    \bar g_n(x), & b<x\le c
\end{cases}\]
\[\tilde G_n(x)=\begin{cases}
    G_n(x),  &  a\le x\le b\\
    \max\left(G_n(b), \bar G_n(b)\right),  &  x=b\\
    \bar G_n(x), & b<x\le c
\end{cases}\]

由性质2,阶梯函数列$\tilde g_n(x)$与$\tilde G_n(x)$的从$a$到$c$的和分别是$s_n+\bar s_n$和$S_n+\bar S_n$.因为$\tilde g_n(x)$与$\tilde G(x)$满足条件:
\begin{enumerate}
    \item $\tilde g_n(x)\le f(x)\le \tilde G(x),\quad a\le x\le c$
    \item 
    \begin{align*}
    \lim_{n\to\infty}\left[(S_n+\bar S_n)-(s_n+\bar s_n)\right]&=\lim_{n\to\infty}\left[(S_n-s_n)+(\bar S_n-\bar s_n)\right]\\
    &=\lim_{n\to\infty}(S_n-s_n)+\lim_{n\to\infty}(\bar S_n-\bar s_n)=0
    \end{align*}
\end{enumerate}
所以根据定积分定义得到
\[\lim_{n\to\infty}(S_n+\bar S_n)=\lim_{n\to\infty}(s_n+\bar s_n)=\int^c_a f(x)\dd x \]

因为$\Int^c_a f(x)\dd x $与$\Int^b_a f(x)\dd x +\Int^c_b f(x)\dd x $都是被$s_n+\bar s_n$与$S_n+\bar S_n$所夹逼的唯一实数,所以
\[\Int^c_a f(x)\dd x=\Int^b_a f(x)\dd x +\Int^c_b f(x)\dd x\]
\end{proof}

\begin{blk}{性质2}
若$f(x)=f_1(x)+f_2(x)$,则:
\[\Int^b_a f(x)\dd x=\Int^b_a f_1(x)\dd x +\Int^b_a f_2(x)\dd x\]
\end{blk}

\begin{proof}
设$\{g_n(x)\}$与$\{G_n(x)\}$是$f_1(x)$在$[a,b]$上的上、下夹逼阶梯函数列,又$\{\bar g_n(x)\}$与$\{\bar G_n(x)\}$是$f_2(x)$在$[a,b]$上的上、下夹逼阶梯函数列.$s_n, S_n, \bar s_n, \bar S_n$是相应的阶梯函数从$a$到$b$的和.于是由
$\Int^b_a f_1(x)\dd x$和$\Int^b_a f_2(x)\dd x$的存在,得:
\begin{align}
g_n(x)\le f_1(x)\le G_n(x),\qquad a\le x\le b\\
\bar g_n(x)\le f_2(x)\le \bar G_n(x),\qquad a\le x\le b
\end{align}
并且
\begin{align}
s_n<\Int^b_a f_1(x)\dd x<S_n,\quad \text{且 } S_n-s_n\to 0\\
\bar s_n<\Int^b_a f_2(x)\dd x<\bar S_n,\quad \text{且 } \bar S_n-\bar s_n\to 0
\end{align}
由$(3.9)+(3.10)$,得到
\begin{equation}
   s_n+ \bar s_n <\Int^b_a f_1(x)\dd x+\Int^b_a f_2(x)\dd x<S_n+\bar S_n
\end{equation}
另一方面,由$(3.7)+(3.8)$,得到
\[g_n(x)+\bar g_n(x)\le f_1(x)+ f_2(x)\le G_n(x)+\bar G_n(x),\qquad a\le x\le b\]
而且,$g_n(x)+\bar g_n(x)$与$G_n(x)+\bar G_n(x)$也是在$[a,b]$上的阶梯函数,我们要说明它们是$f_1(x)+ f_2(x)$在$[a,b]$上的一组夹逼阶梯函数列.

因为:
\begin{align*}
    g_n(x)+\bar g_n(x)\text{从$a$到$b$的和}&=\left[ g_n(x)\text{从$a$到$b$的和}\right]+\left[\bar g_n(x)\text{从$a$到$b$的和}\right]\\
    &=s_n+\bar s_n\\
    G_n(x)+\bar G_n(x)\text{从$a$到$b$的和}&=\left[ G_n(x)\text{从$a$到$b$的和}\right]+\left[\bar G_n(x)\text{从$a$到$b$的和}\right]\\
    &=S_n+\bar S_n
\end{align*}
而且
\begin{equation}
    \left(S_n+\bar S_n\right)-\left(s_n+\bar s_n\right)=\left(S_n-s_n\right)+\left(\bar S_n-\bar s_n\right)\to 0
\end{equation}
所以,由$\Int^b_a \bigl(f_1(x)+ f_2(x)\bigr)\dd x$的定义,得到
\begin{equation}
    \lim_{n\to\infty}\left(S_n+\bar S_n\right)=\lim_{n\to\infty}\left(s_n+\bar s_n\right)=\Int^b_a \bigl(f_1(x)+ f_2(x)\bigr)\dd x
\end{equation}
由(3.10)和(3.13),$\Int^b_a \bigl(f_1(x)+ f_2(x)\bigr)\dd x$与$\Int^b_a f_1(x)\dd x + \Int^b_a  f_2(x)\dd x$
是被$s_n+\bar s_n$与$S_n+\bar S_n$所夹逼的唯一实数,所以
\[\Int^b_a \bigl(f_1(x)+ f_2(x)\bigr)\dd x=\Int^b_a f_1(x)\dd x + \Int^b_a  f_2(x)\dd x\]
\end{proof}

\begin{blk}{性质3}
\[\Int^b_a kf(x)\dd x =k\Int^b_a f(x)\dd x \]
\end{blk}

此法则的证明大致和性质2的证明相同,即当$\{g_n(x)\}$和$\{G_n(x)\}$上、下夹逼$f(x)$时,那么在$k>0$的情形,$\{kg_n(x)\}$和$\{kG_n(x)\}$上、下夹逼$kf(x)$; 在$k<0$的情形,$\{kg_n(x)\}$和$\{kG_n(x)\}$上、下夹逼$kf(x)$. 我们把证明的过程留给读者去写.

\begin{blk}{性质4}
    若$m\le f(x)\le M$, $a\le x\le b$,那么
    \[m(b-a)\le \Int^b_a f(x)\dd x \le M(b-a) \]
    \end{blk}

\begin{proof}
    设$\{g_n(x)\}$与$\{G_n(x)\}$是$f(x)$在$[a,b]$上的夹逼阶梯函数列,$s_n$与$S_n$是相应的阶梯函数在$[a,b]$上的和,根据积分的定义,有
\[\lim_{n\to\infty}s_n=\lim_{n\to\infty}S_n=\int^b_a f(x)\dd x\]
换言之,任给一个正数$\varepsilon$,存在自然数$N$,使得当$n<N$时,有
\begin{equation}
    S_n<\int^b_a f(x)\dd x+\varepsilon
\end{equation}
成立.

又由于$G_n(x)\ge f(x)\ge m,\quad x\in[a,b]$,根据性质1,得到:
\begin{equation}
    S_n\ge m(b-a)
\end{equation}
所以,当$n>N$时,由(3.14)和(3.15),有
\[m(b-a)\le S_n<\int^b_a f(x)\dd x+\varepsilon\]
即
\begin{equation}
    m(b-a)<\int^b_a f(x)\dd x+\varepsilon
\end{equation}
成立.

因为这个不等式(3.16)对于每个正数$\varepsilon$都成立,所以
\[m(b-a)\le \int^b_a f(x)\dd x\]
同样证明,得到
\[M(b-a)\ge \int^b_a f(x)\dd x\]
因此
\[m(b-a)\le \int^b_a f(x)\dd x\le M(b-a)\]
\end{proof}

\begin{figure}[htp]
    \centering
\begin{tikzpicture}[>=latex]
\draw[->, thick](-1.5,0)--(3,0)node[right]{$x$};
\draw[->, thick](-1,-.5)--(-1,4)node[right]{$y$};
\node at (-1,0)[below left]{$O$};
\draw[domain=-.1:2.3, samples=50, very thick]plot(\x, {\x*(\x-1)*(\x-2)+2});
\draw(-.1,0)node[below]{$a$} rectangle (2.3,2.9);
\node at (2.3,0)[below]{$b$};
\draw(1.577,0)node[below]{$x_0$}--(1.577,1.615);
\draw(-.1,1.615)--(2.3,1.615);
\draw[dashed](-1,1.615)node[left]{$m$}--(-.1,1.615);
\draw[dashed](-1,2.9)node[left]{$M$}--(-.1,2.9);

\node at (1.1,2.5){$y=f(x)$};
\end{tikzpicture}
    \caption{}\label{fig:chapter3_sec2_y=f(x)_bounded}
\end{figure}

从几何图形来看,以曲线$y=f(x)$为一边而以线段$[a,b]$为底边的曲边梯形界于以$[a,b]$为底边,高分别等于$m$和$M$的两个矩形之内,
故曲边梯形的面积$\Int^b_a f(x)\dd x$
在两个矩形面积$m(b-a)$和$M(b-a)$之间(图\ref{fig:chapter3_sec2_y=f(x)_bounded}).

这个性质是说分段单调的连续函数$f(x)$在区间$[a,b]$上的定积分是有界的.

\begin{blk}{性质5}
如果$f(x)\le \varphi(x),\; a\le x\le b$,那么:
\[\int^b_a f(x)\dd x\le \int^b_a \varphi(x)\dd x\]
\end{blk}

\begin{proof}
    设$\{g_n(x)\}$与$\{G_n(x)\}$是$f(x)$在$[a,b]$上的夹逼阶梯函数列,又$\{\bar g_n(x)\}$与$\{\bar G_n(x)\}$是$\varphi(x)$在$[a,b]$上的夹逼阶梯函数列,
又$s_n, S_n, \bar s_n, \bar S_n$是相应的阶梯函数从$a$到$b$的和.于是根据定积分的定义,有:
\[s_n<\int^b_a f(x)\dd x<S_n,\qquad \bar s_n<\int^b_a \varphi(x)\dd x<\bar S_n\]
当$n\to\infty$时,得到:
\begin{align*}
\lim_{n\to\infty}S_n =\lim_{n\to\infty}s_n =\int^b_a f(x)\dd x\\
\lim_{n\to\infty}\bar S_n =\lim_{n\to\infty}\bar s_n =\int^b_a \varphi(x)\dd x\\
\end{align*}
换言之,任给正数$\varepsilon$,存在$N_1$,使得当$n>N_1$时,有
\[s_n>\int^b_a f(x)\dd x-\frac{\varepsilon}{2}\]

当$n>N_2$时,有
\[\bar s_n<\int^b_a \varphi(x)\dd x+\frac{\varepsilon}{2}\]

因为$\bar G_n(x)>\varphi(x)>f(x)>g_n(x),\quad x\in[a,b]$,根据性质1,得到:$\bar S_n=s_n$.

所以,当$n>\max(N_1,N_2)$时,有
\[\int^b_a f(x)\dd x<s_n+\frac{\varepsilon}{2}<\bar S_n+\frac{\varepsilon}{2}<\int^b_a \varphi(x)\dd x+\varepsilon\]
因为这个不等式对于每个正数$\varepsilon$都成立,所以必然有
\[\int^b_a f(x)\dd x\le \int^b_a \varphi(x)\dd x\]
\end{proof}

\begin{blk}{性质6:定积分中值定理}
   设函数$f(x)$在闭区间$[a,b]$上分段单调连续,又
$m=f(c)$, $M=f(d)$分别是$f(x)$在$[a,b]$上的最小值和最大值,则在$[a,b]$上至少存在一点$\xi$, 使得下面的等式成立:
\[\int^b_a f (x) \dd x=f(\xi) (b-a)\] 
\end{blk}

\begin{proof}
    因为$m\le f (x) \le M,\quad x\in [a,b]$, 
由性质4, 即得
\[m (b-a)\le \int^b_a f (x) \dd x\le M(b-a)\]
上面不等式的两端各除以$(b-a)$, 得
\[f(c)=m\le \frac{\int^b_a f (x) \dd x}{b-a}\le M=f(d)\]
因为$f(x)$在$[a,b]$上连续,再由连续函数的中间值定理,必存在一个$\xi\in(c,d)$, 使得
\[f(\xi)=\frac{1}{b-a}\int^b_a f (x) \dd x\]
两边再乘以$(b-a)$,得
\[\int^b_a f (x) \dd x=f(\xi)(b-a)\]
这就是我们所要证明的.
\end{proof}

这个公式的几何意义是:以线段$[a,b]$为底边,以曲线$y=f(x)$为曲边的曲边梯形,它的面积等于同一底边
而高为$f(\xi)$的一个矩形的面积(图\ref{fig:chapter3_sec2_integral_mean_value}).因此,$f(\xi)$称为曲边梯形的平均高度.

我们也称
\[f(\xi)=\frac{1}{b-a}\int^b_a f (x) \dd x\]
为$f(x)$在$[a,b]$上的平均值.

\begin{figure}[htp]
    \centering
    \begin{minipage}[t]{0.48\textwidth}
    \centering
    \begin{tikzpicture}[>=latex, xscale=.8]
\draw[->, thick](-.5,0)--(5,0)node[right]{$x$};
\draw[->, thick](0,-.5)--(0,4)node[right]{$y$};
\node at (0,0)[below left]{$O$};
\draw[domain=.7:4, very thick, samples=100]plot(\x, {ln(\x)+1.5})node[above]{$y=f(x)$};
\draw[dashed](.7,0)node[below]{$a$} rectangle (4,2.3);
\draw[dashed](2.23,0)node[below]{$\xi$}--(2.23,2.3)node[below right]{$f(\xi)$};
\draw[dashed](4,2.89)--(4,2.3);
\node at (4,0)[below]{$b$};

    \end{tikzpicture}
    \caption{}\label{fig:chapter3_sec2_integral_mean_value}
    \end{minipage}
    \begin{minipage}[t]{0.48\textwidth}
    \centering
    \begin{tikzpicture}[>=latex, scale=.5]
  \draw[->, thick](-2,0)--(6,0)node[right]{$x$};
\draw[->, thick](0,-8.5)--(0,2)node[right]{$y$};
\node at (0,0)[below left]{$O$};
\draw[domain=-1:4, very thick, samples=100]plot(\x, {2*\x-\x*\x});
\fill[pattern=north east lines, domain=2:4, very thick, samples=100]plot(\x, {2*\x-\x*\x})--(4,0)node[above]{4}
--(2,0)node[above]{2};
\draw(1,0)node[below]{1}--(1,.2);
    \end{tikzpicture}
    \caption{}\label{fig:chapter3_sec2_example3.7}
    \end{minipage}
  \end{figure}

\begin{example}
求$\Int^4_2 (2x-x^2)\dd x$.
\end{example}


\begin{solution}
    根据本节的定积分计算法则得:
\begin{align*}
    \Int^4_2 (2x-x^2)\dd x&=\Int^4_2 2x\dd x+\Int^4_2 (-x^2)\dd x\\
    &=2\Int^4_2 x\dd x-\Int^4_2 x^2\dd x
\end{align*}
利用例3.4的计算结果,得
\[\Int^4_2 x\dd x=\frac{4^2-2^2}{2}=6,\qquad \Int^4_2 x^2\dd x=\frac{4^3-2^3}{3}=18\frac{2}{3}\]
因此
\[\Int^4_2 (2x-x^2)\dd x=2\x 6-18\frac{2}{3}=-6\frac{2}{3}\]
它的几何意义是图\ref{fig:chapter3_sec2_example3.7}中阴影区域的有号面积.
\end{solution}

\begin{example}
  设$f(x)=\begin{cases}
      x^2,& 0\le x\le 1\\
      1,& 1<x\le 2
  \end{cases}$,  求$f(x)$在$[0,2]$上的平均值.
\end{example}

\begin{solution}
由平均值定义,再利用性质1,有    
\begin{align*}
    f(\xi)&=\frac{1}{b-a}\int^b_a f(x)\dd x=\frac{1}{2}\int^2_0 f(x)\dd x\\
    &=\frac{1}{2}\left[\int^1_0 f(x)\dd x+\int^2_1 f(x)\dd x\right]\\
    &=\frac{1}{2}\left[\int^1_0 x^2\dd x+\int^2_1 1\dd x\right]\\
    &=\frac{1}{2}\left[\frac{1}{3}+1\right]=\frac{2}{3}
\end{align*}
\end{solution}

\begin{ex}
\begin{enumerate}
    \item 利用以前定积分的结果和定积分的性质,计算下列定积分:
\begin{enumerate}
    \item $\Int^3_{-1}(2x^2-4x)\dd x$
    \item 若$f(x)=\begin{cases}
        x,& 0\le x<1\\
        x-2,& 1\le x\le 2
    \end{cases}$,则$\Int^2_0 f(x)\dd x=?$
    \item 若$f(x)=\begin{cases}
        x,& 0\le x\le 1\\
        x-2,& 1< x\le 2
    \end{cases}$,则$\Int^2_0 f(x)\dd x=?$
\item 若$g(t)=2t^2+|t|-1,\; -1\le t\le 1$,则$\Int^1_{-1} g(t)\dd t=?$
\end{enumerate}

\item 将图中阴影部分的面积用定积分表示.
\item 设$\ell(t)=mt+b$, ($m,t$是常数),$t\in [c,d]$,

求证:$\Int^d_c (mt+b)\dd t=\frac{\ell(c)+\ell(d)}{2}(d-c)$
\item 证明:
\begin{enumerate}
    \item 若$0<x<10$,则$\frac{1}{1016}\le\frac{1}{x^3+16}\le \frac{1}{16}$
    \item $\frac{5}{508}\le \Int^{10}_0\frac{1}{x^3+16}\dd x\le \frac{5}{8}$
\end{enumerate}
\end{enumerate}
\end{ex}

\begin{figure}[htp]
    \centering
\begin{tikzpicture}[>=latex]
    
\begin{scope}[scale=.6]
\draw[->](-2,0)--(4,0)node[right]{$x$};
\draw[->](0,-3)--(0,7)node[right]{$y$};
\fill[domain=-1:3, samples=50, pattern=north east lines]plot(\x,{2*(\x-1)^2-2})--(3,0)--(-1,0)--(-1,6);
\draw[domain=-1:3, samples=50, very thick]plot(\x,{2*(\x-1)^2-2})node[above]{$y=2x^2-4x$};
\node at (0,0)[above right]{$O$};
\node at (-1,0)[below]{$-1$};
\node at (3,0)[below]{$3$};
\node at (-1,3)[left]{$x=-1$};
\node at (3,3)[right]{$x=3$};
\node at (1,-2)[below]{$(1,-2)$};
\node at (1,-3.8){(a)};
\end{scope}
\begin{scope}[xshift=5cm, yshift=-.8cm, scale=.9]
\draw[->](-1,0)--(4,0)node[right]{$x$};
\draw[->](0,-1)--(0,5.5)node[right]{$y$};
\node at (0,0)[below left]{$O$};
\draw[domain=0:2.2, samples=50, very thick]plot(\x,{\x^2})node[above]{$y=x^2$};
\draw[thick](0,0)--(3.5,3.5)node[right]{$y=x$};
\node at (1,0)[below]{$1$};
\node at (2,0)[below]{$2$};
\draw[dashed](0,1)node[left]{1}--(1,1)--(1,0);
\fill[domain=0:1, samples=50, pattern=north east lines]
plot(\x,{\x^2})--(2,2)--(2,0)--(0,0);
\node at (2,-1.5){(b)};
\end{scope}

\end{tikzpicture}
    \caption*{第2题}
\end{figure}



\subsection*{习题3.2}
\begin{enumerate}
    \item 计算下列定积分:
\begin{multicols}{3}
\begin{enumerate}
    \item $\Int^1_4 |x|\dd x$
    \item $\Int^2_4 |x|\dd x$
    \item $\Int^4_1 |x|\dd x$
\end{enumerate}
\end{multicols}
\item 求$\Int^b_a |x|\dd x$.

提示:分$a<b<0$, $a<0<b$, $0<a<b$三种情况讨论.
\item 证明:
\begin{enumerate}
    \item 若$\frac{\pi}{4}\le x\le\frac{\pi}{2}$,则$\frac{2}{\pi}\le \frac{\sin x}{x}\le \frac{2\sqrt{2}}{\pi}$
    \item $\frac{1}{2}\le \Int^{\tfrac{\pi}{2}}_{\tfrac{\pi}{4}}\frac{\sin x}{x}\dd x\le \frac{\sqrt{2}}{2}$
\end{enumerate}
\item 已知作用在作直线运动的质点上的力是$F=s^2+1$, 试求从距离1到10之间所作的功.
\item 求$y=\begin{cases}
    A\sin\frac{2\pi t}{T}, & 0\le t\le \frac{T}{2}\\
    0,& \frac{T}{2}\le t\le T
\end{cases}$ 在$[0,T]$上的平均值.
\end{enumerate}

\chapter{微积分学基本定理}
在前两章中,我们分别引入了函数的变率(导数),函数的和(定积分)的基本概念,本章将研究函数的导函数与函数的求和函数这两者之间的互逆关系,并说明我们可以用求导函数的逆运算方法来计算定积分。

\section{微积分学基本定理}
\subsection{导函数与求和函数}
函数$f(x)$在点$x_0$处的变化率(导数)的定义是
\[f' (x_0) =\lim_{h\to 0}\frac{f (x_0+h) -f (x_0)}{h}\]
显然。$f(x_0)$的值与$f(x)$在点$x_0$的值以及在点$x_0$的邻近的函数值有关,当点$x_0$在$(a,b)$内变化时,$f(x_0)$也跟着变化,那么$f(x_0)$便是一个新函数称为$f(x)$的导函数。计算一个函数的导函数是一件比较简便的事情。

定积分$\int^b_a f(x)\dif x$的定义是把区间$[a,b]$无限细分
而得到上下夹逼阶梯函数的和的共同极限,其几何意义是曲线$y=f(x)$和直线$x=a$, $x=b$, 及$y=0$所围成的区域的有号面积。

假如我们考虑$f(x)$在一个变动的区间$[a,x]$上的和,
即让区间的左端点固定,右端点变动,则
\[S_f(x)=\text{函数$f$从$a$到$x$的和}=\int^x_a f (x) \dif x\]
可以看作上限变量$x$的函数,在这里,积分符号中的$x$既表示积分变量,又表示积分上限,容易混淆,因此,为了区别起见,我们用字母$t$来代表积分变量,这样上式就写成
\[S_f (x)=\int^x_a f(t)\dif t\]

和函数$S_f(x)$在$x=x_0$处的值$S_f(x_0)$的几何意义就是曲线$y=f(t)$, 直线$t=a$, $t=x_0$, $y=0$所围成的区域的有号面积,它是随区域的变动界线$t=x_0$的变动而变动的(图4.1)。



例如,当变动界限(积分上限)在图中的$PM$位置时,则
\[S_f(x_0)=\int^{x_0}_a f(t)\dif t=ACB\text{的面积}-CPM\text{的面积}\]

当变动界限在图中的$P'M'$位置时,则
\[\begin{split}
    S_f(x'_0)&=\int^{x_0'}_a f(t)\dif t=-\int^a_{x_0'} f(t)\dif t\\
    &=-\left\{-P'DM'\text{的面积}+DAB\text{的面积}  \right\}\\
    &=P'DM'\text{的面积}-DAB\text{的面积}
\end{split}\]

例如,折线函数
\[g(x)=\begin{cases}
    \frac{1}{2}(x-1) & x\in[1,2]\\
    \frac{3}{2}(x-2)+\frac{1}{2}(2-1) & x\in [2,3]\\
    (x-3)+\frac{1}{2}(2-1)+\frac{3}{2}(3-2) & x\in [3,4]
\end{cases}\]
的导函数(除去在折线段的那些交接点处不作定义外)是阶梯函数
\[g'(x)=\begin{cases}
    \frac{1}{2} & x\in [1,2)\\
    \frac{3}{2} & x\in (2,3)\\
    1& x\in (3,4]
\end{cases}\]
反过来,该阶梯函数的和函数,是上述折线函数$g(x)$, 我们有如下的图解关系:
\begin{center}
    \begin{tikzpicture}[>=latex]
        \node (A) at (0,0) {\{折线函数\}};
        \node (B) at (5,0) {\{阶梯函数\}};
\draw[->](A) to [bend left=30]node[above]{求导} (B);
\draw[->](B) to [bend left=30]node[above]{求和} (A);

    \end{tikzpicture}
\end{center}

上述简明的例子表明“微分”与“积分”(或求函数由$a$到$b$的和)之间的运算关系应该是互逆的。

\subsection{微积分学基本定理}

\begin{blk}
  {定理1} 设$f(t)$是在$[a,b]$上的分段单调连续函数,又它的和函数是
\[S_f(x)=\int^x_a f (t) \dif t\]
那么
\[\frac{\dif }{\dif x}S_f(x)=\frac{\dif }{\dif x}\int^x_a f (t) \dif t=f(x),\qquad a\le x\le b\]
\end{blk}






















\end{document}
